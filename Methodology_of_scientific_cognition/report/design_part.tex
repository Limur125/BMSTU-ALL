\chapter{Традиционное планирование и организация научных исследований}

\section{Планирование и организация теоретического исследования}

Теоретическое знание --- это сформулированные общие для какой-либо предметной научной области закономерности, позволяющие объяснить ранее открытые факты и эмпирические закономерности, а также
предсказать и предвидеть будущие события и факты. \cite{lit6}

Теоретическое знание трансформирует результаты, полученные на
стадии эмпирического познания, в более глубокие обобщения, вскрывая
сущности явлений, закономерности возникновения, развития и изменения изучаемого объекта. \cite{lit6}

Существуют различия между эмпирическим и теоретическим знанием. Например, газовые законы Бойля-Мариотта, Шарля и Гей-Люссака --- это эмпирические законы, а обобщение этих газовых законов
на основе молекулярно-кинетической теории, модели идеального газа,
уравнение Клайперона-Менделеева --- это теоретическое знание. \cite{lit6}

Теоретическое исследование начинается с поиска. Выясняется, какая концепция, теория или предметная область могут объединить и собрать воедино все наработанные эмпирические результаты или их
большую часть. Нередко бывает, что часть результатов не ложится в
единое русло и их приходится отбрасывать. Но подчас оказывается, что
чего-то из необходимых эмпирических результатов недостаёт и эмпирическую часть исследования следует продолжить. \cite{lit6}

Когда предметная область определена исследователем, начинается
процесс построения логической структуры теории, концепции и т.п.  \cite{lit6}

Процесс построения логической структуры состоит из двух этапов.
Первый этап --- \textbf{этап индукции} --- восхождение от конкретного к абстрактному. Исследователь должен определить центральное системообразующее звено своей теории: концепцию, систему аксиом или аксиоматических требований, или единый методологический подход и т.д. \cite{lit6}

Причём исследователю в процессе обобщения эмпирических результатов приходится, с одной стороны, постоянно обращаться к своей предметной области в аспекте требований полноты теории (образовавшиеся
«пустоты» в предметной области). В дальнейшем их надо заполнять, в том
числе путём дополнительной опытно-экспериментальной работы либо заимствования результатов у других авторов (естественно, со ссылками). \cite{lit6}

С другой стороны, постоянно соотносить получаемые обобщения и
предметную область с совокупностью получаемых теоретических результатов в аспекте требования полноты, а также непротиворечивости
строящейся концепции, теории. \cite{lit6}

Исследователь на этапе индукции детально инвентаризирует все
имеющиеся у него результаты, все, что может представлять интерес.
И начинает группировать их по определённым основаниям классификаций в первичные обобщения, затем в обобщения второго порядка и так
далее. Происходит индуктивный процесс --- абстрагирование --- восхождение от конкретного к абстрактному --- пока все результаты не сведутся
в авторскую концепцию --- короткую (5---7 строк), но ёмкую формулировку, отражающую в самом общем сжатом виде всю суть теоретической
работы и совокупность результатов.  \cite{lit6}

Следующий этап --- \textbf{этап дедукции}. Дедуктивный процесс --- конкретизация --- восхождение от абстрактного к конкретному. \cite{lit6}

На этом этапе формулировка концепции развивается в совокупности факторов, условий, принципов, моделей, механизмов, теорем и т.д.
Иногда, если проблема исследования расчленяется на несколько относительно независимых аспектов, концепция развивается в несколько концептуальных положений --- а те уже далее развиваются в совокупности принципов и т.п. Принципы также могут развиваться в классы моделей, типы задач и т.д. Так выстраивается логическая структура научной теоретической работы. \cite{lit6}

Только правильно и обоснованно выбранная методика гарантирует
надёжность полученных при выполнении исследований результатов.
Поэтому важным этапом НИР является разработка методики исследования. Методика должна предусматривать теоретические и экспериментальные исследования.  \cite{lit6}



\section{Планирование и организация экспериментального исследования}



%\subsection{Планы экспериментальных исследований}

Эксперимент является важнейшей составной частью научных исследований, в основе которого находится научно поставленный опыт с
точно учитываемыми и управляемыми условиями. В научном языке и
исследовательской работе термин эксперимент обычно используется в
значении, общем для целого ряда сопряжённых понятий: целенаправленное наблюдение, воспроизведение объекта познания, опыт, организация особых условий его существования, проверка предсказания. В это
понятие вкладывается научная постановка опытов и наблюдение исследуемого явления в точно учитываемых условиях, позволяющих следить за ходом его развития и воссоздавать его каждый раз при повторении
этих условий. Само по себе понятие «эксперимент» означает действие,
направленное на создание условий в целях воспроизведения того или
иного явления и по возможности наиболее чистого, т.е. не осложняемого другими явлениями. \cite{lit7}

Основная цель эксперимента --- выявление свойств исследуемых
объектов, проверка справедливости гипотез и на этой основе широкое и
глубокое изучение темы научного исследования. Постановка и организация эксперимента определяются его назначением. Эксперименты, которые проводятся в различных отраслях науки, являются отраслевыми и
имеют соответствующие названия: физические, химические, биологические, социальные, психологические, и т.п.  \cite{lit7}

В зависимости от состояния знания об изучаемом объекте существуют
следующие планы исследования:
\begin{enumerate}
	\item[а)]  поисковый;
	\item[б)]  описательный;
	\item[в)]  экспериментальный.
\end{enumerate}

\textbf{Поисковый} --- применяется в случае отсутствия ясного представления о
проблеме или объекте исследования. Цель плана --- формулировка проблемы.
Поисковый план включает три этапа работы: изучение документов, опросы
экспертов, наблюдение. \cite{lit7}

\textbf{Описательный} --- применяется тогда, когда имеющиеся знания о проблеме
позволяют сформулировать описательную гипотезу. Цель плана --- проверка этой
гипотезы, получение точных характеристик изучаемого объекта. Этот план
предусматривает использование следующего набора исследовательских средств:
\begin{enumerate}[label*=\arabic*)]
	\item выборочное обследование;
	\item опрос;
	\item статистический анализ данных.
\end{enumerate}

\textbf{Экспериментальный} --- применяется тогда, когда имеющееся знание об
объекте позволяет сформулировать объяснительную гипотезу. Цель плана ---
выявить причинно-следственные связи в объекте, раскрыть его структуру,
причины, обуславливающие его функционирование и развитие. \cite{lit7}

Помимо вышеперечисленных стратегических планов существует и
методический план исследования, с помощью которого раскрываются методы
сбора, обработки и анализа информации. \cite{lit7}

Таким образом, для проведения экспериментального исследования любого типа необходимо
\begin{itemize}[label*=---]
	\item сформулировать гипотезу, подлежащую проверке;
	\item создать программы экспериментальных работ;
	\item определить способы и приёмы вмешательства в объект исследования;
	\item обеспечить условия для осуществления процедуры экспериментальных работ;
	\item разработать пути и приёмы фиксирования хода и результатов эксперимента;
	\item подготовить средства эксперимента (модели, установки, приборы, и т.п.);
	\item обеспечить эксперимент необходимым обслуживающим персоналом.
\end{itemize}



%\subsection{Планирование вычислительного эксперимента}

Особым видом экспериментальных исследований является вычислительный эксперимент. Вычислительным экспериментом называют методологию и технологию исследований, основанных на применении прикладной математики и электронно-вычислительных машин как технической базы при использовании математических моделей. Он основывается на создании
математических моделей изучаемых объектов, которые формируются с
помощью особой математической структуры, которая способна отражать свойства объекта, проявляемые им в различных экспериментальных условиях. \cite{lit8}

Теория и практика вычислительного эксперимента создавалась на
основе математического моделирования методов вычислительной математики. Вычислительный эксперимент имеет многовариантный характер,
потому что решение поставленных задач часто зависит от многочисленных входных параметров. Но тем не менее каждый конкретный расчёт в
вычислительном эксперименте проводится при фиксированных значениях всех параметров. В результате вычислительного эксперимента довольно часто ставится задача определения оптимального набора параметров. При создании оптимальной установки приходится проводить
большое число расчётов однотипных вариантов задачи, отличающихся
значением лишь некоторых параметров. Поэтому при организации вычислительного эксперимента экспериментатору необходимо использовать эффективные численные методы. \cite{lit8}

Технологический цикл вычислительного эксперимента делят на несколько этапов. \cite{lit8}
\begin{enumerate}[label*=\arabic*.]
	\item Для исследуемого объекта строится физическая модель. В рассматриваемом явлении она фиксирует разделение всех действующих факторов
	на главные и второстепенные. Последние на этом этапе исследования отбрасываются. Одновременно формулируются допущения и условия применимости модели, а также границы, в которых будут справедливы полученные результаты. Создают математическую модель специалисты, хорошо знающие данную область естествознания или техники, а также математики, представляющие себе возможности решения математической задачи.
	Модель записывается в математических терминах, в виде дифференциальных или интегро-дифференциальных уравнений.
	\item Разрабатывается метод расчёта сформулированной математической
	задачи. Эта задача представляется в виде совокупности алгебраических
	формул, по которым должны проводиться вычисления, а также условий,
	показывающих последовательность применения этих формул. Набор таких
	формул и условий носит название вычислительного алгоритма. 
	\item Разрабатывается алгоритм и программа решения задачи.
	\item При проведении расчётов в программе результат получается в
	виде некоторой цифровой информации, которую затем необходимо
	расшифровать. При вычислительном эксперименте точность информации определяется достоверностью модели, положенной в его основу,
	правильностью программ и алгоритмов для чего обычно проводятся
	предварительные «тестовые» испытания модели.
	\item Обработка результатов расчётов, их анализ и выводы. На данном
	этапе может возникнуть необходимость уточнения математической модели, то есть её упрощения или усложнения; появиться предложения по
	созданию упрощённых инженерных способов решения и формул, дающих возможность получить необходимую информацию более простым
	способом. 
\end{enumerate}

\clearpage
