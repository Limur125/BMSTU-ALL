\chapter{Метод планирования научных исследований на основе ролей}

Китайскими учёными было проведено исследование \cite{china}, описывающее проблемы традиционного планирования и организации научных исследований. Традиционные иерархические структуры недостаточны для современных научных исследований по нескольким ключевым причинам.
\begin{itemize}[label*=---]
	\item Междисциплинарный характер современных исследований: современные исследования часто предполагают совместную работу нескольких дисциплин. Традиционные структуры с трудом справляются со сложными взаимодействиями и сотрудничеством, необходимыми для таких проектов. Чёткие границы полномочий и ответственности становятся размытыми, когда специалистам из разных областей необходимо интегрировать свой опыт.
	\item Динамическое распределение задач и управление ресурсами: современные исследовательские проекты динамичны, с меняющимися задачами и потребностями в ресурсах. Иерархические структуры, как правило, негибки и медленно адаптируются к этим изменениям. Эффективное распределение ресурсов (персонала, оборудования, финансирования) и переназначение задач в зависимости от прогресса в реальном времени становится затруднительным.
	\item Затруднённое сотрудничество и коммуникация: традиционные структуры могут создавать замкнутые пространства между отделами и группами, препятствуя общению и сотрудничеству. В современных исследованиях беспрепятственный поток информации и открытая коммуникация имеют решающее значение для быстрого прогресса и инноваций. Иерархические структуры могут препятствовать этому потоку, создавая «узкие места» и ограничивая межфункциональное взаимодействие.
	\item Сложность современных проектов: масштаб и объем современных исследований зачастую намного больше и сложнее, чем те, на которые были рассчитаны традиционные структуры. Управление большими массивами данных, координация работы нескольких команд и интеграция различных технологий требуют более гибкого и адаптируемого организационного подхода.
	\item Проектно-ориентированная организация: современные исследования часто проводятся в проектных группах, которые собираются для достижения определённой цели, а затем распускаются. Традиционные постоянные иерархические структуры не очень подходят для такого динамичного жизненного цикла проекта. Более гибкая структура, основанная на ролях, позволяет легче формировать и распускать команды, согласовывая знания и опыт с потребностями проекта.
	\item Отсутствие гибкости, присущее традиционным иерархическим структурам, вступают в противоречие с динамичным, совместным и междисциплинарным характером современных научных исследований. Это требует новых организационных подходов, таких как ролевая структура, предложенная в данной научной статье, для эффективного управления сложностями современных исследовательских проектов.
\end{itemize}

В исследовании \cite{china} предлагается новый метод организации и планирования задач научных исследований, основанный на концепции «ролей». Проблемой исследования считается факт того, что существующие структуры управления научными исследованиями не позволяют эффективно распределять задачи, выделять ресурсы и сотрудничать в междисциплинарных проектах. Традиционные иерархические структуры недостаточны для решения сложных задач современных исследований.

В качестве решения поставленной проблемы предлагается внедрить в организацию научных исследований концепцию «ролей». Роль определяется связанными с ней субъектами (людьми, командами или организациями), способностями, полномочиями, политиками (правилами и ограничениями) и целями.

Распределение ресурсов на основе ролей предполагает распределение ресурсов между задачами на основе возможностей и ограничений назначенных ролей. Такой подход включает в себя
\begin{enumerate}[label*=\arabic*)]
	\item декомпозиция задач на атомарные подзадачи;
	\item создание матрицы возможностей и задач;
	\item создание матрицы требований к задачам и ресурсам;
	\item объединение этих матриц для создания матрицы распределения ресурсов между ролями и задачами;
	\item применение ограничений решаемой проблемы для фильтрации недействительных распределений;
	\item создание направленного графа, представляющего зависимости рабочего процесса и ресурсов.
\end{enumerate}

Преимуществом ролевого подхода является его направленность на повышение эффективности за счёт
\begin{enumerate}
	\item[а)]  чёткого определения обязанностей и полномочий;
	\item[б)]  оптимизации распределения ресурсов;
	\item[в)]  содействия лучшему сотрудничеству внутри и между исследовательскими группами.
\end{enumerate}

Авторы исследования \cite{china} предполагают, что эта модель поддаётся вычислениям и может быть реализована с помощью автоматизированных инструментов управления. Таким образом, предложенный метод призван преодолеть разрыв между теорией вычислительной организации и социальными аспектами исследований путём включения концепции ролей, что в конечном итоге должно привести к более эффективному управлению научными исследованиями.
