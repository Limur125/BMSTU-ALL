\chapter{Научное исследование и его этапы}

\section{Понятие научного исследования}
Одним из основных понятий, используемых на практике, является понятие
научной (научно-исследовательской) деятельности. \textbf{Научная (научно-исследовательская)} деятельность (далее --- научная деятельность) ---
деятельность, направленная на получение и применение новых знаний. \cite{lit1}

Формой существования и развития науки является научное исследование.
Деятельность в сфере науки --- \textbf{научное исследование} --- это экспериментальная или
теоретическая деятельность, направленная на получение новых знаний об
основных закономерностях строения, функционирования и развития человека,
общества, окружающей среды. \cite{lit1}

Научные исследования классифицируются по различным признакам. \cite{lit1}

В нормативных правовых актах \cite{npa1,npa2} о науке научные исследования делят по
целевому назначению на фундаментальные, прикладные, поисковые и
разработки.

\textbf{Фундаментальные научные исследования} --- это экспериментальная или
теоретическая деятельность, направленная на получение новых знаний об
основных закономерностях строения, функционирования и развития человека,
общества, окружающей среды.

\textbf{Прикладные научные исследования} --- это исследования, направленные
преимущественно на применение новых знаний для достижения практических
целей и решения конкретных задач. Иными словами, они направлены на решение
проблем использования научных знаний, полученных в результате
фундаментальных исследований, в практической деятельности людей.

Стоит отметить, что некоторые научные исследования зачастую представляют собой сочетание двух
названных видов, и поэтому их следует именовать теоретико-прикладными.

\textbf{Поисковые научные исследования} --- исследования,
направленные на получение новых знаний в целях их последующего
практического применения (ориентированные научные исследования) и (или) на
применение новых знаний (прикладные научные исследования) и проводимые
путём выполнения научно-исследовательских работ.

\textbf{Экспериментальные разработки} --- деятельность, которая основана на
знаниях, приобретённых в результате проведения научных исследований или на
основе практического опыта, и направлена на сохранение жизни и здоровья человека, создание новых материалов, продуктов, процессов, устройств, услуг,
систем или методов и их дальнейшее совершенствование.

По длительности научные исследования можно разделить на
\textbf{долгосрочные, краткосрочные и экспресс-исследования}. \cite{lit2}

В зависимости от форм и методов исследования некоторые авторы
выделяют: \textbf{экспериментальное, методическое, описательное,
экспериментально-аналитическое, историко-биографическое исследования и
исследования смешанного типа}. \cite{lit2}

По источнику финансирования различают научные исследования
\textbf{бюджетные, хоздоговорные и нефинансируемые}. Бюджетные исследования
финансируются из средств бюджета РФ или бюджетов субъектов РФ.
Хоздоговорные исследования финансируются организациями заказчиками по
хозяйственным договорам. Нефинансируемые исследования могут выполняться
по инициативе учёного, индивидуальному плану преподавателя. \cite{lit3}

\section{Этапы проведения научного исследования}

Важным моментом в организации научных исследований является выбор
этапов, процедур и операций, а также их последовательное расположение и
осуществление. \cite{lit4}

При организации исследования всегда необходимо знать, что конечный его
результат – некоторое интегральное знание, так как исходной и конечной целью
всякого исследования является создание или получение новых значений,
выработка которых не осуществляется сразу.~\cite{lit5}

Под \textbf{организацией} исследования следует понимать такую
последовательность и состав процедур и операций, которые гарантируют
достижение цели в оптимальные сроки. \cite{lit5}

В самом общем виде исследование включает в себя следующие этапы:
\begin{enumerate}[label*=\arabic*)]
	\item  постановку и уточнение задачи;
	\item выдвижение гипотез;
	\item теоретическую разработку гипотез, их проверку и оценку;
	\item создание программ и инструкций для экспериментов; 
	\item проведение экспериментальных исследований;
	\item сбор и обработку эмпирических данных;
	\item сравнение выдвинутых гипотез с результатами экспериментов и
	наблюдений;
	\item их оценку, принятие или отбрасывание;
	\item формулирование перечисленных вопросов и постановка новых задач.
\end{enumerate}
Эти основные этапы задают структуру научного исследования, т. к.
определяют отношения между последовательностью соответствующих
процедур и операций, фиксируют их содержание. \cite{lit5}

Необходимо отметить, что далеко не все исследования включают в себя
все перечисленные этапы. Некоторые могут ограничиваться лишь некоторыми
из них. \cite{lit5}

В науке имеет место исследование, носящее чисто эмпирический характер.
Преобладание эмпирических теоретических этапов в исследовании является
показателем уровня развития той или иной научной дисциплины. Но тем не
менее большинство научных исследований включают в себя все перечисленные
выше этапы. Очень важным в организации исследования является правильная 
его методологическая организация, с одной стороны, и обоснованный выбор
одного пути решения исследовательской проблемы из множества других, с
другой стороны. \cite{lit5}

В первом случае необходимо знать, что исследование методологически
организованно тогда, когда все его этапы выделены и описаны одновременно с
постановкой основной исходной задачи. Это описание включает в себя
\begin{enumerate}[label*=\arabic*)]
	\item полный перечень этапов достижения поставленной цели;
	\item схему взаимосвязей всех задач исследования;
	\item перечень всех процедур и операций, необходимых для достижения
	целей;
	\item требования к средствам, необходимым для реализации процедур и
	операций;
	\item перечень предполагаемых трудностей в осуществлении
	этапов исследования.
\end{enumerate}
Данное описание называется \textbf{исследовательским проектом} или
\textbf{исследовательской программой}.

Второй случай ориентирует исследователя на то, что при выборе из ряда
исследований, предпринимаемых для решения одной и той же задачи,
преимущество имеет то, которое ведёт к получению необходимого знания
через:
\begin{enumerate}
	\item[а)]  использование меньшего числа процедур и операций;
	\item[б)]  использование менее затратных процедур и операций.
\end{enumerate}
Помимо этого, необходимо помнить и тот факт, что исследовательские
проекты являются тем более эффективнее в методологическом отношении, чем
более простую структуру исследования они предполагают. \cite{lit5}
