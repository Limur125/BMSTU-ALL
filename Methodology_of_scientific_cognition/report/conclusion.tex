\ssr{ЗАКЛЮЧЕНИЕ}

В рамках настоящей работы был проведён анализ методов планирования и организации научных исследований. 

Было рассмотрено понятие научного исследования и выделены следующие этапы его проведения:
\begin{enumerate}[label*=\arabic*)]
	\item  постановка и уточнение задачи;
	\item выдвижение гипотез;
	\item теоретическая разработка гипотез, их проверка и оценка;
	\item создание программ и инструкций для экспериментов; 
	\item проведение экспериментальных исследований;
	\item сбор и обработка эмпирических данных;
	\item сравнение выдвинутых гипотез с результатами экспериментов и
	наблюдений;
	\item их оценка, принятие или отбрасывание;
	\item формулирование перечисленных вопросов и постановка новых задач.
\end{enumerate}

Далее были рассмотрены аспекты традиционного планирования и организации научных исследований. План проведения теоретического исследования состоит из этапов индукции и дедукции. В свою очередь, для проведения экспериментального исследования необходимо
\begin{itemize}[label*=---]
	\item сформулировать гипотезу, подлежащую проверке;
	\item создать программы экспериментальных работ;
	\item определить способы и приёмы вмешательства в объект исследования;
	\item обеспечить условия для осуществления процедуры экспериментальных работ;
	\item разработать пути и приёмы фиксирования хода и результатов эксперимента;
	\item подготовить средства эксперимента (модели, установки, приборы, и т.п.);
	\item обеспечить эксперимент необходимым обслуживающим персоналом. 
\end{itemize}

Для противопоставления традиционным методам планирования и организации научных исследований был проанализирован метод планирования на основе ролей. Данный метод призван преодолеть разрыв между теорией вычислительной организации и социальными аспектами исследований путём включения концепции ролей, что в конечном итоге должно привести к более эффективному управлению научными исследованиями.

