\ssr{ВВЕДЕНИЕ}

Планирование и организация научных исследований являются ключевыми элементами успешного проведения научных изысканий. Эффективное планирование позволяет оптимизировать использование ресурсов, сократить сроки проведения исследований и повысить качество получаемых результатов.

Актуальность данной темы обусловлена стремительным развитием науки и технологий. Современные научные исследования становятся все более сложными и требуют междисциплинарного подхода. В этих условиях эффективное планирование позволяет учёным ориентироваться в постоянно растущем потоке информации и достигать поставленных целей. Кроме того, возрастающая конкуренция в научной сфере делает необходимым применение инновационных методов организации научной деятельности, которые позволяют повысить результативность исследований и ускорить получение новых знаний.

Целью данной работы является анализ методов планирования и организации научных исследований. Для достижения поставленной цели необходимо решить следующие задачи.

\begin{enumerate}[label*=\arabic*.]
	\item Рассмотреть понятие научного исследования и этапы его проведения.
	\item Рассмотреть традиционные подходы к планированию и организации научных исследований.
	\item Проанализировать метод планирования научных исследований на основе ролей.
\end{enumerate}

\clearpage
