\documentclass[a4paper,14pt]{extreport}

\usepackage{cmap} % Улучшенный поиск русских слов в полученном pdf-файле
\usepackage[T2A]{fontenc} % Поддержка русских букв
\usepackage[utf8]{inputenc} % Кодировка utf8
\usepackage[english,russian]{babel} % Языки: русский, английский
%\usepackage{pscyr} % Нормальные шрифты
\usepackage{enumitem}

\usepackage[normalem]{ulem}
\usepackage{cancel}
\usepackage{float}
\usepackage[14pt]{extsizes}

\usepackage{caption}
\captionsetup{labelsep=endash}
\captionsetup[figure]{name={Рисунок}}

\usepackage{amsmath}

\usepackage{geometry}
\geometry{left=30mm}
\geometry{right=15mm}
\geometry{top=20mm}
\geometry{bottom=20mm}

\usepackage{titlesec}
\titleformat{\section}
	{\normalsize\bfseries}
	{\thesection}
	{1em}{}
\titlespacing*{\chapter}{0pt}{-30pt}{8pt}
\titlespacing*{\section}{\parindent}{*4}{*4}
\titlespacing*{\subsection}{\parindent}{*4}{*4}

\usepackage{setspace}
\onehalfspacing % Полуторный интервал

\frenchspacing
\usepackage{indentfirst} % Красная строка

\usepackage{titlesec}
\titleformat{\chapter}{\LARGE\bfseries}{\thechapter}{20pt}{\LARGE\bfseries}
\titleformat{\section}{\Large\bfseries}{\thesection}{20pt}{\Large\bfseries}

\usepackage{listings}
\usepackage{xcolor}

% Для листинга кода:
\lstset{ %
	language=c++,   					% выбор языка для подсветки	
	basicstyle=\small\sffamily,			% размер и начертание шрифта для подсветки кода
	numbers=left,						% где поставить нумерацию строк (слева\справа)
	%numberstyle=,					% размер шрифта для номеров строк
	stepnumber=1,						% размер шага между двумя номерами строк
	numbersep=5pt,						% как далеко отстоят номера строк от подсвечиваемого кода
	frame=single,						% рисовать рамку вокруг кода
	tabsize=4,							% размер табуляции по умолчанию равен 4 пробелам
	captionpos=t,						% позиция заголовка вверху [t] или внизу [b]
	breaklines=true,					
	breakatwhitespace=true,				% переносить строки только если есть пробел
	escapeinside={\#*}{*)},				% если нужно добавить комментарии в коде
	backgroundcolor=\color{white},
}

\usepackage{pgfplots}
\usetikzlibrary{datavisualization}
\usetikzlibrary{datavisualization.formats.functions}

\usepackage{graphicx}
\newcommand{\img}[3] {
	\begin{figure}[h!]
		\center{\includegraphics[height=#1]{inc/img/#2}}
		\caption{#3}
		\label{img:#2}
	\end{figure}
}
\newcommand{\boximg}[3] {
	\begin{figure}[h]
		\center{\fbox{\includegraphics[height=#1]{inc/img/#2}}}
		\caption{#3}
		\label{img:#2}
	\end{figure}
}

\usepackage[justification=centering]{caption} % Настройка подписей float объектов

\usepackage[unicode,pdftex]{hyperref} % Ссылки в pdf
\hypersetup{hidelinks}

\usepackage{csvsimple}

\setlength{\parindent}{1.25cm}

\makeatletter
\renewcommand\@biblabel[1]{#1.}
\makeatother

\newcommand{\code}[1]{\texttt{#1}}


\begin{document}

	\setcounter{page}{3}
	
	\def\contentsname{СОДЕРЖАНИЕ}
	
	\tableofcontents
	
	\chapter*{ВВЕДЕНИЕ}
	\addcontentsline{toc}{chapter}{ВВЕДЕНИЕ}
	
	Целью работы является овладение умениями и навыками выполнения индивидуального задания на практику, умениями и навыками разработки систем автоматизации и цифровизации производства.
	
	Задачи практики:
	\begin{itemize}
		\item получить навыки проведения анализа профессионально-технической информации; 
		\item изучить правила и регламенты работы организации прохождения практики, технологий, используемых в ходе разработки программного обеспечения;
		\item освоить технологии, используемые на предприятии, при разработке программного обеспечения;
		\item разработать систему для автоматизации и цифровизации производства.
	\end{itemize}

	\chapter{ОСНОВНАЯ ЧАСТЬ}

	\section{Характеристика предприятия} 
	Научно-производственное предприятие «Исток» им. Шокина» основано в 1943 году и расположено в городе Фрязино Московской области, Россия.
	
	Основное направление деятельности - новые разработки и серийное производство современных и перспективных изделий СВЧ-электроники для всех видов связи и радиолокации.
	
	В настоящее время «АО «НПП «Исток» им. Шокина» поддерживает около 30\% всей номенклатуры изделий СВЧ-электроники, выпускаемой в России, что определяет его головную роль в отрасли.
	
	Предприятие обладает замкнутыми технологическими циклами разработки и производства СВЧ-транзисторов, монолитных интегральных схем, модулей СВЧ любой функциональной сложности, электровакуумных СВЧ-приборов и комплексированных СВЧ-устройств на их основе, радиоэлектронной аппаратуры и ее составных частей.
	
	Сегодня «АО «НПП «Исток» им. Шокина» реализует стратегию цифровизации предприятия, ключевыми моментами которой являются прозрачность информационных систем, бизнес-анализ и оптимизация бизнес процессов, применение современных технологий в производстве (интернет вещей, мониторинг оборудования, MES-системы, системы технического обслуживания и ремонта), сквозная интеграция - сплошной цифровой информационный поток.
	
	Главными приоритетами развития предприятия является выпуск широкого ассортимента конкурентоспособной на мировом рынке продукции, повышение уровня эффективности производственных процессов, построение эффективной структуры менеджмента и создание комфортных условий труда для всех сотрудников предприятия.
	\newpage
	\section{Характеристика отдела предприятия} 
	
	Отдел автоматизированных систем управления занимается разработкой ПО для внутренней инфраструктуры предприятия, в частности, разработкой цифрового производства и подсистемы внутрицехового планирования.
	
	\section{Характеристика проделанной работы}
	Были изучены инструкции по технике личной, противопожарной и информационной безопасности, правила и регламенты работы организации прохождения практики, а также изучены общие принципы организации аналитической работы на предприятии.
	
	В рамках производственной практики необходимо было разработать модуль оформления заявок, собирающий заказы необходимых материалов для подразделений предприятия.
	
	В разработанном веб-приложении пользователь может просматривать список заявок, производить простые манипуляции с заявками, такие как добавление новой, удаление, редактирование и утверждение. 
	
	Для разработки интерфейса приложения была использована технология Blazor~\cite{blazor}, позволяющая разрабатывать веб-приложения, основанные на языке программирования C\# и HTML. Для работы с базой данных была использована технология Entity Framework Core~\cite{efcore}. 
	
	Entity Framework Core (EF Core) представляет собой технологию для доступа к данным. EF Core позволяет работать базами данных, но представляет собой более высокий уровень абстракции: EF Core позволяет абстрагироваться от самой базы данных и ее таблиц и работать с данными независимо от типа хранилища.
	
	В рамках выполнения данной задачи необходимо было создать две табличные сущности: заявки и материалы.
	
	Список свойств класса заявки:
	
	\begin{itemize}
		\item Id (int);
		\item номер заявки (int);
		\item дата создания (DateTime);
		\item вычисляемое поле FullNumber (string);
		\item статус (перечисление со значениями: Создана, Удалена, Утверждена);
		\item вычисляемое поле TextStatus (string);
		\item подразделение (string);
		\item автор (string).
	\end{itemize}
	
	Список свойств класса материала:
	
	\begin{itemize}
		\item Id (int);
		\item статус (перечисление со значениями: создан, удалён);
		\item вычисляемое поле TextStatus (string);
		\item наименование материала (string);
		\item код материала (string, с ограничением на 10 символов);
		\item количество (int, с ограничением на минимальное значение 1);
		\item комментарий (string);
		\item ссылка на заявку ProposalId (int, об этом подробнее ниже).
	\end{itemize}

	В результате применения полученных навыков было разработано веб-приложение по оформлению заявок.
	
	На рисунках \ref{fig:reqList}, \ref{fig:reqInfo}, \ref{fig:reqMaterials}  представлен пользовательский интерфейс приложения.
	
\begin{figure}[H]
	\centering
	\includegraphics[width=1\linewidth]{"Screenshot 2023-08-20 113401"}
	\caption{Список заявок}
	\label{fig:reqList}
\end{figure}

\begin{figure}[H]
	\centering
	\includegraphics[width=1\linewidth]{"Screenshot 2023-08-20 113430"}
	\caption{Информация по заявке}
	\label{fig:reqInfo}
\end{figure}

\begin{figure}[H]
	\centering
	\includegraphics[width=1\linewidth]{"Screenshot 2023-08-20 113513"}
	\caption{Список материалов конкретной заявки}
	\label{fig:reqMaterials}
\end{figure}

	
	\chapter*{ЗАКЛЮЧЕНИЕ}
	\addcontentsline{toc}{chapter}{ЗАКЛЮЧЕНИЕ}
	
	В ходе прохождения производственной практики была достигнута поставленная цель.
	
	Были решены все поставленные задачи:
	\begin{itemize}
		\item получены навыки проведения анализа профессионально-технической информации; 
		\item изучены правила и регламенты работы организации прохождения практики, технологий, используемых в ходе разработки ПО; 
		\item освоены технологии, используемые на предприятии, при разработке ПО;
		\item разработана система для автоматизации и цифровизации производства.
	\end{itemize}

\newpage

\renewcommand\bibname{СПИСОК ИСПОЛЬЗОВАННЫХ ИСТОЧНИКОВ}
\addcontentsline{toc}{chapter}{СПИСОК ИСПОЛЬЗОВАННЫХ ИСТОЧНИКОВ}
\bibliographystyle{utf8gost705u} 
\bibliography{51-biblio}

\end{document}