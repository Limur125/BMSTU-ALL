\chapter*{ВВЕДЕНИЕ}
\addcontentsline{toc}{chapter}{ВВЕДЕНИЕ}

С бурным развитием технологий мобильной связи~\cite{development}, провайдеры услуг инвестируют огромные средства в сетевую инфраструктуру. Из-за высокой стоимости точное и эффективное планирование мобильной сети становится очень важным. С быстрым ростом масштабов сети и числа пользователей, необходимыми стали эффективные количественные методы оптимального расположения базовых станций.

Оптимальное расположение базовых станций и подключение к ним клиентов позволит обеспечить необходимую производительность сети и требуемое качество услуг при минимальной стоимости. Задача размещения с одним источником обслуживания при минимальной стоимости покрытия. Задача размещения с одним источником обслуживания при наличии ограничений на его пропускную способность относится к классу NP-трудных, поэтому точные методы, обладающие экспоненциальной зависимостью времени работы от количества входных данных, с увеличением размерности задачи теряют свою эффективность. Поэтому при создании сетей большой размерности целесообразно отказаться~\cite{req-cost} от классических методов решения в пользу современных технологий оптимизации с полиномиальной сложностью, позволяющих за приемлемое время получить результаты, близкие к оптимальным.

Цель научно-исследовательской работы: классификация методов выбора расположения базовых станций.

Задачами данной научно-исследовательской работы являются:
\begin{itemize}[label=---]
	\item проведение анализа предметной области расположения базовых станций;
	\item описание существующих методов методов выбора расположения базовых станций;
	\item выделение критериев сравнения методов  выбора расположения базовых станций;
	\item классификация методов выбора расположения базовых станций по выделенным критериям.
\end{itemize}
