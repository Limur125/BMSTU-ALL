Задано множество элементарных областей, на которые поделена целевая область. Все множество можно разбить на две категории:
\begin{itemize}
	\item[---] области, с которых необходимо собирать информацию;
	\item[---] станции для сбора и передачи на шлюз данных;
\end{itemize}

Задано множество вершин $A = a_i, i = \overline{0,n}$ на целевой области.
Каждая вершина $a_i$ имеет координаты $(x_i, y_i)$.

Множество $A$ состоит из двух подмножеств:
\begin{itemize}
	\item[---] $A_1$ множество вершин, соответствующее географическим центрам областей, с которых необходимо собирать информацию.
	\item[---] $A_2$ множество мест, где размещены базовые станции.
\end{itemize}

С вершин $A_1$ необходимо собирать информацию. Каждой вершине $a_i \in A_1$ приписана величина $\nu_i$ – максимальный объем информации в единицу времени, который генерирует соответствующая вершине область. 

По определению: \[A_1 \cap A_2 = \emptyset;$\] \[A_1 \cup A_2 = A.\]

Все вершины пронумерованы так, что: 

\[A_1 = {a_i}, i = \overline{1,n_1};\]

\[A_2 = {a_i}, i = \overline{n_1 + 1, n}.\]

Каждой базовой станции, размещенной на вершине множества $A_2$ приписаны три параметра $s_i = {{r_i_j}, {R_i_j}, \vartheta_i}$, где:
\begin{itemize}
	\item[---] ${r_i_j}$ множество радиусов телекоммуникационного покрытия базовой станции. Параметр $r_i_j$ характеризует дальность связи между базовой станцией размещенной в вершине $a_i, a_i \in A_2$ и географическим центром области в вершине, $a_j \in A_1$;
	\item[---] ${R_i_j}$ множество радиусов связи базовой станции. Параметр $R_i_j$ характеризует дальность связи между станциями $a_i \in A_2$ и $a_j \in A_2$, $i = \overline{n_1 + 1,n}, j = \overline{n_1 + 1, n}, i \ne j$;
	\item[---] $𝑖\vartheta_i$ объем информации в единицу времени, который может быть получен от областей, обслуживаемых базовой станцией, или других базовых станций.
\end{itemize}

Также задана специальная базовая станция --- шлюз $s_0 = {{R_0_j}, \vartheta_0}$, размещенная на вершине $a_0$ с координатами ${x_0, y_0}$. Шлюз $s_0$ не имеет телекоммуникационного покрытия и служит для сбора всей информации в сети. По условию задачи
величина $\vartheta_0 = \infty$, со шлюзом и между собой
могут быть связаны только вершины множества $A_2$, то есть только базовые станции.

Требуется проверить, что при заданном наборе и размещении базовых станций на множества $A_2$ вся имеющаяся информация с областей множества $А_1$ может быть собрана и передана системой базовых станций до шлюза $s_0$.

Пусть – $B = A_2$ множество потенциальных передатчиков и $T = A_1$ – множество 
приёмников, тогда:
\begin{equation}
	z_\beta = 
	\begin{cases}
		1,& \text{если базовая станция $\beta$ активна},\\
		0,& \text{иначе}
	\end{cases}\text{,}
\end{equation}
где $\beta \in B$;

\begin{equation}
	\chi_\tau_\beta = 
	\begin{cases}
		1,& \text{если приемник $\tau$ обслуживается базовой станцией $\beta$},\\
		0,& \text{иначе}
	\end{cases}\text{,}
\end{equation}
где $\beta \in B$, $\tau \in T$.

Для всех приёмников $\tau$, для которых $\chi_\tau_\beta = 1$:
\begin{equation}
	\frac{a_\tau_\beta z_\beta}{\mu + \sum_{b \in B}a_\tau_b z_b} \ge \delta, 
\end{equation}
где $\beta \in B$, $\tau \in T$, $a_\tau_\beta > 0$ --- измеренная в $\tau$ мощность сигнала полученного от $\beta$, $\delta \ge 0$ --- нижняя граница SIRN, требуемое качество связи, $\mu > 0$ --- шум в системе.

Будем считать что приемник находится в радиусе телекоммуникационного покрытия базовой станции тогда и только тогда, когда у него будет хорошее качество связи с базовой станцией, то 
есть $\delta \ge 13$дБ. Характеристики качества сигнала приведены в таблице \ref{tbl:sig-quality}.

\captionsetup{justification=raggedright, singlelinecheck=false}
\begin{table}[H]
	\centering
	\begin{threeparttable}
	\caption{\label{tbl:sig-quality}Характеристики качества сигнала с использованием SINR}
	\begin{tabular}{|c|c|}
		\hline
		Качество сигнала & SINR (дБ)\\\hline
		Отличное& $\delta \ge 20$\\\hline
		Хорошее& $13 \ge \delta < 20$ \\\hline
		Среднее& $0 \ge \delta < 13$ \\\hline
		Плохое& $\delta < 0$\\\hline
	\end{tabular}	
\end{threeparttable}
\end{table}

Для нахождения мощности сигнала между отдельными приёмниками и 
излучателями  используется формула~\cite{atb}:

\begin{equation}
	a_\tau_\beta = W_\beta - PL(f, d),
\end{equation}
где $W_\beta$ --- мощность излучателя базовой станции, $PL(f, d)$ --- потери 
сигнала с расстоянием $d$ на частоте $f$. 

Её формула~\cite{pl} $PL(f, d)$ представлена ниже:
\begin{equation}
	PL(f, d) = FSLP(f, 1m) + 10т\lg(d) + \chi_\sigma,
\end{equation}

где $n$ --- константа, получаемая из опытных данных. При решении будем
использовать данные компании Nokia, $\chi_\sigma$ --- стандартное отклонение, $FSLP(f, 1m)$ --- потери сигнала на расстоянии одного метра, которые вычисляется по 
формуле:
\begin{equation}
	FSLP(f, 1m)= 20 \lg\frac{4\pi}{c}
\end{equation}