\chapter*{\hfill ВВЕДЕНИЕ \hfill}
\addcontentsline{toc}{chapter}{ВВЕДЕНИЕ}

Лучистый перенос энергии играет существенную роль в высокотемпературных газодинамических процессах. Принято считать, что характерной температурой газа, при которой его тепловое излучение начинает заметно влиять на теплообмен, является температура около $10^4$ K. При очень высоких температурах излучение вещества влияет и на его динамику. К таким явлениям относятся, например, процессы в звездных атмосферах, вхождение летательных аппаратов в атмосферу, сильноточные электрические разряды, лазерная плазма~[\cite{found_gas}]~[\cite{mod_gas}]. В настоящее время актуальным является создание мощных источников ультрафиолетового и мягкого рентгеновского излучения для проектирования новых технологий наноиндустрии, управляемого термоядерного синтеза, лабораторной астрофизики и фундаментальных исследований свойств вещества в экстремальных условиях. Для всех этих задач востребованным является высокоточное предсказательное моделирование, учитывающее геометрию конструкций и реальные свойства материалов. При выполнении такого рода расчетов в большинстве случаев необходимо решать связанные радиационно-газодинамические задачи, т.к. газодинамические параметры формируются под влиянием радиационного теплопереноса, а поле теплового излучения зависит от лучеиспускательной способности газа и его прозрачности. Повсеместное активное внедрение высокопроизводительной вычислительной техники позволяет сделать такие расчеты серийными и тем самым является важным фактором успешного практического использования радиационно-газодинамических моделей в анализе экспериментальных данных об излучающих средах и при разработке новых технологий с применением источников излучений. 

Цель научно-исследовательской работы: анализ методов и алгоритмов трассировки лучей в задачах расчета переноса селективного излучения в осветительных системах с высокотемпературными неоднородными средами.

Задачами данной научно-исследовательской работы являются:
\begin{itemize}[label=---]
    \item ввести основные понятия предметной области переноса селективного излучения в высокотемпературных неоднородных средах;
    \item описать алгоритмы трассировки лучей в осветительных системах, содержащих объемно излучающие и поглощающие элементы;
    \item сформулировать критерии сравнения методов;
    \item провести критический анализ существующих методов решения задач расчета переноса селективного излучения в высокотемпературных неоднородных средах.
\end{itemize}
