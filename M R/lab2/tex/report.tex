\documentclass[12pt]{report}
\usepackage[utf8]{inputenc}
\usepackage[russian]{babel}
\usepackage[14pt]{extsizes}
\usepackage{listings}
\usepackage{graphicx}
\usepackage{amsmath,amsfonts,amssymb,amsthm,mathtools} 
\usepackage{pgfplots}
\usepackage{filecontents}
\usepackage{float}
\usepackage{indentfirst}
\usepackage{eucal}
\usepackage{enumitem}
%s\documentclass[openany]{book}
\frenchspacing

\usepackage{titlesec}
\titleformat{\section}
{\normalsize\bfseries}
{\thesection}
{1em}{}
\titlespacing*{\chapter}{0pt}{-30pt}{8pt}
\titlespacing*{\section}{\parindent}{*4}{*4}
\titlespacing*{\subsection}{\parindent}{*4}{*4}

\usepackage{indentfirst} % Красная строка

\usetikzlibrary{datavisualization}
\usetikzlibrary{datavisualization.formats.functions}

\usepackage{amsmath}

\usepackage{amssymb}

% Для листинга кода:
\lstset{ %
	language=lisp,                 % выбор языка для подсветки (здесь это С)
	texcl=true,
	extendedchars=\true,
	basicstyle=\small\sffamily, % размер и начертание шрифта для подсветки кода
	numbers=left,               % где поставить нумерацию строк (слева\справа)
	numberstyle=\tiny,           % размер шрифта для номеров строк
	stepnumber=1,                   % размер шага между двумя номерами строк
	numbersep=5pt,                % как далеко отстоят номера строк от подсвечиваемого кода
	showspaces=false,            % показывать или нет пробелы специальными отступами
	showstringspaces=false,      % показывать или нет пробелы в строках
	showtabs=false,             % показывать или нет табуляцию в строках
	frame=single,              % рисовать рамку вокруг кода
	tabsize=2,                 % размер табуляции по умолчанию равен 2 пробелам
	captionpos=t,              % позиция заголовка вверху [t] или внизу [b] 
	breaklines=true,           % автоматически переносить строки (да\нет)
	breakatwhitespace=false, % переносить строки только если есть пробел
	escapeinside={\#*}{*)},  % если нужно добавить комментарии в коде
	%inputencoding=utf8x,
	%extendedchars=\true
}



\usepackage[left=2cm,right=2cm, top=2cm,bottom=2cm,bindingoffset=0cm]{geometry}
% Для измененных титулов глав:
\usepackage{titlesec, blindtext, color} % подключаем нужные пакеты
\definecolor{gray75}{gray}{0.75} % определяем цвет
\newcommand{\hsp}{\hspace{20pt}} % длина линии в 20pt
% titleformat определяет стиль
\titleformat{\chapter}[hang]{\Huge\bfseries}{\thechapter\hsp\textcolor{gray75}{|}\hsp}{0pt}{\Huge\bfseries}


% plot
\usepackage{pgfplots}
\usepackage{filecontents}
\usetikzlibrary{datavisualization}
\usetikzlibrary{datavisualization.formats.functions}

\begin{document}
	\begin{titlepage}
	\newgeometry{pdftex, left=2cm, right=2cm, top=2.5cm, bottom=2.5cm}
	\fontsize{12pt}{12pt}\selectfont
	\noindent \begin{minipage}{0.15\textwidth}
		\includegraphics[width=\linewidth]{b_logo.jpg}
	\end{minipage}
	\noindent\begin{minipage}{0.9\textwidth}\centering
		\textbf{Министерство науки и высшего образования Российской Федерации}\\
		\textbf{Федеральное государственное бюджетное образовательное учреждение высшего образования}\\
		\textbf{«Московский государственный технический университет имени Н. Э.~Баумана}\\
		\textbf{(национальный исследовательский университет)»}\\
		\textbf{(МГТУ им. Н. Э.~Баумана)}
	\end{minipage}
	
	\noindent\rule{18cm}{3pt}
	\newline\newline
	\noindent ФАКУЛЬТЕТ $\underline{\text{«Информатика и системы управления»~~~~~~~~~~~~~~~~~~~~~~~~~~~~~~~~~~~~~~~~~~~~~~~~~~~~~~~}}$ \newline\newline
	\noindent КАФЕДРА $\underline{\text{«Программное обеспечение ЭВМ и информационные технологии»~~~~~~~~~~~~~~~~~~~~~~~}}$\newline\newline\newline\newline\newline\newline

	\fontsize{18pt}{18pt}\selectfont
	\begin{center}
        \textbf{ОТЧЕТ ПО ПРАКТИКУМУ №2}\\
        \textbf{по курсу «Архитектура ЭВМ»}\\
        ~\\
        \fontsize{16pt}{16pt}\selectfont
        \text{«Обработка и визуализация графов}
        \text{в вычислительном комплексе Тераграф»}
	\end{center}
    ~\\

	\fontsize{14pt}{14pt}\selectfont
	\noindent\text{Студент}
    \uline{
        ~~~~~~~~~~~~~~~~~~~~~~~~~~~~~~
        Золотухин Алексей Вячеславович
        ~~~~~~~~~~~~~~~~~~~~~~~~~~~~~~
    }
    \newline\newline
	\noindent\text{Группа}
    \uline{
        ~~~~~~~~~~~~~~~~~~~~~~~~~~~~~~~~~~~~~~~~~~~~
        ИУ7-54Б
        ~~~~~~~~~~~~~~~~~~~~~~~~~~~~~~~~~~~~~~~~~~~~
    }
    \newline\newline
	\noindent\text{Оценка (баллы)}
    \uline{
        ~~~~~~~~~~~~~~~~~~~~~~~~~~~~~~~~~~~~~~~~~~~~~~~~~
        ~~~~~~~~~~~~~~~~~~~~~~~~~~~~~~~~~~~~~~~~~~~~~~~~~
    }
    \newline\newline
	\noindent\text{Преподаватель}
    \uline{
        ~~~~~~~~~~~~~~~~~~~~~~~~~~~~~~
        Ибрагимов С.В.
        ~~~~~~~~~~~~~~~~~~~~~~~~~~~~~~
    }
    \newline
    ~\\
	~\\
	~\\
    \vspace{17mm}

	\begin{center}
		\the\year~г.
	\end{center}

    \restoregeometry
\end{titlepage}


	\setcounter{page}{2}
\section{Условие лабораторной работы}
В информационный центр приходят клиенты через интервалы времени 10$\pm$2 минуты. Если все три имеющихся оператора заняты, клиенту отказывают в обслуживании. Операторы имеют разную производительность и могут обеспечивать обслуживание среднего запроса за 20$\pm$5, 40$\pm$10 и 40$\pm$20 минут. Клиенты стараются занять свободного оператора с максимальной производительностью. Полученные запросы сдаются в приёмный накопитель, откуда они выбираются на обработку. На первой картинке запросы от 1 и 2 оператора, на второй от третьего оператора. Время обработки на первом и втором компьютере равно 15 и 30 минут. Смоделировать процесс обработки 100 запросов, которые пришли. Определить вероятность отказа.

В процессе взаимодействия клиентов возможны два режима:
\begin{enumerate}
	\item Режим нормального обслуживания, когда клиент выбирает одного свободного оператора.
	\item Режим отказа.
\end{enumerate}

Эндогенные переменные этой модели -- время обработки задания $i$-м оператором и время решения задачи на $j$-м компьютере.

Экзогенные переменные -- число обслуженных клиентов и число клиентов, получивших отказ.

\newpage

\section{Теоретическая часть}
В этом разделе будет дано описание распределений, использованных в лабораторной работе, а также подходов к решению задачи.

\subsection{Равномерное распределение}

Функция плотности распределения $f(x)$ случайной величины $X$, имеющей равномерное распределение на отрезке $[a, b]$ ($X \sim R(a, b)$), где $a, b \in R$, имеет следующий вид:
\begin{equation}
	f(x)=\begin{cases}
		\frac{1}{b - a}, & x \in [a, b] \\
		0, & \text{иначе}.
	\end{cases}
\end{equation}

Соответствующая функция распределения $F(x) = \int_{-\infty}^{x}f(t)dt$ принимает вид: 
\begin{equation}
	F(x)=\begin{cases}
		0, & x < a, \\
		\frac{x - a}{b - a}, & x \in [a, b] \\
		1, & x > b
	\end{cases}
\end{equation}

\subsection{Визуальное представление модели}
Визуальное представление модели представлена на рисунке 1:

\captionsetup{justification=centering}
\begin{figure}[h]
	\begin{center}
		\includegraphics[width=0.7\linewidth]{inc/flow.png}
	\end{center}
	\caption{Структурная схема потока}
\end{figure}

\section{Демонстрация работы программы}

На рисунке 2 представлена демонстрация работы программы.

\begin{figure}[h]
	\centering
	\includegraphics[width=0.9\linewidth]{inc/demo}
	\caption{Демонстрация работы программы}
	\label{fig:demo}
\end{figure}



\end{document}