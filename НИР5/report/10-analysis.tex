\chapter{Аналитический раздел}

\section{Методы генерации звучащей речи}

С помощью одного метода синтезированная речь создается путем объединения фрагментов записанной речи, хранящихся в базе данных. 
В другом методе синтезатор моделирует речевой тракт и другие характеристики человеческого голоса для создания полностью синтезированной речи. 
Качество полученной речи определяется по её сходству с человеческим голосом и по её способности быть понятной.

Система преобразования текста в речь состоит из следующих частей.
\begin{enumerate}
\item Графематический анализ --- этап, который обеспечивает выделение синтаксических или структурных единиц из входного текста, который может представлять собой линейную структуру, содержащую единый фрагмент текста \cite{graph}:
\begin{itemize}[label=---]
	\item выделение в тексте предложений;
	\item разметка текста на буквы, цифры, знаки пунктуации и специальные символы;
	\item выделение слов в предложениях.
\end{itemize}
\item Морфологический анализ – переход от словоформ к их леммам (словарным
формам лексем), или основам:
\begin{itemize}[label=---]
	\item нормализация текста --- расшифровка обозначений и аббревиатур, а также трансформация чисел и знаков в буквенно-словесную форму;
	\item определение места ударения и морфо-грамматических характеристик слов в предложении;
	\item cнятие омонимии.
\end{itemize}
\item Синтаксический анализ, т.е. выявление грамматической структуры предложений
текста --- синтаксических связей между словами в предложении.
\item Семантический анализ, при котором определяется смысл фраз, например, в рамках следующих задач:
\begin{itemize}[label=---]
	\item определение тональности текста (общей либо по отношению к некоторому объекту);
	\item выделение именованных сущностей (персоналий, локаций, должностей);
	\item  разрешение анафоры --- если считать местоимения указателями, то можно провести аналогию между этой задачей и разыменованием указателя, т.е. требуется определить, на какое слово в тексте ссылается местоимение.
\end{itemize}
\item Просодическая обработка текста --- придание тексту интонационного оформления.
\item Построение транскрипции по правилам.
\item Вычисление физических параметров (частоты основного тона, спектра, интенсивности и длительности) интонации для синтагм синтезируемого текста.
\item Выбор наиболее подходящих звуковых элементов из базы данных (фраз, предложений, слов, слогов и т.д.).
\end{enumerate}

Речевые синтезаторы делятся на два типа по ограниченности словарной базы \cite{quality}. 
\begin{enumerate}
	\item С ограниченной словарной базой. В синтезаторах с ограниченным словарем речь хранится в виде отдельных слов или предложений, которые выводятся в определенной последовательности в процессе синтеза речевого сообщения. Все фразы в таких системах произносятся диктором заранее.
	\item C неограниченной словарной базой. В синтезаторах с неограниченным словарем элементами речи являются фонемы или слоги, поэтому в них слова строятся по фонетическим правилам. Системы данного типа являются перспективными, т.к. они работают с любым подходящим словарем.
\end{enumerate}

На сегодняшний день в сфере синтеза речи выделяют три основные группы методов по принципу работы, а также гибридный метод.
\begin{enumerate}
	\item Параметрический синтез. Речевой сигнал представлен набором непрерывно изменяющихся во времени параметров. Данный метод речевого синтеза целесообразно использовать в случаях, когда набор текстовых сообщений ограничен и редко подвержен изменению.
	\item Компилятивный синтез. Принцип работы данного метода заключается в составлении сообщения из предварительно записанного словаря исходных элементов синтеза. Очевидно, что содержание синтезируемых сообщений фиксируется объёмом словаря.
	\item Синтез речи по фонетическим правилам. В этом методе часто в качестве исходных элементов используются полуслоги (дифоны) --- сегменты, содержащие половину согласного и половину примыкающего к нему гласного. При этом появляется возможность синтезировать речь по заранее не заданному тексту. При синтезе речи данным методом могут возникать проблемы управления интонационными характеристиками. Качество такого синтеза не совсем соответствует качеству естественной речи, поскольку на границах «сшивки» дифонов часто возникают искажения.
	\item Гибридный метод. Оптимальная последовательность звуковых элементов подбирается из речевого корпуса диктора по классическому алгоритму Unit Selection, но с применением статистической интонационной модели, обученной на той же базе, что позволяет повысить естественность звучания синтезируемой речи по сравнению с реализацией на Unit Selection или только на основе технологии скрытых марковских моделях \cite{hybrid}.
\end{enumerate}

\subsection{Оценка качества синтезированной речи}

При разработке систем автоматического синтеза речи очень важным является вопрос оценки качества синтеза речи \cite{quality}.
В процессе оценки качества учитываются следующие основные характеристики:
\begin{itemize}[label=---]
\item разборчивость речи;
\item естественность (натуральность) речи;
\item мультимодальность речи;
\item многоязычие.
\end{itemize}

Основным критерием оценки качества синтеза речи является разборчивость синтезированной речи. Оценка разборчивости речи производится в соответствии с ГОСТ Р 59880-2021.

Еще одной важной характеристикой, используемой для оценки качества синтезатора речи, является естественность (натуральность) речи. 
Натуральность речи можно оценить, но нет объективных критериев --- это только субъективное впечатление слушателя. 
Ведь даже разные люди имеют разное произношение, которое иногда может даже показаться неестественным. Такую оценку можно получить с привлечением респондентов.

Мультимодальность речи --- это отражение эмоционального состояния говорящего, индивидуальность его голоса, стиль речи, акцент и т.п. --- это будут различные модальности речи.
В системах автоматического синтеза речи эта характеристика выражается в возможности синтеза различных типов голосов и их индивидуальных особенностей. 

Многоязычие относится к лингвистическим способностям и подразумевает возможность синтеза речи на нескольких естественных языках. 

\subsection{Метод Unit Selection}

Метод Unit Selection в настоящее время является основной технологией автоматического синтеза речи, так как он позволяет получать синтезированную речь, которая по своим характеристикам наиболее приближена к естественной \cite{synthesis_book}.

Метод является разновидностью конкатенативного синтеза речи, то есть в процессе создания речевого сигнала используются заранее записанные звуки естественной речи. 
В данном методе для каждой базовой единицы синтеза производится выбор наиболее подходящего элемента из множества вариантов. 
Для этого записываются специальные звуковые базы, размер которых может составлять до нескольких десятков часов звучащей речи. 
В процессе синтеза метод строит оптимальную последовательность звуковых единиц, в которой учитывается, насколько выбранный элемент соответствует описанию необходимых характеристик звука и насколько хорошо каждый из выбранных элементов будет соединяться с соседними. 
При этом из базы в качестве оптимальных могут быть выбраны не отдельные звуки, а их цепочки или даже целые предложения. 
Такой подход позволяет минимизировать модификации речевого сигнала, что повышает естественность синтезируемой речи.

Преимущество метода --- при наличии подходящих звуков в базе, получается речь высокого качества.

Недостатки метода:
\begin{itemize}[label=---]
	\item объем базы данных;
	\item критична полнота базы звуковых данных, в том числе доя возможности синтеза речи в заданной модальности в базе должны присутствовать фрагменты речи в этой же тональности;
	\item существует проблема смены диктора;
	\item качество синтеза ухудшается в случае отсутствия подходящего звукового элемента в базе данных. 
\end{itemize}

\subsection{Синтез, основанный на скрытых марковских моделях}

Данный тип синтеза является гибридом подходов, основанных на правилах и речевом корпусе. 
В этом случае происходит описание звуковой базы данных параметрической моделью.
Параметры обобщаются множеством статистических моделей, представляющих собой скрытые марковские модели, которые содержат в себе шаблоны речевых элементов \cite{hmm}.

По сравнению с методом Unit Selection, подход, в основе которого лежат модели речи, имеет следующие преимущества и недостатки.

Преимущества метода:
\begin{itemize}[label=---]
	\item автоматическое обучение моделей c выбором параметров (например, спектральных характеристик, частоты основного тона, длительности и т.д.), которое возможно выполнять на относительно небольшом речевом материале, позволяет существенно сократить объем требуемой памяти;
	\item в речи не наблюдаются разрывы, присутствующие при конкатенативном синтезе;
	\item синтез, основанный на моделях, позволяет легко модифицировать характеристики голоса.
\end{itemize}

Недостатки метода --- речь, полученная на основе моделей, более роботизирована, чем при Unit Selection.

\section{Методы генерации междометий}

Междометие --- это часть речи, объединяющая неизменяемые слова, которые выражают эмоции и волевые побуждения, не назы­вая их. 
Проблема синтеза междометий заключается в том, что в разном контексте одно и то же междометие может звучать по-разному из-за разной эмоциональной окраски фразы.
Использование междометий — это один из типичных для человеческого общения способов выражения отношения к предмету разговора в повседневной жизни.

Существуют несколько задач генерации междометий.
\begin{enumerate}
	\item Генерация междометия в текстовом виде, а также выбор интонации и транскрипции.
	\item Генерация междометия в звуковом виде на основе заранее заданного текста и, возможно, метаописания междометия.  
	\item Генерация звучащего междометия на основе модификации готовой речи:
	\begin{itemize}[label=---]
		\item аналитическая модификация на основании выбранного математического аппарата и анализа амплитудно-частотных характеристик записи;  
		\item модификация записи на основе машинного обучения.
		\end{itemize}
\end{enumerate}

Результаты решения первой задачи могут служить входом для методов решения второй и третьей задач. Пусть есть запись междометия <<ааа>> с монотонной интонацией и продолжительностью 2 секунды. Если задать метаописание междометия (например, номерной тип интонации <<радость 2>>), возможно преобразовать базовую запись согласно требуемой интонации при совпадении транскрипции.

Для русского языка Yandex.SpeechKit --- одно из лучших открытых программных средств озвучки речи, но его недостаток заключается в том, что не может генерировать междометия. Также существует сервис от Сбера SaluteSpeech. В отличие от SpeechKit в нем присутствует возможность добавления междометий в синтезируемую речь, но их количество довольно ограничено, а также их настройка произодится автоматически \cite{sber}.

Основные этапы метода синтеза звучащих междометий, который сможет решить третью задачу, следующие. 
\begin{enumerate}
	\item Анализ базовой записи голоса и извлечение признаков записи голоса(например, высоты, частоты, длительности звука). 
	\item Модификация признаков записи голоса.
	\item Синтез модифицированной записи голоса. 
\end{enumerate}

Одним из наиболее трудоемких этапов синтеза междометий является модификация признаков записи голоса.
Характеристики сигнала бывают трёх типов \cite{param_signals}: временные, частотные, энергетические. 
Признаки извлекаются для формирования набора характеристик, информативно отражающих свойства исходных данных. 
Это позволяет уменьшить размерность звуковой базы. 

\section{Алгоритмы модификации записи голоса}

Естественность сигнала зависит от объема речевой базы, которая содержит различные звуковые единицы с различной частотой основного тона. В современных системах такие базы достигают десяти часов речи. Однако даже такого количества недостаточно и приходится прибегать к модификации.

\subsection{Алгоритм TD-PSOLA}

Широко распространены алгоритмы, работающие во временной области, наиболее популярным из которых является технология TD-PSOLA (Time-Domain Pitch-Synchronous-Overlap-Add) \cite{mod_algo}. Данный алгоритм работает периодосинхронно, т.е. каждый обрабатываемый фрагмент представляет собой один период. Обязательным условием для этого является возможность определить частоту основного тона сигнала с высокой точностью, т.к. от этого напрямую зависит качество работы этого алгоритма. Далее сигнал разбивается на фрагменты, взвешенные окном Хеннинга, которое захватывает два соседних периода с перекрытием в один период.

Эти взвешенные фрагменты затем могут быть перекомбинированы путём перемещения их центров и наложением с добавлением перекрывающихся частей.
Непосредственная модификация частоты основного тона выполняется путём распределения полученных взвешенных фреймов на новые значения частоты. 

При сохранении длительности фонограммы, в целом слушатели не замечают неестественностей в сигнале при небольших модификациях частоты основного тона \cite{taylor_2009}.

Когда алгоритм применяется для модификации речи, которой выделяются периоды, качество его работы чрезвычайно высоко, и пока степень изменения частоты основного тона не слишком значительна $(\pm10~\%)$ от оригинала, качество речи может быть «идеальным», в том смысле, что слушатель не может заметить в речи какой-то неестественности. С точки зрения вычислительной нагрузки на аппаратные ресурсы, алгоритм прост и может применяться в приложениях реального времени \cite{td_comp}. Поэтому зачастую TD-PSOLA рассматривается как приемлемое решение для проблемы модификации частоты основного тона. Также, работая во временной области, он вносит неконтролируемые искажения в сигнал и, при уменьшении частоты основного тона, существенно редуцируется энергия на границах <<склеек>> фреймов.

\subsection{Алгоритм SPECINT}

В связи с психоакустическими эффектами малейшие искажения в относительном положении формант и изменения огибающей (зависимость амплитуды сигнала от времени) основного тона ведут к побочным эффектам, из-за которых речь становится неестественной, непривычной для нашего восприятия, как следствие, человек при её прослушивании быстро утомляется и не может длительное время внимательно её воспринимать. Поэтому одним из основополагающих действий является получение огибающей основного тона исходного сигнала и её воспроизведение на сигнале новой длительности.

Немаловажно сохранение энергетической огибающей (зависимость амплитуды от частоты), поскольку при увеличении или уменьшении частоты основного тона появляются неизбежные её искажения, что также приводит к снижению естественности речи.

Перед тем как понизить или повысить основной тон, увеличить или уменьшить длительность, необходимо получить значения основного тона на всём модифицируемом участке. При модификации нужно изменить требуемые характеристики аллофонов так, чтобы огибающая основного тона осталась прежней, то есть измениться должен только масштаб (частоты и времени), иначе при малейшем изменении спектральной картины будут слышны режущие слух новые интонации в речи даже при незначительных модификациях. На каждом периоде аллофона вычисляется значение его основного тона, заполняется вектор значений. Далее полученная огибающая изменяется по тону, затем путём сплайн-интерполяции она растягивается или сжимается на требуемую длительность. В итоге получается модель аллофона после модификации, под которую модифицируется исходный аллофон.

Модификация сигнала под требуемую модель происходит следующим образом.

Путём дискретного преобразования Фурье (ДПФ) получается спектр сигнала и рассматриваются отдельно вещественные и мнимые его составляющие.

В спектральной области на частотах, кратных частоте периода, образуются пики(локальные максимумы). Далее эти пики интерполируются на весь диапазон частот, равный половине частоты дискретизации, и вычисляются значения сплайнов в точках, соответствующих пикам нового периода. После выполнения обратного ДПФ получается период с требуемой частотой.

Однако при таком подходе без дополнений невозможно контролировать амплитуду результирующего сигнала, т.е. её абсолютное значение будет отличным от исходного, что сделает сигнал громче или тише \cite{furie}.

Для сохранения исходных величин амплитуды вычисляется нормирующий коэффициент, на который домножаются значения коэффициентов вещественной и мнимой части. В результате получаются пики, находящиеся на огибающей, которая нормирована таким образом, чтобы после обратного ДПФ получились те же значения амплитуд, как и в исходном сигнале.

Данный алгоритм позволяет получать хорошее качество модификации при увеличении или уменьшении частоты основного тона до двух раз \cite{mod_algo}. Особенно хорошие результаты получаются в случаях, когда сигнал уже имеет естественную огибающую частоты основного тона. Хотя для высоких частот основного тона существует лишь малое количество гармоник для точного формирования данной огибающей, что сказывается на качестве результата. Также существенным недостатком для применения данного метода является потребление огромного количества вычислительных ресурсов \cite{mod_algo}, так как выполняются сложные математические операции, такие как, например ДПФ.
%В связи с \textcolor{orange}{психоакустическими} эффектами малейшие искажения в относительном положении \textcolor{orange}{формант}, изменения
% \textcolor{orange}{огибающей} основного тона ведут к побочным эффектам, из-за которых речь становится неестественной, непривычной для нашего восприятия, как следствие человек при её прослушивании быстро утомляется и не может длительное время внимательно её воспринимать \cite{}. Вследствие этого одним из основополагающих действий является получение огибающей основного тона исходного сигнала и её воспроизведение на сигнале новой \textcolor{orange}{длины}.
% длина в метрах?

%Немаловажно сохранение энергетической огибающей, поскольку при увеличении или уменьшении частоты основного тона появляются неизбежные её искажения, что также приводит к снижению естественности речи.

%Перед тем, как понизить или повысить основной тон, увеличить или уменьшить длительность, необходимо получить значения основного тона на всём модифицируемом участке. При модификации необходимо изменить требуемые \textcolor{orange}{характеристики аллофонов} так, чтобы траектория основного тона осталась прежней, т.е. измениться должен только масштаб (частоты и времени), иначе при малейшем изменении спектральной картины мы услышим режущие слух, новые интонации в речи даже при незначительных модификациях.

%Для этого анализируется сигнал с целью получения вектора значений частоты основного тона на всём его протяжении. В системе синтеза русской речи это аллофон. То есть на каждом периоде аллофона вычисляется значение его основного тона, заполняется некоторый массив данных (вектор значений). Далее полученная огибающая изменяется по тону (поднимается или опускается), затем путём сплайн-интерполяции она растягивается или сжимается на требуемую длину. В итоге получаем модель аллофона после модификации, под которую мы должны модифицировать исходный аллофон.
% вектора значений частоты основного тона на всём его протяжении (...) это аллофон. 
% след.фраза: на каждом периоде аллофона вычисляется значение его основного тона, заполняется некоторый массив данных (вектор значений)
% основной тон от основного тона

%\textbf{Модификация сигнала посредством периодосинхронного дискретного преобразования Фурье}

%Модификация сигнала под требуемую модель происходит следующим образом \cite{furie}. Каждый период модифицируется под необходимые параметры. Путём дискретного преобразования Фурье получаем спектр сигнала. В спектральной области получаются пики на частотах, кратных частоте периода. Далее пики интерполируются на весь диапазон частот, равный половине \textcolor{orange}{частоты дискретизации}, и вычисляются значения \textcolor{orange}{сплайнов} в точках, соответствующих пикам нового периода. Далее, выполнив обратное дискретное преобразование Фурье, получается период с требуемой частотой.

%Однако при таком подходе без дополнений невозможно контролировать амплитуду результирующего сигнала. Точнее огибающая амплитуды сохранится, но абсолютное её значение будет отличным от исходного, что сделает сигнал громче или тише, т.к. этот параметр напрямую зависит от того, повышается или понижается основной тон. С увеличением частоты основного тона амплитуда уменьшается, с уменьшением --- увеличивается.

%Для сохранения исходных величин амплитуды вычисляется нормирующий коэффициент, на который умножаются значения коэффициентов вещественной и мнимой части. В результате получаются пики, находящиеся на огибающей, которая нормирована таким образом, чтобы после обратного ДФП получились те же значения амплитуд, как и в исходном сигнале.

%\textbf{Модификация длительности}

%Изменения основного тона приводят к изменению длины аллофона, звук которого подвергается модификации. Это обуславливает потерю естественности, диктор начинает говорить то быстрее, то медленнее. Подобное явление нередко возникает и при компилятивном синтезе. В таких случаях появляется необходимость исправить длительность аллофонов.

%Повышение основного тона периодического сигнала по вышеописанному алгоритму уменьшает его длительность. Для её восстановления обычно используется повтор периодов сигнала. При этом необходимо избежать возникновения двух основных дефектов, снижающих качество синтезированного сигнала. Первый связан с тем, что при каждом повторе сбивается фаза сигнала, что выражается в характерном потрескивании при воспроизведении сигнала. Второй связан с тем, что многократное повторение одного периода человеческое ухо воспринимает как гудение или звон.

\section{Классификация алгоритмов модификации записи голоса}

Для классификации алгоритмов модификации записи голоса были выбраны следующие критерии.
\begin{enumerate}
	\item Качество синтезируемой речи. 
	\item Степень изменения --- критерий, показывающий, насколько сильно характеристики исходного звука отличаются от измененных.
	\item Побочные эффекты, возникающие после обработки звука алгоритмом (например, при использовании алгоритма TD-PSOLA на границах <<склеек>> фреймов уменьшается громкость). 
	\item Трудоемкость алгоритма.
\end{enumerate}

В таблице \ref{tbl:class} приведена классификация по выделенным критериям.

\captionsetup{justification=raggedright, singlelinecheck=false}

\begin{table}[H]
	\begin{threeparttable}
	\caption{\label{tbl:class}Классификация алгоритмов записи голоса}

		\begin{tabular}{|c|c|c|}
		\hline		
		\multirow{2}{*}{\makecell{Критерий \\ сравнения}}& \multicolumn{2}{c|}{Алгоритм}\\\cline{2-3}
		& TD-PSOLA & SPECINT \\\hline
		Качество речи & высокое & высокое \\\hline
		Степень изменения & до 10\% от оригинала & до 100\% от оригинала \\\hline
		Побочные эффекты & редуцируется энергия  & изменение дли- \\
		& на границах &ны аллофона\\
		&<<склеек>> фреймов&\\\hline
		Трудоемкость & низкая & высокая \\\hline
\end{tabular}
\end{threeparttable}
\end{table}

Из таблицы видно, что алгоритм TD-PSOLA применим в системах когда требуется выполнить быструю но небольшую коррекцию звука. Алгоритм SPECINT хорошо подойдет там, где необходимо довольно сильно изменить голос диктора, при отсутствии временных рамок. 

\section*{Вывод}

В данном разделе были описаны методы синтеза речи, их достоинства и недостатки, а также описаны этапы синтеза междометий. Были проанализированы алгоритмы модификации записи голоса, описаны их достоинства и недостатки. 