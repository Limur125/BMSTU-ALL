\chapter*{\hfill ВВЕДЕНИЕ \hfill}
\addcontentsline{toc}{chapter}{ВВЕДЕНИЕ}

Синтез речи – искусственное создание звучащей речи человека. 
Одним из самых важных свойств задачи синтеза является качество получаемой речи. 
Применение технологии синтеза речи на современном коммерческом уровне зависит именно от качества. 
Под системами автоматического синтеза речи понимают системы, преобразующие текст и другую информацию в звучащую речь. 

Технология автоматического синтеза речи применяется в самых 
различных отраслях и направлениях:

\begin{itemize}[label=---]
\item телекоммуникации (call-центры);
\item мобильные устройства;
\item промышленные и бытовые электронные устройства;
\item автомобильная индустрия;
\item образовательные системы;
\item Internet-сервисы;
\item системы ограничения доступа;
\item аэрокосмическая промышленность;
\item военно-промышленный комплекс.
\end{itemize}

Синтезаторы речи обладают широкими возможностями применения. 
Например, позвонив в информационную службу, можно уже услышать не роботизированную речь, а приятный естественный голос. 
Автоинформационная система с технологией синтеза речи, вступит в беседу с каждым дозвонившимся и поможет в получении информации. 
Такая система освобождает операторов от ответов на часто повторяющиеся вопросы.
Также технология синтеза речи открывает широкие возможности для людей с ограниченными возможностями здоровья. 
Для слепых и слабовидящих разработаны говорящие машины. 
Для немых предусмотрены специальные устройства синтеза речи, в которых сообщение набирается на клавиатуре, что позволяет им без проблем общаться с другими людьми.

На сегодняшний день благодаря электронным словарям и переводчикам на основе технологии синтеза речи возможно изучение иностранных языков с постановкой правильного произношения. 
Еще одним примером синтеза речи могут служить различные системы звукового оповещения: телефонная справочная информация, объявление станций в метро, информация об отправлении автобуса или поезда.

Цель научно-исследовательской работы: классификация существующих методов генерации междометий.

Задачами данной научно-исследовательской работы являются:
\begin{itemize}[label=---]
	\item описание существующих методов генерации звучащей речи;
	\item проведение анализа предметной области генерации междометий;
	\item выделение критериев сравнения методов генерации междометий;
	\item классификация методов генерации междометий по выделенным критериям.
\end{itemize}
