\chapter*{Введение}
\addcontentsline{toc}{chapter}{Введение}

В современном мире компьютерная графика находит свое применение в самых различных областях. Типичный пример – это кино и игры.

Алгоритмы получения реалистичных изображений на сегодняшний день получают очень много внимания. Эти алгоритмы сложны в реализации и являются одними из самых затратных по ресурсам. Они должны учитывать не только формы и текстуры объектов, но и множество других физических явлений, таких как преломление, отражение, рассеивание света.
 
Очевидно, что получение более качественного изображения на выходе алгоритма, требует больше времени и памяти для синтеза. Хотя для генерации статических изображений это не является проблемой, при создании динамической сцены могут возникать большие сложности, так как каждый раз на очередном временном интервале придется рассчитывать все заново.

Целью данной курсовой работы является создание реалистичной сцены, визуализирующей такое явление, как распространение света в неравномерно задымленном помещении. Явление будет продемонстрировано на примере комнаты, в которой находятся источник дыма и источник света.

Для достижения поставленной цели, необходимо решить следующие задачи:
\begin{itemize}[label=---]
	\item проанализировать существующие алгоритмы построения изображения и выбрать из них те, что в лучше помогут в решении поставленной задачи;
	\item описать модель трехмерной сцены, в том числе и объекты, из которых она состоит;
	\item реализовать выбранные алгоритмы;
	\item разработать программу для отображения сцены;
	\item провести эксперимент по замеру производительности полученного программного обеспечения.
\end{itemize}

