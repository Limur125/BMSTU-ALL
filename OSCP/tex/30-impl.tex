\chapter{ТЕХНОЛОГИЧЕСКАЯ ЧАСТЬ}

\section{Выбор средств реализации}

В качестве языка программирования был выбран язык C. На этом языке
реализованы все модули ядра и драйверы операционной системы Linux.

\section{Реализация изменения чувствительности мыши}

В листинге~\ref{lst:mouse-irq} представлена реализация алгоритма перехвата событий мыши. 

\captionsetup{justification=raggedright,singlelinecheck=off}
\begin{lstlisting}[label=lst:mouse-irq, caption= Реализация алгоритма перехвата событий мыши]
static void usb_mouse_irq(struct urb *urb)
{
	struct usb_mouse *mouse = urb->context;
	signed char *data = mouse->data;
	struct input_dev *dev = mouse->dev;
	int status;
	
	float delta_x = data[1] * PRE_SCALE_X;
	float delta_y = data[2] * PRE_SCALE_Y;
	
	static float carry_x = 0.0f;
	static float carry_y = 0.0f;
	
	analyze_wheel(&delta_x, &delta_y, data[3]);
	
	delta_x *= POST_SCALE_X;
	delta_y *= POST_SCALE_Y;
	delta_x += carry_x;
	delta_y += carry_y;
	carry_x = delta_x - my_round(delta_x);
	carry_y = delta_y - my_round(delta_y);
	
	switch (urb->status) {
		case 0:	
		break;
		case -ECONNRESET:
		case -ENOENT:
		case -ESHUTDOWN:
		return;
		default:
		goto resubmit;
	}
	
	input_report_key(dev, BTN_LEFT,   data[0] & 0x01);
	input_report_key(dev, BTN_RIGHT,  data[0] & 0x02);
	input_report_key(dev, BTN_MIDDLE, data[0] & 0x04);
	input_report_key(dev, BTN_SIDE,   data[0] & 0x08);
	input_report_key(dev, BTN_EXTRA,  data[0] & 0x10);
	
	input_report_rel(dev, REL_X,     Leet_round(delta_x));
	input_report_rel(dev, REL_Y,     Leet_round(delta_y));
	input_report_rel(dev, REL_WHEEL, data[3]);
	
	input_sync(dev);
	resubmit:
	status = usb_submit_urb (urb, GFP_ATOMIC);
	if (status)
	dev_err(&mouse->usbdev->dev,
	"can't resubmit intr, %s-%s/input0, status %d\n",
	mouse->usbdev->bus->bus_name,
	mouse->usbdev->devpath, status);
}
\end{lstlisting}

В листинге~\ref{lst:mouse-anal} представлена реализация алгоритма анализа событий колесика мыши. 

\captionsetup{justification=raggedright,singlelinecheck=off}
\begin{lstlisting}[label=lst:mouse-anal, caption= Реализация анализа событий колесика мыши]
	static void analyze_wheel(float *delta_x, float *delta_y, int wheel)
	{
		
		if(wheel > 0)
		{
			printk("ACCELERATOR: Increase mouse speed. accel_sens = %d", (int)(accel_sens*10));
			accel_sens += ACCELERATION;
			if (SENS_CAP > 0 && accel_sens >= SENS_CAP) {
				accel_sens = SENS_CAP;
			}
			
		}
		else if(wheel < 0)
		{
			printk("ACCELERATOR: Decrease mouse speed. accel_sens = %d", (int)(accel_sens*10));
			accel_sens -= ACCELERATION;
			if (accel_sens < 0.1f) {
				accel_sens = 0.1f;
			}
			
		}
		
		accel_sens /= SENSITIVITY;
		*delta_x *= accel_sens;
		*delta_y *= accel_sens;
	}
\end{lstlisting}

Для определения нажатой клавиши используется поле \textit{data} из структуры usb\_mouse. Биты байта \textit{data[0]} отвечают за тип нажатой клавиши. Байты \textit{data[1]} и \textit{data[2]} отвечают за перемещение мыши горизонтали и вертикали соответственно. Байт \textit{data[3}] отвечает за вращение колесика мыши, при вращении вверх он равен $1$, при вращении вниз --- $-1$, когда колесико не вращают --- нулю.

\section{Реализация функции регистрации драйвера}

В приложении А представлена реализация функции probe.

В листинге~\ref{lst:disconnect} представлена реализация функции disconnect.
\begin{lstlisting}[label=lst:disconnect, caption=Реализация функции disconnect]
static void usb_mouse_disconnect(struct usb_interface *intf)
{
	struct usb_mouse *mouse = usb_get_intfdata (intf);
	
	usb_set_intfdata(intf, NULL);
	if (mouse) {
		usb_kill_urb(mouse->irq);
		input_unregister_device(mouse->dev);
		usb_free_urb(mouse->irq);
		usb_free_coherent(interface_to_usbdev(intf), 8, mouse->data, mouse->data_dma);
		kfree(mouse);
	}
}
\end{lstlisting}

В листинге~\ref{lst:open} представлена реализация функции open.
\begin{lstlisting}[label=lst:open, caption=Реализация функции open]
static int usb_mouse_open(struct input_dev *dev)
{
	struct usb_mouse *mouse = input_get_drvdata(dev);
	
	mouse->irq->dev = mouse->usbdev;
	if (usb_submit_urb(mouse->irq, GFP_KERNEL))
		return -EIO;
	
	return 0;
}
\end{lstlisting}

В листинге~\ref{lst:close} представлена реализация функции close.
\begin{lstlisting}[label=lst:close, caption=Реализация функции close]
static void usb_mouse_close(struct input_dev *dev)
{
	struct usb_mouse *mouse = input_get_drvdata(dev);
	
	usb_kill_urb(mouse->irq);
}
\end{lstlisting}

\section{Реализация Makefile}

В листинге~\ref{lst:make} представлена реализация Makefile.
\begin{lstlisting}[label=lst:make, caption=Реализация Makefile]
obj-m += accelerator.o
ccflags-y += -msse -mpreferred-stack-boundary=4

all:
	make -C /lib/modules/$(shell uname -r)/build M=$(PWD) modules

clean:
	make -C /lib/modules/$(shell uname -r)/build M=$(PWD) clean
\end{lstlisting}