\chapter*{ВВЕДЕНИЕ}
\addcontentsline{toc}{chapter}{ВВЕДЕНИЕ}

Чувствительность мыши --- это важная характеристика мыши. Она отвечает за то, насколько быстро перемещается указатель мыши при перемещении мыши. При более высокой чувствительности указатель движется быстрее и проходит большее расстояние на экране, чем пользователь физически перемещает мышь. При более низкой чувствительности указатель движется медленнее и требует больше усилий для перемещения по экрану, но обеспечивает лучшую точность. 

Мышь -- координатное устройство для управления курсором и отдачи различных команд компьютеру.
Оно широко распространено среди пользователей компьютеров, поэтому для более гибкой настройки чувствительности имеет смысл добавить возможность ее с помощью колесика.

Тема курсовой работы -- разработать программное обеспечение, позволяющее изменять чувствительность USB-мыши с помощью колесика.