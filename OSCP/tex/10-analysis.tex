\chapter{Аналитический раздел}

\section{Постановка задачи}
В соответствии с заданием на курсовую работу по дисциплине <<Операционные системы>> требуется разработать программное обеспечение, позволяющее изменять чувствительность USB-мыши с помощью колесика.

Для выполнения задания требуется решить следующие задачи:
\begin{enumerate}
	\item Провести анализ существующих подходов к изменению функциональности внешних устройств в Linux.
	\item Разработать алгоритмы, необходимые для реализации программное обеспечение.
	\item Разработать ПО, предоставляющая требуемую функциональность.
	\item Провести исследование разработанного программное обеспечение.
\end{enumerate}

При двойном нажатии на колесико мыши долена включиться настройка чувствительности и выключиться старая функциональность колесика. При прокрутке колесика мыши вниз чувствительность мыши должна понижаться, при прокрутке вверх --- увеличиваться. Для отключения настройки и восстановления старой функциональности нужно один раз нажать на колесико мыши.
Функциональность мыши должна сохраниться: с её помощью можно перемещать курсор и нажимать правую и левую кнопки мыши.

Для разработки и тестирования данной работы используется мышь Logitech B100~\cite{mouse} и операционная система Ubuntu-22.04~\cite{ubuntu}.

\section{Анализ драйверов устройства}
Драйверы устройств являются одной из разновидностей модулей ядра.
Драйверы полностью скрывают детали, касающиеся работы устройства и предоставляют четкий программный интерфейс для работы с аппаратурой. 
В Unix каждое аппаратное устройство представлено файлом устройства в каталоге /dev. 

\subsection{Анализ возможных типов драйвера}
В Unix/Linux драйверы бывают трех типов:
\begin{itemize}[label=---]
	\item встроенные -- выполнение этих драйверов инициализируется при запуске системы;
	\item драйверы, код которых поделен между ядром и специальной утилиты;
	\item драйверы, реализованные как загружаемые модули ядра.
\end{itemize}

Среди последних выделяют HID-драйверы. 
Класс HID является одним из наиболее часто используемых классов USB. 
Класс HID состоит в основном из устройств, предназначенных для интерактивного взаимодействия с компьютером. 

Для изменения функциональности мыши требуется разработать именно HID-драйвер.

\subsection{Анализ алгоритма регистрации USB-драйвера в системе}
Для выполнения задания требуется разработать драйвер мыши. Регистрация USB-драйвера подразумевает~\cite{usb_drivers}:
\begin{enumerate}
	\item Заполнение структуры \textit{usb\_driver}.
	\item Регистрация структуры в системе.
\end{enumerate}

Сначала требуется инициализировать поля структуры \textit{usb\_driver}. 

Структура usb\_driver состоит из следующих полей~\cite{usb_driver}:
\begin{itemize}[label=---]
	\item \textbf{name} -- имя драйвера, должно быть уникальным среди USB-драйверов.
	\item \textbf{id\_table} -- массив структур \textit{usb\_device\_id}, который содержит список всех типов USB-устройств, которые обслуживает драйвер.
	\item \textbf{probe} -- функция обратного вызова, является точкой входа драйвера. Она будет вызвана только для тех устройств, которые соответствуют параметрам, перечисленным в структуре \textit{usb\_device\_id}.
	\item \textbf{disconnect} -- функция обратного вызова, которая вызывается при отключении устройства от драйвера.
\end{itemize}

В функции \textbf{probe} для каждого подключаемого устройства выделяется структура в памяти, заполняется, затем регистрируется, например, символьное устройство, и проводится регистрация устройства в \textit{sysfs}.

При установке собственного драйвера сначала необходимо выгрузить модуль \textit{usbhid}, который автоматически регистрирует все стандартные драйверы в системе.
Данный модуль устанавливает стандартный драйвер мыши и не позволяет установить свой.

\section{Анализ подсистемы ввода/вывода USB}
Подсистема ввода/вывода выполняет запросы файловой подсистемы и подсистемы управления процессами для доступа к периферийным устройствам (дискам, магнитным лентам, терминалам и т.д.). Она обеспечивает необходимую буферизацию данных и взаимодействует с драйверами устройств — специальными модулями ядра, непосредственно обслуживающими внешние устройства.

Для использования подсистемы ввода/вывода требуется инициализировать структуру \textit{input\_dev} \cite{input_dev}. 
Поле evbit этой структуры отвечает за то, какие события могут происходить на устройстве.
Для мыши возможны два вида событий: \textit{EV\_KEY} \cite{EV_KEY} -- нажатия кнопок мыши, и \textit{EV\_REL} -- изменения относительного положения курсора на экране.

Для вызова событий, связанных с клавишами используется системный вызов \textit{input\_report\_key} \cite{input_report_key}, который принимает устройство ввода, кнопку, на которую вызывается событие, и дополнительная информация о событии.

\section{URB}
Сообщение, передаваемое от драйвера USB-устройства системе, называется USB Request Block или URB \cite{ldd}. 
Оно описывается структурой \textit{struct urb}.
URB используется для передачи или приёма информации от конечной точки на заданное USB-устройство в асинхронном режиме \cite{urb}.
Каждая конечная точка может обрабатывать очередь из URB, следовательно, на одну конечную точку может быть выслать множество URB.
URB создаются динамически и содержат внутренний счётчик ссылок, что позволяет автоматически освобождать память, когда блок запроса больше никем не используется~\cite{ldd}.

Существуют четыре типа URB.
\begin{enumerate}
	\item control --- используются для конфигурирования устройства во время подключения, для управления устройством и получения статусной информации в процессе работы.
	\item bulk --- применяются при необходимости обеспечения гарантированной доставки данных от хоста к 	функции или от функции к хосту, но время доставки не ограничено. Приоритет у таких передач самый низкий, они могут приостанавливаться при большой загрузке шины. 
	\item interrupt --- используются в том случае, 	когда требуется передавать одиночные пакеты данных небольшого размера. Каждый пакет должен быть передан за определенное время. Операции передачи выполняются асинхронно и должны обслуживаться не медленнее, чем того требует устройство.
	\item isochronous --- применяются для обмена данными в <<реальном времени>>, когда на каждом временном интервале требуется передавать строго определенное количество данных, но доставка информации не гарантирована. Изохронные URB обычно используются в 	мультимедийных устройствах для передачи аудио- и видеоданных.
\end{enumerate}

\section{Анализ способов изменения функциональности внешних устройств}

Для изменения функциональности внешних устройств существует два основных подхода~\cite{ryaz}. 

\begin{enumerate}
	\item Report. Report descriptor определяет структуру report, содержащий всю информацию необходимую USB-хосту для определения формата данных и действий. HID report содержит фактические значения данных без какой-либо дополнительной метаинформации. HID report могут отправляться с устройства (<<Input report>>, т.е. события ввода), на устройство (<<Output report>>, например, для изменения светодиодов) или использоваться для настройки устройства (<<Feature report>>).
	\item URB используется для передачи или приёма данных в или из заданной оконечной точки USB на заданное USB устройство в асинхронном режиме. В зависимости от потребностей, драйвер USB устройства может выделить для одной оконечной точке много URB или может повторно использовать один URB для множества разных оконечных точек. Каждая оконечная точка в устройстве может обрабатывать очередь URB, так что перед тем, как очередь опустеет, к одной оконечной точке может быть отправлено множество URB. 
\end{enumerate}

\section{Анализ USB}

\subsection{Инициализация структуры usb\_driver}

Для создания USB-драйвера создается экземпляр структуры \textit{usb\_driver}. Создание экземпляра приведено на листинге~\ref{lst:usb-driver}.

\begin{lstlisting}[label=lst:usb-driver, caption=Структура usb\_driver]
	static struct usb_driver usb_mouse_driver = {
		.name		= "accelerator",
		.probe		= usb_mouse_probe,
		.disconnect	= usb_mouse_disconnect,
		.id_table	= usb_mouse_id_table,
	};
\end{lstlisting}

Для регистрации USB-драйвера в системе используется системный вызов \textit{usb\_register}.

\subsection{Структура для хранения информации о мыши}

Для передачи данных, связанных с мышью, была создана структура usb\_mouse, приведенная на листинге~\ref{lst:usb-mouse}.
\captionsetup{justification=raggedright,singlelinecheck=false}
\begin{lstlisting}[label=lst:usb-mouse, caption=Структура usb\_mouse]
	struct usb_mouse {
		char name[128];
		char phys[64];
		struct usb_device *usbdev;
		struct input_dev *dev;
		struct urb *irq;
		
		signed char *data;
		dma_addr_t data_dma;
	};
\end{lstlisting}

\begin{itemize}[label=---]
	\item \textit{data} --- данные, передаваемые мышью при прерывании;
	\item \textit{input\_dev} --- структура для использования подсистемы входа/выхода;
	\item \textit{usb\_device} --- представление usb-устройства;
	\item \textit{irq} --- указатель на обработчик прерываний.
\end{itemize}


\subsection{События мыши}

Для реализации драйвера необходимо перехватывать следующие события мыши:

\begin{itemize}[label=---]
	\item нажатие правой кнопки мыши;
	\item нажатие левой кнопки мыши;
	\item нажатие колесика мыши;
	\item вращение колесика мыши;
	\item перемещение мыши.
\end{itemize}

\section*{Выводы}

В результате проведенного анализа было решено:
\begin{itemize}[label=---]
	\item для выполнения задания разработать HID-драйвер, который должен быть реализован как загружаемый модуль ядра;
	\item для изменения функциональности USB-мыши использовать URB interrupt;
	\item для изменения чувствительности мыши будет использован коэффициент, на который будут умножаться координаты перемещения мыши. Каждый раз когда пользователь будет прокручивать колесико мыши вверх, коэффициент будет увеличиваться, тем самым будет увеличиваться расстояние, которое будет проходить курсор на экране.
\end{itemize}
