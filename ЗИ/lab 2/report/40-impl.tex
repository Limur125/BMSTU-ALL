\chapter{Технологический раздел}
\label{cha:impl}

\section{Средства реализации}

В качестве языка программирования для реализации данной лабораторной работы использовался язык программирования С++ \cite{cplusplus}, так как он позволяет работать с файлами и массивами. В качестве среды разработки использовалась Visual Studio Code \cite{vscode}.

\section{Реализация алгоритмов}

В листингах \ref{code:rotor}--\ref{code:enigma} представлена реализация алгоритма шифрования DES.

\begin{lstlisting}[label=code:rotor,caption=Функция шифрования файла]
int FileEncryption::cipher(string input, string output, bool mode)
{
	ifstream ifile;
	ofstream ofile;
	ui64 buffer;
	
	if(input.length()  < 1) 
	ifile = ifstream(stdin);
	else
	ifile.open(input, ios::binary | ios::in | ios::ate);
	
	if(output.length() < 1) 
	ofile = ofstream(stdout);
	else
	ofile.open(output, ios::binary | ios::out);
	
	ui64 size = ifile.tellg();
	ifile.seekg(0, ios::beg);
	
	ui64 block = size / 8;
	if(mode) block--;
	
	for(ui64 i = 0; i < block; i++)
	{
		ifile.read((char*) &buffer, 8);
		if (mode)
		buffer = des.decrypt(buffer);
		else
		buffer = des.encrypt(buffer);
		
		ofile.write((char*) &buffer, 8);
	}
	
	if(mode == false)
	{
		ui8 padding = 8 - (size % 8);
		
		if (padding == 0)
		padding  = 8;
		
		buffer = (ui64) 0;
		if(padding != 8)
		ifile.read((char*) &buffer, 8 - padding);
		
		ui8 shift = padding * 8;
		buffer <<= shift;
		buffer  |= (ui64) 0x0000000000000001 << (shift - 1);
		
		buffer = des.encrypt(buffer);
		ofile.write((char*) &buffer, 8);
	}
	else
	{
		ifile.read((char*) &buffer, 8);
		buffer = des.decrypt(buffer);
		
		ui8 padding = 0;
		
		while(!(buffer & 0x00000000000000ff))
		{
			buffer >>= 8;
			padding++;
		}
		
		buffer >>= 8;
		padding++;
		
		if(padding != 8)
		ofile.write((char*) &buffer, 8 - padding);
	}
	
	ifile.close();
	ofile.close();
	return 0;
}
\end{lstlisting}

\begin{lstlisting}[label=code:reflector,caption=Функция шифратора]
ui64 DES::des(ui64 block, bool mode)
{
	block = ip(block);
	
	ui32 L = (ui32) (block >> 32) & L64_MASK;
	ui32 R = (ui32) (block & L64_MASK);
	
	for (ui8 i = 0; i < 16; i++)
	{
		ui32 F;
		if (mode)
		F = f(R, sub_key[15 - i]);
		else
		F = f(R, sub_key[i]);
		ui32 temp = R;
		R = L ^ F;
		L = temp;
	}
	
	block = (((ui64) R) << 32) | (ui64) L;
	return fp(block);
}
\end{lstlisting}

\begin{lstlisting}[label=code:enigma,caption=Функция Фейстеля]
ui32 DES::f(ui32 R, ui64 k)
{
	ui64 s_input = 0;
	for (ui8 i = 0; i < 48; i++)
	{
		s_input <<= 1;
		s_input |= (ui64) ((R >> (32-EXPANSION[i])) & LB32_MASK);
	}
	s_input = s_input ^ k;
	ui32 s_output = 0;
	for (ui8 i = 0; i < 8; i++)
	{
		char s = (s_input >> (42 - 6 * i)) & 0x3f;
		char row = ((s >> 4) & 0b10) | s & 1;
		char column = (s >> 1) & 0b1111;
		
		s_output <<= 4;
		s_output |= (ui32) (SBOX[i][16*row + column] & 0x0f);
	}
	
	ui32 f_result = 0;
	for (ui8 i = 0; i < 32; i++)
	{
		f_result <<= 1;
		f_result |= (s_output >> (32 - PBOX[i])) & LB32_MASK;
	}
	
	return f_result;
}
\end{lstlisting}

\section{Тестирование реализации алгоритма}

Было проведено тестирование на следующих входных данных:

1. Входящая последовательность байтов:

D09BD0B8D0BDD0B5D0B9D0BDD18BD0B9

Зашифрованный текст:

AB3B4DA99A15DCA7C13358E9D65EA07F
	 
Расшифрованная последовательность байтов:

D09BD0B8D0BDD0B5D0B9D0BDD18BD0B9

2. Входящая последовательность байтов:

4141414141414141

Зашифрованная последовательность байтов:

4D41E973A3BF9604

Расшифрованная последовательность байтов:

4141414141414141


Все тесты пройдены успешно.

\section*{Вывод}
В данном разделе были перечислены средства разработки, с помощью которых был реализованы алгоритм шифрования DES, приведена реализация алгоритма.


