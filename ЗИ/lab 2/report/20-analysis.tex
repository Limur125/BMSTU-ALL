\chapter{Аналитический раздел}
\label{cha:analysis}

\section{Алгоритм шифрования DES}

DES (Data Encryption Standard) - это симметричный шифровальный алгоритм, разработанный в 1970-х годах, который использует блочное шифрование с фиксированной длиной блока в 64 бита. Вот основные шаги и логика работы DES:

1. Начальная перестановка (Initial Permutation):
Исходный текст (64 бита) проходит через начальную перестановку, где биты переставляются в определенном порядке согласно предопределенной таблице перестановок.

2. Раунды (Rounds):
DES состоит из 16 раундов шифрования, каждый из которых включает несколько шагов:
\begin{itemize}
	\item Расширение (Expansion): 32-битный входной блок расширяется до 48 бит путем перестановки и дублирования некоторых битов.
	\item Ключ раунда (Round Key): к 48-битному расширенному блоку применяется 48-битный ключ раунда, полученный из основного ключа DES.
	\item Скремблирование (Substitution): 48-битный блок проходит через S-блоки (Substitution-boxes), которые заменяют блоки по 6 бит на блоки по 4 бита с использованием заранее определенных таблиц замен.
	\item Перестановка (Permutation):  после замены, полученный блок по 32 бита проходит через таблицу перестановки, которая перемешивает биты в блоке.
	\item Обработка ключа (Key Mixing): к полученному блоку применяется операция XOR с ключом раунда для обеспечения взаимодействия ключа и данных.
\end{itemize}

3. Последняя перестановка (Final Permutation):
После 16 раундов, 64-битный блок проходит через последнюю перестановку, обратную начальной перестановке, чтобы получить зашифрованный текст.

Основным элементом DES является ключ, который состоит из 56 бит, и который используется для генерации ключей раунда. Ключ разбивается на две половины, и каждая половина сдвигается влево на определенное количество бит в зависимости от номера раунда. Затем, из полученных половинок формируется ключ раунда.

Таким образом, DES использует комбинацию перестановок, замен и операций XOR для шифрования данных. Эти шаги повторяются 16 раз, в каждом раунде используется уникальный ключ. Результат - зашифрованный блок данных, который без знания правильного ключа практически невозможно расшифровать.

Для DES рекомендовано несколько режимов:

\begin{itemize}
	\item ECB (англ. electronic code book) --- режим «электронной кодовой книги» (простая замена);
	\item CBC (англ. cipher block chaining) --- режим сцепления блоков;
	\item CFB (англ. cipher feed back) --- режим обратной связи по шифротексту;
	\item OFB (англ. output feed back) --- режим обратной связи по выходу.
\end{itemize}

\section*{Вывод}
В данном разделе был рассмотрен алгоритм симметричного шифрования DES.