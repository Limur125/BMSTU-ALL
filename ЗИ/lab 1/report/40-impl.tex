\chapter{Технологический раздел}
\label{cha:impl}

\section{Средства реализации}

В качестве языка программирования для реализации данной лабораторной работы использовался язык программирования С \cite{c_lang}, так как он позволяет работать с файлами и  строками. В качестве среды разработки использовалась Visual Studio.

\section{Реализация алгоритмов}

В листингах \ref{code:struct}, \ref{code:enigma} представлена реализация алгоритма работы шифровальной машины ``Энигма''.

\begin{lstlisting}[label=code:struct, caption=Структура enigma\_t]
typedef struct enigma_t 
{
	int counter;
	int size_rotor;
	int num_rotors;
	char* reflector;
	char* com_panel;
	char** rotors;
} enigma_t;
\end{lstlisting}

\begin{lstlisting}[label=code:enigma,caption=Реализация алгоритма работы шифровальной машины ``Энигма'']
char enigma_encrypt(enigma_t* enigma, char ch, int* rc) 
{
	int rotor_queue;
	char new_ch;
	if (ch - 'A' >= enigma->size_rotor) {
		*rc = 0;
		return 0;
	}
	new_ch = enigma->com_panel[ch - 'A'];
	for (int i = 0; i < enigma->num_rotors; i++)
		new_ch = enigma->rotors[i][new_ch - 'A'];
	new_ch = enigma->reflector[new_ch - 'A'];
	for (int i = enigma->num_rotors - 1; i >= 0; i--) 
	{
		new_ch = enigma_rotor_find(enigma, i, new_ch, rc);
		if (*rc != 0) 
			return 0;
	}
	new_ch = enigma->com_panel[ch - 'A'];
	rotor_queue = 1;
	enigma->counter += 1;
	for (int i = 0; i < enigma->num_rotors; i++) 
	{
		if (enigma->counter % rotor_queue == 0) 
			enigma_rotor_shift(enigma, i);
		rotor_queue *= enigma->size_rotor;
	}
	*rc = 0;
	return new_ch;
}
\end{lstlisting}

\section{Тестирование}

В списке приведены тесты для функции, реализующей алгоритм шифрования машины ``Энигма''.

\begin{table}[H]
	\begin{center}
		\caption{\label{tbl:test} Функциональные тесты}
		\begin{tabular}{|c|c|c|}
			\hline
			Исходный текст&Зашифрованный текст&Расшифрованный текст \\
			\hline
			ААААААА&BDXZBDF&ААААААА \\
			\hline
			HELLOWORLD&RSMONISVWN&HELLOWORLD\\
			\hline
		\end{tabular}
	\end{center}
\end{table}

\section*{Вывод}
В данном разделе были перечислены средства разработки, с помощью которых были реализованы алгоритм работы шифровальной машины ``Энигма'', приведена реализация алгоритма.


