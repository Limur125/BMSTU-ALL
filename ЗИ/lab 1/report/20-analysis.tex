\chapter{Аналитический раздел}
\label{cha:analysis}

\section{Шифровальная машина ``Энигма''}

Шифровальная машина ``Энигма'' внешне выглядит как печатающая машинка, за исключением того факта, что шифруемые символы не печатаются автоматически на определённый лист бумаги, а указываются на панели посредством загорания лампочки.

Шифровальная машина ``Энигма'' обладает тремя основными механизмами.

\textbf{Роторы.} Сердце всех шифровальных машин того времени. Со стороны классической криптографии они реализуют полиалфавитный алгоритм шифрования, а их определённо выстроенная позиция представляет собой один из основных ключей шифрования. Каждый ротор не эквивалентен другому ротору, потому как обладает своей специфичной настройкой. Выбор позиций, для вставки роторов, также играл свою роль, потому как образовывал свойство некоммутативности.

\textbf{Рефлектор (отражатель).} Статичный механизм, позволяющий шифровальным машинам типа``Энигма'' не вводить помимо операции шифрования дополнительную операцию расшифрования. Hефлектор представляет собой частный случай моноалфавитного шифра — парный шифр, особенностью которого является инволютивность шифрования, где функция шифрования E становится равной функции расшифрования D, иными словами E = D. Это приводит к следующим выводам: если существует сообщение M, то справедливы становятся следующие утверждения D(E(M)) = E(E(M)) = M, E(E(E(M))) = E(M) = D(M). 

\textbf{Коммутаторы.} Своеобразный "множитель" возможных вариаций ключей шифрования. Представляет собой также как и рефлектор парный шифр, но в отличие от последнего является динамическим механизмом, то-есть редактируемым и сменяемым. Коммутаторы, можно их рассматривать как некие кабеля, вставляются в коммутационную панель, на которой изображены символы английского алфавита. Один коммутатор имеет два конца, каждый из которых вставляется в два отверстия коммутационной панели. Связь коммутатора с двумя отверствиями, над которыми изображены символы алфавита и представляет собой связь парного шифра между выбранными двумя символами. Так например, если коммутатор был вставлен в два отверстия (Q, D), то это говорит о том, что Q и D стали парными символами при шифровании. На одну шифровальную машину давалось десять коммутаторов, существовало 26 возможных отверстий в коммутационной панели (количество символов алфавита), один коммутатор одновременно связывал два символа такой панели, то-есть все коммутаторы затрагивают в общей сложности 20 отверстий.

\section*{Вывод}
В данном разделе была рассмотрена логика работы шифровальной машины ``Энигма''.