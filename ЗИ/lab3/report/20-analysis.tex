\chapter{Аналитический раздел}
\label{cha:analysis}

\section{Алгоритм шифрования AES}

AES, или Advanced Encryption Standard, - это алгоритм шифрования с симметричным ключом. Это одно из самых универсальных и наиболее любимых технических решений в сфере криптографии.
В основе AES лежит блочный шифр, который использует 128-битный размер блока и 128, 192 или 256-битные ключи для шифрования данных. AES256 - это версия стандарта с 256-битными ключами. Этот стандарт широко считается самым безопасным стандартом цифровой криптографии, который обычно используется для наиболее надежной сквозной шифрованной связи.
AES был разработан двумя бельгийскими криптографами, Джоаном Деменом и Винсентом Риджменом, и был принят в качестве официального стандарта в 2001 году Национальным институтом стандартов и технологий США. Такое достижение свидетельствует о широком признании, которое получил стандарт. Уже более 20 лет AES256 и шифрование AES в целом является одним из наиболее предпочтительных решений для разработчиков, желающих создать систему, в которой коммуникации хорошо защищены от постороннего или внешнего влияния и утечек. Вот основные шаги и логика работы AES:

1. Раунды (Rounds):
Алгоритм AES использует различное количество раундов в зависимости от длины ключа. Он использует 10 раундов для 128-битного ключа, 12 раундов для 192-битного ключа и 14 раундов для 256-битного ключа. Чем больше раундов используется в процессе шифрования, тем надежнее шифрование. Каждый раунд включает в себя следующие шаги.
\begin{itemize}
	\item Алгоритм AES использует блок подстановки (S-Box) для замены значений в процессе шифрования. S-Box --- это таблица значений, которые используются для замены входных значений в процессе шифрования.
	\item Сдвиг строки (ShiftRow). Основная цель сдвига строки заключается в достижении разброса байтов в каждой строке, что представляет собой линейное преобразование.
	\item Смешивание колонок (MixColumn). Смешивание столбцов должно заменить преобразование умножением матрицы состояний и постоянной матрицы C для достижения диффузии в столбцах.
	\item Сложение по модулю 2 с ключом.
\end{itemize}

Основным элементом AES является ключ, который состоит из 128, 192 или 256 бит, и который используется для генерации ключей раунда. Ключ разбивается на четыре части, затем, из полученных  формируется ключ раунда.

Для АES рекомендовано несколько режимов:

\begin{itemize}
	\item ECB (англ. electronic code book) --- режим «электронной кодовой книги» (простая замена);
	\item CBC (англ. cipher block chaining) --- режим сцепления блоков;
	\item CFB (англ. cipher feed back) --- режим обратной связи по шифротексту;
	\item OFB (англ. output feed back) --- режим обратной связи по выходу.
\end{itemize}

\section*{Вывод}
В данном разделе был рассмотрен алгоритм симметричного шифрования AES.