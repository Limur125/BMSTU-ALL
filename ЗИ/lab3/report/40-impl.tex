\chapter{Технологический раздел}
\label{cha:impl}

\section{Средства реализации}

В качестве языка программирования для реализации данной лабораторной работы использовался язык программирования С++ \cite{cplusplus}, так как он позволяет работать с файлами и массивами. В качестве среды разработки использовалась Visual Studio Code \cite{vscode}.

\section{Реализация алгоритмов}

В листингах \ref{code:rotor}--\ref{code:enigma} представлена реализация алгоритма шифрования АES.

\begin{lstlisting}[label=code:rotor,caption=Функция шифрования файла]
int FileEncryption::cipher(string input, string output)
{
	    ifstream ifile;
	ofstream ofile;
	char inbuffer[16], outbuffer[16];
	
	if(input.length()  < 1) 
	ifile = ifstream(stdin);
	else
	ifile.open(input, ios::binary | ios::in | ios::ate);
	
	if(output.length() < 1) 
	ofile = ofstream(stdout);
	else
	ofile.open(output, ios::binary | ios::out);
	size_t size = ifile.tellg();
	ifile.seekg(0, ios::beg);
	
	
	size_t block = size / 16;
	
	for(size_t i = 0; i < block; i++)
	{
		ifile.read(inbuffer, 16);
		aes.Encrypt(inbuffer, 16, outbuffer);
		ofile.write(outbuffer, 16);
	}
	if (size % 16 != 0)
	{
		int padding = 16 - (size % 16);
		
		memset(inbuffer, 0, 16);
		ifile.read(inbuffer, 16 - padding);
		aes.Encrypt(inbuffer, 16, outbuffer);
		ofile.write(outbuffer, 16);
	}
	ifile.close();
	ofile.close();
	return 0;
}
\end{lstlisting}

\begin{lstlisting}[label=code:reflector,caption=Функция шифратора]
void AES::EncryptBlock(const unsigned char in[], unsigned char out[],
unsigned char key[]) {
	unsigned char state[4][Nb];
	unsigned int i, j, round;
	unsigned char* roundKeys = new unsigned char[4 * Nb * (Nr + 1)];
	
	KeyExpansion(key, roundKeys);
	
	for (i = 0; i < 4; i++) {
		for (j = 0; j < Nb; j++) {
			state[i][j] = in[i + 4 * j];
		}
	}
	
	AddRoundKey(state, roundKeys);
	
	for (round = 1; round <= Nr - 1; round++) {
		SubBytes(state);
		ShiftRows(state);
		MixColumns(state);
		AddRoundKey(state, roundKeys + round * 4 * Nb);
	}
	
	SubBytes(state);
	ShiftRows(state);
	AddRoundKey(state, roundKeys + Nr * 4 * Nb);
	
	for (i = 0; i < 4; i++) {
		for (j = 0; j < Nb; j++) {
			out[i + 4 * j] = state[i][j];
		}
	}
	delete[] roundKeys;
}
\end{lstlisting}

\begin{lstlisting}[label=code:enigma,caption=Функция расширения ключей]
void AES::KeyExpansion(const unsigned char key[], unsigned char w[]) {
	unsigned char temp[4];
	unsigned char rcon[4];
	
	unsigned int i = 0;
	while (i < 4 * Nk) {
		w[i] = key[i];
		i++;
	}
	
	i = 4 * Nk;
	while (i < 4 * Nb * (Nr + 1)) {
		temp[0] = w[i - 4 + 0];
		temp[1] = w[i - 4 + 1];
		temp[2] = w[i - 4 + 2];
		temp[3] = w[i - 4 + 3];
		
		if (i / 4 % Nk == 0) {
			RotWord(temp);
			SubWord(temp);
			Rcon(rcon, i / (Nk * 4));
			XorWords(temp, rcon, temp);
		}
		else if (Nk > 6 && i / 4 % Nk == 4) {
			SubWord(temp);
		}
		
		w[i + 0] = w[i - 4 * Nk] ^ temp[0];
		w[i + 1] = w[i + 1 - 4 * Nk] ^ temp[1];
		w[i + 2] = w[i + 2 - 4 * Nk] ^ temp[2];
		w[i + 3] = w[i + 3 - 4 * Nk] ^ temp[3];
		i += 4;
	}
}
\end{lstlisting}

\section{Тестирование реализации алгоритма}

Было проведено тестирование на следующих входных данных:

1. Входящая последовательность байтов:

00, 11, 22, 33, 44, 55, 66, 77, 88, 99, aa, bb, cc, dd, ee, ff

Зашифрованный текст:

69, c4, e0, d8, 6a, 7b, 04, 30, d8, cd, b7, 80, 70, b4, c5, 5a
	 
Расшифрованная последовательность байтов:

00, 11, 22, 33, 44, 55, 66, 77, 88, 99, aa, bb, cc, dd, ee, ff

2. Входящая последовательность байтов:

00, 11, 22, 33, 44, 55, 66, 77,
88, 99, aa, bb, cc, dd, ee, ff,
10, 11, 12, 13, 14, 15, 16, 17,
18, 19, 1a, 1b, 1c, 1d, 1e, 1f

Зашифрованная последовательность байтов:

69, c4, e0, d8, 6a, 7b, 04, 30, d8, cd, b7,
80, 70, b4, c5, 5a, 07, fe, ef, 74, e1, d5,
03, 6e, 90, 0e, ee, 11, 8e, 94, 92, 93

Расшифрованная последовательность байтов:

00, 11, 22, 33, 44, 55, 66, 77,
88, 99, aa, bb, cc, dd, ee, ff,
10, 11, 12, 13, 14, 15, 16, 17,
18, 19, 1a, 1b, 1c, 1d, 1e, 1f

Все тесты пройдены успешно.

\section*{Вывод}
В данном разделе были перечислены средства разработки, с помощью которых был реализованы алгоритм шифрования AES, приведена реализация алгоритма.


