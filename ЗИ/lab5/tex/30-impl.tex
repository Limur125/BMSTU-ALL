\chapter{ТЕХНОЛОГИЧЕСКАЯ ЧАСТЬ}


\section{Выбор средств реализации}

В качестве языка программирования для реализации данной лабораторной работы
использовался язык программирования С, так как он позволяет работать с файлами и
массивами. В качестве среды разработки использовалась Visual Studio.

\section{Реализация алгоритма Хаффмана}

В листингах~\ref{lst:create-server} представлена реализация алгоритма Хаффмана.

\captionsetup{justification=raggedright,singlelinecheck=off}
\begin{lstlisting}[label=lst:create-server, caption=Реализация алгоритма Хаффмана]
Node* generate_huffman_tree(const std::map <char, ll>& value) {
	std::vector<Node*> store = sort_by_character_count(value);
	Node* one, * two, * parent;
	sort(begin(store), end(store), sortbysec);
	if (store.size() == 1) {
		return combine(store.back(), nullptr);
	}
	while (store.size() > 2) {
		one = *(store.end() - 1); two = *(store.end() - 2);
		parent = combine(one, two);
		store.pop_back(); store.pop_back();
		store.push_back(parent);
		
		std::vector<Node*>::iterator it1 = store.end() - 2;
		while ((*it1)->count < parent->count && it1 != begin(store)) {
			--it1;
		}
		std::sort(it1, store.end(), sortbysec);
	}
	one = *(store.end() - 1); two = *(store.end() - 2);
	return combine(one, two);
}
\end{lstlisting}

\section{Тестирование}

\textbf{Тест с одной буквой}

Исходный размер файла = 8 байт

Размер сжатого файла = 6 байт

Исходный текст = 3131 3131 3131 3131

Сжатый текст = 6100 3101 ff00

\textbf{Тест с двумя буквами и частотой 3:1}

Исходный размер файла = 8 байт

Размер сжатого файла = 9 байт

Исходный текст = 3131 3132 3131 3132

Сжатый текст = 6101 3001 0162 0031 0011

\textbf{Тест с тремя буквами и частотой 1:1:1}

Исходный размер файла = 9 байт

Размер сжатого файла = 15 байт

Исходный текст = 3132 3331 3233 3132 3300

Сжатый текст = 6102 3101 0262 3030 0263 3130 8c01 0062
