\chapter{Аналитический раздел}
\label{cha:analysis}

\section{Алгоритм шифрования RSA}
RSA (Rivest-Shamir-Adleman) --- это криптографический алгоритм, используемый для шифрования и подписи данных.

В отличии от симметричных алгоритмов шифрования, имеющих всего один ключ для шифрования и расшифровки информации, в алгоритме RSA используется 2 ключа --- открытый (публичный) и закрытый (приватный).

В ассиметричной криптографии и алгоритме RSA, в частности, публичный и приватный ключи являются двумя частями одного целого и неразрывны друг с другом. Для шифрования информации используется открытый ключ, а для её расшифровки приватный.

Рассмотрим процедуру создания публичного и приватного ключей:

\begin{enumerate}
	\item[1.] Выбираем два случайных простых числа $p$ и $q$.
	\item[2.] Вычисляем их произведение: $N = p * q$.
	\item[3.] Вычисляем функцию Эйлера: $\varphi(N) = (p-1) * (q-1)$
	\item[4.] Выбираем число $e$ (обычно простое, но необязательно), которое меньше $\varphi(N)$ и является взаимно простым с $\varphi(N)$.
	\item[5.] Ищем число $d$, обратное числу $e$ по модулю $\varphi(N)$, т.е. остаток от деления $(d*e)$ и $\varphi(N)$ должен быть равен 1.
\end{enumerate}

После произведённых вычислений, у нас будут: $e$ и $n$ --- открытый ключ, $d$ и $n$ --- закрытый ключ.

Алгоритм RSA широко используется в криптографии для обеспечения конфиденциальности данных и аутентификации. Однако для его безопасной реализации необходимо выбирать достаточно большие ключи, чтобы обеспечить защиту от взлома при помощи методов факторизации.

\section{Алгоритм хеширования SHA1}
Алгоритм шифрования SHA-1 (Secure Hash Algorithm 1) является одним из членов семейства криптографических хеш-функций, разработанных для обеспечения безопасной хеширования данных. SHA-1 преобразует входные данные произвольной длины в фиксированный хеш-код длиной 160 бит (20 байт).

Шаги алгоритма SHA-1 включают в себя:
\begin{itemize}
	\item Инициализацию пяти 32-битных переменных, используемых для сохранения промежуточных результатов.
	\item Предварительную обработку входных данных, включая добавление бита заполнения и добавление длины сообщения.
	\item Разделение входных данных на блоки фиксированной длины и последовательное применение операций на каждом блоке, включая логические операции, сдвиги и битовые операции.
	\item Получение конечного 160-битного хеш-кода.
\end{itemize}

SHA-1 был широко использован в различных криптографических приложениях, однако с течением времени был выявлен ряд уязвимостей и стал уступать более современным алгоритмам хеширования, таким как SHA-256 или SHA-3, из-за возможности коллизий (ситуации, когда разным входным данным соответствует одинаковый хеш-код). В связи с этим, рекомендуется использовать более сильные алгоритмы хеширования в криптографических приложениях.

\section*{Вывод}
В данном разделе были рассмотрены алгоритмы RSA и SHA1.