\chapter{Технологический раздел}
\label{cha:impl}

\section{Средства реализации}

В качестве языка программирования для реализации данной лабораторной работы использовался язык программирования С~\cite{cplusplus}, так как он позволяет работать с файлами и массивами. В качестве среды разработки использовалась Visual Studio~\cite{vscode}.

\section{Реализация алгоритмов}

В листингах \ref{code:rsa}--\ref{code:sha} представлены реализации алгоритмов RSA и SHA1.

\begin{lstlisting}[label=code:rsa,caption=Функция шифрования]
static void bignum_pow_mod(bignum* result, bignum* base, bignum* expo, bignum* mod)
{
	bignum *a = bignum_alloc(), *b = bignum_alloc();
	bignum *tmp = bignum_alloc();
	
	bignum_copy(base, a);
	bignum_copy(expo, b);
	bignum_fromint(result, 1);
	
	while(!bignum_iszero(b))
	{
		if(b->data[0] & 1)
		{
			bignum_imultiply(result, a);
			bignum_imod(result, mod);
		}
		bignum_idivide_2(b);
		bignum_copy(a, tmp);
		bignum_imultiply(a, tmp);
		bignum_imod(a, mod);
	}
	bignum_free(a);
	bignum_free(b);
	bignum_free(tmp);
}
\end{lstlisting}

\begin{lstlisting}[label=code:sha,caption=Функция хеширования]
int calculate_sha1(struct sha* sha1, unsigned char* text, uint32_t length)
{
	unsigned int i, j;
	unsigned char* buffer;
	uint32_t bits;
	uint32_t temp, k;
	uint32_t lb = length * 8;
	
	bits = padded_length_in_bits(length);
	buffer = (unsigned char*)malloc((bits / 8) + 8);
	if (buffer == NULL)
	{
		printf("\nError allocating memory...");
		return 1;
	}
	memcpy(buffer, text, length);
	*(buffer + length) = 0x80;
	for (i = length + 1; i < (bits / 8); i++)
	*(buffer + i) = 0x00;
	*(buffer + (bits / 8) + 4 + 0) = (lb >> 24) & 0xFF;
	*(buffer + (bits / 8) + 4 + 1) = (lb >> 16) & 0xFF;
	*(buffer + (bits / 8) + 4 + 2) = (lb >> 8) & 0xFF;
	*(buffer + (bits / 8) + 4 + 3) = (lb >> 0) & 0xFF;
	sha1->digest[0] = 0x67452301;
	sha1->digest[1] = 0xEFCDAB89;
	sha1->digest[2] = 0x98BADCFE;
	sha1->digest[3] = 0x10325476;
	sha1->digest[4] = 0xC3D2E1F0;
	for (i = 0; i < ((bits + 64) / 512); i++)
	{
		for (j = 0; j < 80; j++)
		sha1->w[j] = 0x00;
		for (j = 0; j < 16; j++)
		{
			sha1->w[j] = buffer[j * 4 + 0];
			sha1->w[j] = sha1->w[j] << 8;
			sha1->w[j] |= buffer[j * 4 + 1];
			sha1->w[j] = sha1->w[j] << 8;
			sha1->w[j] |= buffer[j * 4 + 2];
			sha1->w[j] = sha1->w[j] << 8;
			sha1->w[j] |= buffer[j * 4 + 3];
		}
		for (j = 16; j < 80; j++)
		sha1->w[j] = (ROTL(1, (sha1->w[j - 3] ^ sha1->w[j - 8] ^ sha1->w[j - 14] ^ sha1->w[j - 16])));
		sha1->a = sha1->digest[0];
		sha1->b = sha1->digest[1];
		sha1->c = sha1->digest[2];
		sha1->d = sha1->digest[3];
		sha1->e = sha1->digest[4];
		for (j = 0; j < 80; j++)
		{
			if ((j >= 0) && (j < 20))
			{
				sha1->f = ((sha1->b) & (sha1->c)) | ((~(sha1->b)) & (sha1->d));
				k = 0x5A827999;
			}
			else if ((j >= 20) && (j < 40))
			{
				sha1->f = (sha1->b) ^ (sha1->c) ^ (sha1->d);
				k = 0x6ED9EBA1;
			}
			else if ((j >= 40) && (j < 60))
			{
				sha1->f = ((sha1->b) & (sha1->c)) | ((sha1->b) & (sha1->d)) | ((sha1->c) & (sha1->d));
				k = 0x8F1BBCDC;
			}
			else if ((j >= 60) && (j < 80))
			{
				sha1->f = (sha1->b) ^ (sha1->c) ^ (sha1->d);
				k = 0xCA62C1D6;
			}
			temp = ROTL(5, (sha1->a)) + (sha1->f) + (sha1->e) + k + sha1->w[j];
			sha1->e = (sha1->d);
			sha1->d = (sha1->c);
			sha1->c = ROTL(30, (sha1->b));
			sha1->b = (sha1->a);
			sha1->a = temp;
			temp = 0x00;
		}
		sha1->digest[0] += sha1->a;
		sha1->digest[1] += sha1->b;
		sha1->digest[2] += sha1->c;
		sha1->digest[3] += sha1->d;
		sha1->digest[4] += sha1->e;
		
		buffer = buffer + 64;
	}
	return 0;
}
\end{lstlisting}

\section{Тестирование реализации алгоритма}

Было проведено тестирование на следующих входных данных:

\begin{itemize}
	\item пустой файл;
	\item текстовый файл;
	\item файл формата jpg;
	\item архив формата zip.
\end{itemize}

\textbf{Для пустого файла:}

Generated digest = 156048787866750836371052389361754228214758992386

Encrypted SHA = 17633805474718278270253609790759353902590069560638221485698\\842800649636322228260579448361623151181869931903421073024919049779316618182042127312\\375760569671703065193471404441495723654235735540795221847107792797935305199108882157\\733856959251307392490224563446078546197345064195543458791559048774988

Decrypted Hash value = 156048787866750836371052389361754228214758992386

\textbf{Для тестового файла:}

Generated digest = 814791834312585864639464326153728085570323449260

Encrypted SHA = 42157081277406470013073850129045554791472746091058921771610\\723480583508617056222082184585159607204206613463065949844017030094304607497332007071\\105325619659848883673849560453947105225616603396579414783878943710868010171474956786\\905970047470367402468013765787507993531407072415817430021901405853277

Decrypted Hash value = 814791834312585864639464326153728085570323449260

\textbf{Для файла формата jpg:}

Generated digest = 1409020340804790199905952219604609752067879784200

Encrypted SHA = 19103206488314901193602721676026819143228791192128868501291\\492547749797308670542784759867840656331065119871125745909395805445444774130520325221\\994785352571009075261212034375387755975249077705109802221734282837422098353125431130\\603510124987891316601968660135414848729486594786332974691868670718731

Decrypted Hash value = 1409020340804790199905952219604609752067879784200

\textbf{Для файла формата zip:}

Generated digest = 473167284371037228757425872343191003776458840519

Encrypted SHA = 51357917249820376328272609690760465593193767998085452470290\\179336515816606072832197911566795998442523385227357758892870540437143734972750875097\\142066875795374111842899792560248716344596275105960648713571838881344549073726587203\\357436698659712994628877385078771476384888654081112202853247772599912

Decrypted Hash value = 473167284371037228757425872343191003776458840519

Для каждого из файлов был сформирован хэш. Хэш после дешифрования совпадает с хешом, поданным на вход функции дешифратору.

Все тесты пройдены успешно.

\section*{Вывод}
В данном разделе были перечислены средства разработки, с помощью которых был реализованы алгоритмы RSA и SHA1, приведена реализация алгоритмов.


