\documentclass[12pt]{report}
\usepackage[utf8]{inputenc}
\usepackage[russian]{babel}
%\usepackage[14pt]{extsizes}
\usepackage{listings}
\usepackage{graphicx}
\usepackage{amsmath,amsfonts,amssymb,amsthm,mathtools} 
\usepackage{pgfplots}
\usepackage{filecontents}
\usepackage{float}
\usepackage{comment}
\usepackage{indentfirst}
\usepackage{eucal}
\usepackage{enumitem}
\usepackage{pdfpages}
%s\documentclass[openany]{book}
\frenchspacing

\usepackage{indentfirst} % Красная строка

\usetikzlibrary{datavisualization}
\usetikzlibrary{datavisualization.formats.functions}

\usepackage{amsmath}


% Для листинга кода:
\lstset{ %
	language=c,                 % выбор языка для подсветки (здесь это С)
	basicstyle=\small\sffamily, % размер и начертание шрифта для подсветки кода
	numbers=left,               % где поставить нумерацию строк (слева\справа)
	numberstyle=\tiny,           % размер шрифта для номеров строк
	stepnumber=1,                   % размер шага между двумя номерами строк
	numbersep=5pt,                % как далеко отстоят номера строк от подсвечиваемого кода
	showspaces=false,            % показывать или нет пробелы специальными отступами
	showstringspaces=false,      % показывать или нет пробелы в строках
	showtabs=false,             % показывать или нет табуляцию в строках
	frame=single,              % рисовать рамку вокруг кода
	tabsize=2,                 % размер табуляции по умолчанию равен 2 пробелам
	captionpos=t,              % позиция заголовка вверху [t] или внизу [b] 
	breaklines=true,           % автоматически переносить строки (да\нет)
	breakatwhitespace=false, % переносить строки только если есть пробел
	escapeinside={\#*}{*)}   % если нужно добавить комментарии в коде
}


\usepackage[left=2cm,right=2cm, top=2cm,bottom=2cm,bindingoffset=0cm]{geometry}
% Для измененных титулов глав:
\usepackage{titlesec, blindtext, color} % подключаем нужные пакеты
\definecolor{gray75}{gray}{0.75} % определяем цвет
\newcommand{\hsp}{\hspace{20pt}} % длина линии в 20pt
% titleformat определяет стиль
\titleformat{\section}[hang]{\Huge\bfseries}{\thechapter\hsp\textcolor{gray75}{|}\hsp}{0pt}{\Huge\bfseries}


% plot
\usepackage{pgfplots}
\usepackage{filecontents}
\usetikzlibrary{datavisualization}
\usetikzlibrary{datavisualization.formats.functions}

\begin{document}
	%\def\sectionname{} % убирает "Глава"
	\thispagestyle{empty}
	\begin{titlepage}
		\noindent \begin{minipage}{0.15\textwidth}
			\includegraphics[width=\linewidth]{b_logo}
		\end{minipage}
		\noindent\begin{minipage}{0.9\textwidth}\centering
			\textbf{Министерство науки и высшего образования Российской Федерации}\\
			\textbf{Федеральное государственное бюджетное образовательное учреждение высшего образования}\\
			\textbf{~~~«Московский государственный технический университет имени Н.Э.~Баумана}\\
			\textbf{(национальный исследовательский университет)»}\\
			\textbf{(МГТУ им. Н.Э.~Баумана)}
		\end{minipage}
		
		\noindent\rule{18cm}{3pt}
		\newline\newline
		\noindent ФАКУЛЬТЕТ $\underline{\text{«Информатика и системы управления»}}$ \newline\newline
		\noindent КАФЕДРА $\underline{\text{«Программное обеспечение ЭВМ и информационные технологии»}}$\newline\newline\newline\newline\newline
		
		\begin{center}
			\noindent\begin{minipage}{1.1\textwidth}\centering
				\Large\textbf{Отчет по лабораторной работе №9}\newline
				\textbf{по дисциплине <<Функциональное и логическое}\newline
				\textbf{~~~программирование>>}\newline\newline
			\end{minipage}
		\end{center}
		
		\noindent\textbf{Тема} $\underline{\text{Использование правил в программе на Prolog.}}$\newline\newline
		\noindent\textbf{Студент} $\underline{\text{Золотухин А. В.~~~~~~~~~~~~~~~~~~~~~~~~~~~~~~~~~~~~~~~~~~~~~~~~~~~~~~~~~~~~~~~~~}}$\newline\newline
		\noindent\textbf{Группа} $\underline{\text{ИУ7-64Б~~~~~~~~~~~~~~~~~~~~~~~~~~~~~~~~~~~~~~~~~~~~~~~~~~~~~~~~~~~~~~~~~~~~~~~~~}}$\newline\newline
		\noindent\textbf{Оценка (баллы)} $\underline{\text{~~~~~~~~~~~~~~~~~~~~~~~~~~~~~~~~~~~~~~~~~~~~~~~~~~~~~~~~~~~~~~~~~~~~~~~~}}$\newline\newline
		\noindent\textbf{Преподаватели} $\underline{\text{Толпинская Н.Б., Строганов Ю. В.~~~~~~~~~~~~~~~~~~~~~~~~~~}}$\newline\newline\newline
		
		\begin{center}
			\vfill
			Москва~---~\the\year
			~г.
		\end{center}
	\end{titlepage}
\section*{Задание 1}
Создать базу знаний: «ПРЕДКИ», позволяющую наиболее эффективным способом (за меньшее количество шагов, что обеспечивается меньшим количеством предложений БЗ - правил), и используя разные варианты (примеры) одного вопроса, определить (указать: какой вопрос для какого варианта):
\begin{enumerate}
    \item по имени субъекта определить всех его бабушек (предки 2-го колена),
    \item по имени субъекта определить всех его дедушек (предки 2-го колена),
    \item по имени субъекта определить всех его бабушек и дедушек (предки 2-го
колена),
    \item по имени субъекта определить его бабушку по материнской линии (предки 2-го колена),
    \item по имени субъекта определить его бабушку и дедушку по материнской линии (предки 2-го колена).
\end{enumerate}
Минимизировать количество правил и количество вариантов вопросов. Использовать конъюнктивные правила и простой вопрос.

\chapter*{Решение}
\begin{lstlisting}
	name = string.
	sex = string.

PREDICATES
	parent(name, name, sex).
	grandParent(name, name, sex, sex).

CLAUSES
	parent(bestla, bolthorn, m).
	parent(odin, borr, m).
	parent(odin, bestla, f).
	parent(frigg, fjorginn, m).
	parent(frigg, fjorgyn, f).
	parent(thor, odin, m).
	parent(thor, jord, f).
	parent(heimdall, odin, m).
	parent(heimdall, nine, f).
	parent(tyr, odin, m).
	parent(baldr, odin, m).
	parent(baldr, frigg, f).
	parent(hed, odin, m).
	parent(hed, frigg, f).
	parent(loki, farabuti, m).
	parent(loki, laufeya, f).
	parent(fenrir, loki, m).
	parent(hel, loki, m).
	parent(jormungandr, loki, m).
	
	grandParent(CName, GPName, PSex, GPSex) :- 
		parent(CName, PName, PSex), parent(PName, GPName, GPSex).

GOAL
	grandParent(baldr, GrandParent, _, f).
	%grandParent(baldr, GrandParent, _, m).
	%grandParent(baldr, GrandParent, _, _).
	%grandParent(baldr, GrandParent, f, f).
	%grandParent(baldr, GrandParent, f, _).
\end{lstlisting}

Порядок формирования результата для 1-го вопроса:

\includepdf[pages=-]{reportparent.pdf}

\section*{Задание 2}
В одной программе написать правила, позволяющие найти:
\begin{enumerate}
	\item Максимум из двух чисел
	\begin{enumerate}
		\item без использования отсечения;
		\item с использованием отсечения.
	\end{enumerate}
	\item Максимум из трех чисел
	\begin{enumerate}
		\item без использования отсечения;
		\item с использованием отсечения.
	\end{enumerate}
\end{enumerate}

\chapter*{Решение}
\begin{lstlisting}
	domains
	num = integer
	
	predicates
	max2(num, num, num)
	max3(num, num, num, num)
	
	max2Cut(num, num, num)
	max3Cut(num, num, num, num)
	
	clauses
	max2(N1, N2, N2) :- N2 >= N1.
	max2(N1, N2, N1) :- N1 >= N2.
	
	max3(N1, N2, N3, N3) :- N3 >= N1, N3 >= N2.
	max3(N1, N2, N3, N2) :- N2 >= N1, N2 >= N3.
	max3(N1, N2, N3, N1) :- N1 >= N2, N1 >= N3.
	
	max2Cut(N1, N2, N2) :- N2 >= N1, !.
	max2Cut(N1, _, N1).
	
	max3Cut(N1, N2, N3, N3) :- N3 >= N1, N3 >= N2, !.
	max3Cut(N1, N2, _, N2) :- N2 >= N1, !.
	max3Cut(N1, _, _, N1).
	
	goal
	%max2(1, 2, Max).
	max3(1, 3, 2, Max).
	%max2Cut(1, 4, Max).
	%max3Cut(6, 4, 5, Max).
\end{lstlisting}

Порядок формирования результата для 2(a), 2(b):

\includepdf[pages=-]{reportmax.pdf}


\end{document}