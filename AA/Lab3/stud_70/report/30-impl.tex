\chapter{Технологическая часть}

В данном разделе будут приведены требования к программному обеспечению, средства реализации и листинги кода.

\section{Требования к программному обеспечению}

К программе предъявляется ряд требований.

На вход подаётся массив сравнимых элементов (целые числа). 

На выходе — тот же массив, но в отсортированном виде.


\section{Средства реализации}

В качестве языка программирования для реализации данной лабораторной работы был выбран язык программирования C\# \cite{sharplang}. 

Язык C\# является полностью объектно-ориентированным. 
Все необходимые библиотеки для реализации поставленной задачи являются стандартными.

Время работы алгоритмов было замерено с помощью функции clock() \cite{cpplangtime}.

\section{Сведения о модулях программы}
Программа состоит из пяти модулей.
\begin{enumerate}
	\item Programm.cs --- главный файл программы, в котором располагается код меню;
	\item BaseSort.cs --- файл с базовым классом алгоритмов.
	\item BubbleSort.cs, QuickSort.cs, CombSort.cs --- файлы с кодами алгоритмов
\end{enumerate}


\section{Реализация алгоритмов}
В листингах \ref{lst:quick}, \ref{lst:bubble}, \ref{lst:comb} представлены реализации алгоритмов сортировок (быстрой, пузырьком и расческой).

\begin{lstlisting}[label=lst:quick,caption=Класс быстрой сортировки]
internal class QuickSort<Type> : BaseSort<Type>
{
	public QuickSort(Comparator comp) : base(comp) { }
	public override Type[] Sort(Type[] array) 
	{
		return Sort(array, 0, array.Length - 1);
	}
	
	private Type[] Sort(Type[] array, int left, int right)
	{
		int i = left;
		int j = right;
		Type pivot = array[left];
		while (i <= j)
		{
			while (comp(array[i], pivot) < 0)
			i++;
			while (comp(array[j], pivot) > 0)
			j--;
			if (i <= j)
			{
				(array[j], array[i]) = (array[i], array[j]);
				i++;
				j--;
			}
		}
		
		if (left < j)
		Sort(array, left, j);
		if (i < right)
		Sort(array, i, right);
		return array;
	}
}
\end{lstlisting}

\begin{lstlisting}[label=lst:bubble,caption= Класс сортировки пузырьком]
internal class BubbleSort<Type> : BaseSort<Type>
{
	public BubbleSort(Comparator comp) : base(comp) { }
	public override Type[] Sort(Type[] array)
	{
		int n = array.Length;
		for (int i = 0; i < n - 1; i++)
		for (int j = 0; j < n - i - 1; j++)
		if (comp(array[j], array[j + 1]) > 0)
		(array[j + 1], array[j]) = (array[j], array[j + 1]);
		return array;
	}
}
\end{lstlisting}

\begin{lstlisting}[label=lst:comb,caption=Класс сортировки расческой]
internal class CombSort<Type> : BaseSort<Type>
{
	readonly int coef = 127;
	public CombSort(Comparator comp) : base(comp) { }
	
	public override Type[] Sort(Type[] array)
	{
		int gap = array.Length;
		bool swaps = true;
		
		while (gap > 1 || swaps)
		{
			gap = gap * 100 / coef;
			
			if (gap < 1)
			gap = 1;
			
			swaps = false;
			
			for (int i = 0; i + gap < array.Length; i++)
			{
				int igap = i + gap;
				
				if (comp(array[i], array[igap]) > 0)
				{
					(array[igap], array[i]) = (array[i], array[igap]);
					swaps = true;
				}
			}
		}
		return array;
	}
}
\end{lstlisting}

\section{Модульные тесты}

В таблице \ref{tbl:functional_test} приведены тесты для методов, реализующих алгоритмы сортировки. Тесты пройдены успешно.

\captionsetup{justification=raggedright, singlelinecheck=false}

\begin{table}[H]
		\caption{\label{tbl:functional_test} Модульные тесты}
	\begin{center}
			\begin{tabular}{|c|c|c|}
			\hline
			Входной массив & Ожидаемый результат & Результат \\ 
			\hline
			$[1,2,3,4]$ & $[1,2,3,4]$  & $[1,2,3,4]$\\
			$[5,4,3,2,1]$  & $[1,2,3,4,5]$ & $[1,2,3,4,5]$\\
			$[3,2,-5,0,1]$  & $[-5,0,1,2,3]$  & $[-5,0,1,2,3]$\\
			$[4]$  & $[4]$  & $[4]$\\
			$[]$  & $[]$  & $[]$\\
			\hline
		\end{tabular}
	\end{center}
\end{table}


\section*{Вывод}

Были реализованы все три алгоритма сортировок (быстрой, пузырьком и расческой). Выполнено модульное тестирование реализаций трех алгоритмов сортировки.
