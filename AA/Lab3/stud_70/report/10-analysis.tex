\chapter{Аналитическая часть}
В этом разделе будут представлены описания алгоритмов быстрой сортировки, сортировки %fixed
пузырьком и сортировки %fixed
расческой\cite{sort_algs}.

\section{Сортировка пузырьком}

В начале сравниваются первые два элемента списка. Если первый больше второго, они меняются местами. Если они уже в нужном порядке, то остаются как есть. Затем осуществляется переход к следующей паре элементов, сравниваются их значения и меняются местами при необходимости. Этот процесс продолжается до последней пары элементов в списке.

По достижении конца списка, процесс повторяется для каждого элемента снова. В этом случае необходимо обойти список $n$ раз.

Очевидно, что для оптимизации алгоритма нужно остановить его, когда закончится сортировка. Если бы элементы были отсортированы, то не пришлось бы их менять местами. Таким образом, всякий раз, когда меняются элементы, флаг устанавливается в $True$, чтобы повторить процесс сортировки. На каждой итерации флаг сбрасывается. Если перестановок не произошло, флаг останется $False$, и алгоритм остановится.


\section{Быстрая сортировка}

Общая идея алгоритма состоит в следующем.
\begin{enumerate}
	\item Выбрать из массива элемент, называемый опорным. Это может быть любой из элементов массива. От выбора опорного элемента не зависит корректность алгоритма, но в отдельных случаях может сильно зависеть его эффективность.
	\item Сравнить все остальные элементы с опорным и переставить их в массиве так, чтобы разбить массив на три непрерывных отрезка, следующих друг за другом: элементы, меньшие опорного, равные опорно-
	\newpage
	му и большие опорного (по значению ключа).
	\item Для отрезков, содержащих $"$меньшие$"$ и $"$большие$"$ значения, выполнить рекурсивно ту же последовательность операций, если длина отрезка больше единицы.
\end{enumerate}

На практике массив обычно делят не на три, а на две части: например, $"$меньшие опорного$"$ и $"$равные и большие$"$; такой подход упрощает алгоритм разделения.

\section{Сортировка расческой}

\textbf{Сортировка расческой} --- разновидность пузырьковой сортировки.

При пузырьковом алгоритме сравниваются постоянно два элемента. В сортировке расческой эти элементы берутся не соседними, а как бы по краям $"$расчески$"$ --- первый и последний. Расстояние между сравниваемыми элементами наибольшее из возможных, то есть, это максимальный размер расчески. Теперь уменьшая $"$расческу$"$ на единицу и начинается сравнение элементов находящихся ближе: первого и предпоследнего, второго и последнего.

Затем расстояние между сравниваемыми элементами снова уменьшается и сравниваются первый с перед предпоследним, второй с предпоследним, третий с последним. Дальнейшие итерации проводятся постепенно уменьшая размер $"$расчески$"$, то есть уменьшая расстояние между сравниваемыми элементами.

Первая длина $"$расчески$"$, то есть расстояние между сравниваемыми элементами, выбирается с учетом специального коэффициента --- 1,247. В начале сортировки расстояние между числами равно отношению размера массива и указанного числа. Затем, отсортировав массив с этим шагом, шаг делится на это число и выполняется новая итерация. Продолжаем выполнять действия до тех пор, пока разность индексов не достигнет единицы. В этом случае массив доупорядочивается обычным пузырьковым методом.


\section*{Вывод}

В данном разделе были описаны три алгоритма сортировки, а именно: быстрой, пузырьком и расческой. 



