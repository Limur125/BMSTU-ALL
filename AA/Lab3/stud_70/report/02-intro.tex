\chapter*{Введение}
\addcontentsline{toc}{chapter}{Введение}

Существует огромное количество вариаций сортировок. Эти алгоритмы необходимо
уметь сравнивать, чтобы выбирать наилучшим образом подходящие для конкретного случая. Алгоритм сортировки -- это алгоритм для упорядочивания элементов в массиве или списке. В случае, когда
элемент списка имеет несколько полей, поле, служащее критерием порядка, называется
ключом сортировки. 

Цель данной лабораторной: изучение и исследование трудоемкости алгоритмов сортировки. %TODO!!
%// и в заключении тоже долен быть отчёт о достижении цели с расшифровкой, что именно достигнуто (скопируйте цель и переформулируйте в пассивную форму/прошедшее время)
%// Далее ищите в тех-файлах TODO и fixed

Задачи данной лабораторной:

\begin{itemize}
	\item проанализировать и реализовать три различных алгоритма сортировки;
	\item рассчитать их трудоемкость;
	\item сравнить их временные характеристики экспериментально;
	\item на основании проделанной работы сделать выводы об эффективности реализаций рассмотренных алгоритмов сортировки.
	% fixed
\end{itemize}
