\chapter*{\hfill{\centering Заключение}\hfill}
\addcontentsline{toc}{chapter}{Заключение}

В ходе выполнения лабораторной работы были решены следующие задачи:

\begin{itemize}[label=---] 
	\item описаны методы решения.
	\item описана реализация, реализован метод.
	\item выбран класс данных, составлен набор данных.
	\item проведена параметризация метода на основании муравьиного алгоритма для выбранного класса данных.
	\item проведен сравнительный анализ двух методов.
	\item даны рекомендации о применимости метода решения задачи коммивояжера на основе муравьиного алгоритма.
\end{itemize}

Поставленная цель достигнута: изучен муравьиный алгоритм на материале решения задачи коммивояжёра.

В результате проведенных экспериментов были выявлены оптимальные параметры для метода на основе муравьиного алгоритма при выбранном классе эквивалентности: $\alpha = 0.5$, $\rho = 0.1$, $ t_{max} = 500$. 

Метод полного перебора следует использовать для матриц небольшого размера (до 8) и в случае, если необходимо получить точное решение. 
В остальных случаях муравьиный алгоритм является более эффективным по времени, если достаточно получить хорошее решение по выбранной метрике (например, отклонение длины маршрута, полученного методом на основе муравьиного алгоритма, от оптимальной длины маршрута).