\chapter{Технологическая часть}

В данном разделе будут приведены требования к программному обеспечению, средства реализации и листинги кода.

\section{Требования к программе}

Программа на вход получает матрицу смежности графа.

Выход программы: суммарная стоимость этого маршрута --- целое число и минимальный по стоимости маршрут --- последовательность целых чисел.

Алгоритм полного перебора должен возвращать кратчайший путь.

Программа должна производить параметризацию реализации метода на основе муравьиного алгоритма.

Программа должна замерять время выполнения реализаций метода полного перебора и метода на основе муравьиного алгоритма

\section{Средства реализации}

В качестве языка программирования для реализации данной лабораторной работы был выбран язык программирования C\# \cite{sharplang}, т.к. его средств достаточно для реализации поставленной задачи.

Время работы алгоритмов было замерено с помощью свойства \\TotalProcessorTime класса Process, которое возвращает объект TimeSpan, указывающий количество времени, потраченного процессом на загрузку ЦП \cite{cpplangtime}.

\section{Сведения о модулях программы}
Программа состоит из следующих модулей.
\begin{enumerate}
	\item Programm.cs --- главный файл программы, в котором располагается код меню.
	\item AntAlgorithm.cs --- файл с классом, реализующим муравьиный алгоритм.
	\item BruteForce.cs  --- файлы с классом, реализующим алгоритм полного перебора.
	\item Map.cs, Path.cs --- файлы классами маршрута и графа соответственно.
\end{enumerate}


\section{Реализация алгоритмов}
В листингах \ref{lst:ant}, \ref{lst:bf} представлены реализации муравьиного алгоритма и алгоритма полного перебора.
\captionsetup{justification=raggedright, singlelinecheck=false}

\begin{lstlisting}[label=lst:ant,caption=Реализация муравьиного алгоритма]
static class AntAlgorithm
{
	public static Path GetRoute(Map map, int maxTime, double alpha, double beta, double Q, double pho)
	{
		Random r = new Random();
		
		Path shortest = new Path(null, map, int.MaxValue);
		
		int count = map.N;
		double[,] pher = InitPheromone(0.1, count);
		
		for (int time = 0; time < maxTime; time++)
		{
			List<Ant> ants = InitAnts(map);
			double[,] deltaPher = InitPheromone(0, count);
			for (int i = 0; i < count - 1; i++)
				foreach (Ant ant in ants)
				{
					int curTown = ant.LastVisited();
					
					double sum = 0;
					for (int town = 0; town < count; town++)
						if (!ant.IsVisited(town))
						{
							double tau = pher[curTown, town];
							double eta = 1.0 / map[curTown, town];
							sum += Math.Pow(tau, alpha) * Math.Pow(eta, beta);
						}
		
					double check = r.NextDouble();
					int newTown = 0;
					for (; check > 0; newTown++)
						if (!ant.IsVisited(newTown))
						{
							double tau = pher[curTown, newTown];
							double eta = 1.0 / map[curTown, newTown];
							double chance = Math.Pow(tau, alpha) * Math.Pow(eta, beta) / sum;
							check -= chance;
						}
					newTown--;
					ant.VisitTown(newTown);
					deltaPher[curTown, newTown] += Q / map[curTown, newTown];
					deltaPher[newTown, curTown] += Q / map[newTown, curTown];
				}
			foreach (Ant ant in ants)
			{
				if (ant.GetDistance() < shortest.N)
					shortest = ant.GetPath();
			}
			Ant elite = new Ant(map, shortest.Way[0]);
			for (int i = 1; i < count; i++)
			{
				int newTown = shortest.Way[i];
				int curTown = elite.LastVisited();
				elite.VisitTown(newTown);
				deltaPher[curTown, newTown] += Q / map[curTown, newTown];
				deltaPher[newTown, curTown] += Q / map[newTown, curTown];
			}
			for (int k = 0; k < count; k++)
				for (int t = 0; t < count; t++)
				{
					pher[k, t] = (1 - pho) * pher[k, t] + deltaPher[k, t];
					pher[k, t] = pher[k, t] < 0.1 ? 0.1 : pher[k, t];
				}
		}
		return shortest;
	}
	
	private static List<Ant> InitAnts(Map map)
	{
		List<Ant> ants = new List<Ant>();
		for (int i = 0; i < map.N; i++)
			ants.Add(new Ant(map, i));
		return ants;
	}
	
	private static double[,] InitPheromone(double num, int size)
	{
		double[,] phen = new double[size, size];
		for (int i = 0; i < size; i++)
			for (int j = 0; j < size; j++)
				phen[i, j] = num;
		return phen;
	}
	
}
\end{lstlisting}

\begin{lstlisting}[label=lst:bf, caption=Реализация алгоритма полного перебора]
	static class BruteForce
	{
		public static Path GetRoute(Map map)
		{
			Path shortest = new Path(null, map, int.MaxValue);
			List<int> a = new List<int>();
			for (int i = 0; i < map.N; i++)
				a.Add(i);
			foreach (List<int> cur in GetAllRoutes(a, new List<int>()))
			{
				Path check = new Path(cur, map, -1);
				check.GetDistance();
				if (shortest.N > check.N)
					shortest = check;
			}
			return shortest;
		}
		
		private static IEnumerable<List<int>> GetAllRoutes(List<int> arg, List<int> awithout)
		{
			if (arg.Count == 1)
			{
				var result = new List<List<int>> { new List<int>() };
				result[0].Add(arg[0]);
				return result;
			}
			else
			{
				var result = new List<List<int>>();
				
				foreach (var first in arg)
				{
					var others0 = new List<int>(arg.Except(new int[1] { first }));
					awithout.Add(first);
					var others = new List<int>(others0.Except(awithout));
					
					var combinations = GetAllRoutes(others, awithout);
					awithout.Remove(first);
					
					foreach (var tail in combinations)
					{
						tail.Insert(0, first);
						result.Add(tail);
					}
				}
				return result;
			}
		}
	}
\end{lstlisting}

\section{Тестирование}

В таблице \ref{tbl:test} приведены тесты для методов, реализующих алгоритм полного перебора и муравьиный алгоритм. 
Тесты пройдены успешно.
При тестировании муравьиного алгоритма тесты прошли успешно, так как матрицы смежности маленькие, и поэтому муравьиный алгоритм возвращает такие же результаты, как и алгоритм полного перебора.

\begin{table}[H]
	\begin{center}
		
		\caption{\label{tbl:test} Тестирование методов}
		\begin{tabular}{|c@{\hspace{5mm}}|c@{\hspace{5mm}}|c@{\hspace{5mm}}|c@{\hspace{7mm}}|c@{\hspace{7mm}}|c@{\hspace{7mm}}}
			\hline
			Матрица смежности & Ожидаемый результат & Действительный результат \\ \hline

			$\begin{pmatrix}
				0 &  1 &  10 &  7\\
				1 &  0 &  1 &  2\\
				10 &  1 &  0 &  1\\
				7 &  2 &  1 &  0
			\end{pmatrix}$ &
			3 -- 0 1 2 3&
			3 -- 0 1 2 3\\

			\hline
			$\begin{pmatrix}
				0 &  3 &  5 &  7\\
				7 &  0 &  1 &  2\\
				5 &  1 &  0 &  1\\
				7 &  2 &  1 &  0
			\end{pmatrix}$ &
			8 -- 0 2 1 3&
			8 -- 0 2 1 3\\
			\hline
			$\begin{pmatrix}
				0 &  3 &  5\\
				3 &  0 &  1\\
				5 &  1 &  0
			\end{pmatrix}$ &
			4 -- 0 1 2&
			4 -- 0 1 2\\
			\hline
		\end{tabular}
	\end{center}
\end{table}

\section*{Вывод}

Были реализованы муравьиный алгоритм и алгоритм полного перебора.
