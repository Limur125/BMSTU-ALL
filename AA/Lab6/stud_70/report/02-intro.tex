\chapter*{\hfill{\centering Введение}\hfill}

\addcontentsline{toc}{chapter}{Введение}

Задача коммивояжёра — одна из самых известных задач комбинаторной оптимизации, заключающаяся в поиске самого выгодного маршрута, проходящего через указанные города хотя бы по одному разу с последующим возвратом в исходный город.

Задача коммивояжёра \cite{Mudrov} относится к числу транс вычислительных: уже при относительно небольшом числе городов (66 и более) она не может быть решена методом перебора вариантов никакими теоретически мыслимыми компьютерами за время, меньшее нескольких миллиардов лет.

Муравьиный алгоритм — один из эффективных полиномиальных алгоритмов для нахождения приближённых решений задачи коммивояжёра, а также решения аналогичных задач поиска маршрутов на графах.
Суть подхода заключается в анализе и использовании модели поведения муравьёв, ищущих пути от колонии к источнику питания, и представляет собой метаэвристическую оптимизацию.

Цель работы: изучить муравьиный алгоритм на материале решения задачи коммивояжёра.

Задачи лабораторной работы:
\begin{itemize}[label=---] 
	\item описать методы решения.
	\item описать реализацию, реализовать метод.
	\item выбрать класс данных, составить набор данных.
	\item провести параметризацию метода на основании муравьиного алгоритма для выбранного класса данных.
	\item провести сравнительный анализ двух методов.
	\item дать рекомендации о применимости метода решения задачи коммивояжера на основе муравьиного алгоритма.
\end{itemize}

