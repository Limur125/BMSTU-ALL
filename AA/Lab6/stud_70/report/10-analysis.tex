\chapter{Аналитическая часть}
В данном разделе содержится описание задачи коммивояжера и методы её
решения.


\section{Задача коммивояжера}
В общем случае задача коммивояжера может быть сформулирована следующим образом: найти самый выгодный (самый короткий, самый дешевый, и т.д.) маршрут, начинающийся в исходном городе и проходящий ровно один раз через каждый из указанных городов.

Проблему коммивояжера можно представить в виде модели на графе, то есть, используя вершины и ребра между ними. Таким образом, $M$ вершин графа соответствуют $M$ городам, а ребра $(i, j)$ между вершинами $i$ и $j$ — пути сообщения между этими городами. 
Каждому ребру $(i, j)$ можно сопоставить критерий выгодности маршрута $c_{ij} \geq 0$,который можно понимать как, например, расстояние между городами, время или стоимость поездки.  
В целях упрощения задачи и гарантии существования маршрута обычно считается, что модельный граф задачи является полностью связным, то есть, что между произвольной парой вершин существует ребро.

\section{Метод полного перебора}

Пусть  дано $M$ - число городов, $D$ - матрица смежности, каждый элемент которой - вес пути из одного города в другой. Существует метод грубой силы решения поставленной задачи, а именно полный перебор всех возможных маршрутов в заданном графе с нахождением минимального по весу. Этот метод гарантированно даст идеальное решение (глобальный минимум по весу). Однако стоит учитывать, что сложность такого алгоритма составляет $M!$ и время выполнения программы, реализующий такой подход, будет расти экспоненциально в зависимости от размеров входной матрицы.

\section{Муравьиный алгоритм}

На практике чаще всего необходимо получить решение как можно быстрее, при этом требуемое решение не обязательно должно быть наилучшем, был разработан ряд методов, называемых эвристическими, которые решают поставленную задачу за гораздо меньшее время, чем метод полного перебора.
В основе таких методов лежат принципы из окружающего мира, которые в дальнейшем могут быть формализованы.

Одним из таких методов является муравьиный метод \cite{Dorigo}. 
Он применим к решению задачи коммивояжера и основан на идее муравейника. 

Пусть у муравья есть 3 чувства:
\begin{itemize}[label=---]
	\item зрение (муравей может оценить длину ребра);
	\item обоняние (муравей может унюхать феромон - вещество, выделяемое муравьем, для коммуникации с другими муравьями);
	\item память (муравей запоминает свой маршрут).
\end{itemize}

Благодаря введению обоняния между муравьями возможен непрямой обмен информацией.\vspace{\baselineskip}

Введем вероятность $P_{k, ij}(t)$ выбора следующего города $j$ на маршруте муравьем $k$, который в текущий момент времени $t$ находится в городе $i$.
\begin{equation}\label{form:way} 
	p_{i,j}={\frac {(\tau _{i,j}^{\alpha })(\eta _{i,j}^{\beta })}{\sum (\tau _{i,j}^{\alpha })(\eta _{i,j}^{\beta })}}
\end{equation}
$
	\text{где} \\
	\tau _{i,j}	- \text{ феромон на ребре ij;} \\
	\eta _{i,j}	- \text{ привлекательность города j;} \\
	\alpha	- \text{параметр влияния длины пути;} \\
	\beta	- \text{параметр влияния феромона.}
$

Очевидно, что при $\beta = 0$ алгоритм превращается в классический жадный алгоритм, а при $\alpha = 0$ он быстро сойдется к некоторому субоптимальному решению. 
Выбор правильного соотношения параметров является предметом исследований, и в общем случае производится на основании опыта.

После того, как муравей успешно проходит маршрут, он оставляет на всех пройденных ребрах феромон, обратно пропорциональный длине пройденного пути:
\begin{equation}\label{form:add} 
	\Delta \tau _{k, ij}= {\frac{Q}{L_{k}}}
\end{equation}
$
	\text{где} \\
	Q - \text{количество феромона, переносимого муравьем;} \\
	L_{k} - \text{стоимость k-го пути муравья (обычно длина).}
$

После окончания условного дня наступает условная ночь, в течение которой феромон испаряется с ребер с коэффициентом $\rho$. 
Количество феромона на следующий день вычисляется по следующей формуле:
\begin{equation}\label{form:eva} 
	\tau _{i,j}(t+1)=(1-\rho )\tau _{i,j}(t)+\Delta \tau _{i,j}(t),
\end{equation}
$
	\text{где} \\
	\rho _{i,j} - \text{доля феромона, который испарится;} \\
	\tau _{i,j}(t) - \text{количество феромона на дуге ij;} \\
	\Delta \tau _{i,j}(t) - \text{количество отложенного феромона.}
$

Таким образом, псевдокод муравьиного алгоритма можно представить так:
\begin{enumerate}
	\item Ввод матрицы расстояний $D$, количества городов $M$;
	\item Инициализация параметров алгоритма — $\alpha$ $\beta$, $Q$, $tmax$, $\rho$;
	\item Инициализация ребер — присвоение <<привлекательности>> $\eta_{ij}$ и начальной концентрации феромона  $\tau_{start}$;
	\item	Размещение муравьев по городам;
	\item	Инициализация начального кратчайшего маршрута $L_{p}$ = null и определение длины кратчайшего маршрута $L_{min}$ = inf;
	\item	Цикл по времени жизни колонии $t = 1..t_{max}$;
	\begin{enumerate}
		\item	Цикл по всем муравьям $k = 1..M$;
		\begin{enumerate}
			\item	Построить маршрут $T_{k}(t)$ по формуле \eqref{form:way}  и рассчитать длину получившегося маршрута $L_{k}(t)$;
			\item	Обновить феромон на маршруте по формуле \eqref{form:add};
			\item	Если $L_{k}(t)$ < $L_{min}$, то $L_{min}$ = $L_{k}(t)$ и $L_{p} = T_{k}(t)$;
		\end{enumerate}
		\item	Конец цикла по муравьям;
		\item Цикл по всем ребрам графа;
		\begin{enumerate}
			\item Обновить следы феромона на ребре по формуле \eqref{form:eva};
		\end{enumerate}
		\item Конец цикла по ребрам;
	\end{enumerate}
	\item Конец цикла по времени;
	\item Вывести кратчайший маршрут $L_{p}$ и его длину $L_{min}$.
\end{enumerate}
\section{Модификация с элитными муравьями}
Одним из усовершенствований является введение в алгоритм так называемых «элитных муравьёв». 
В дополнение коррекции феромона согласно формуле \eqref{form:eva} дополнительно добавляется количество феромона, пропорциональное длине лучшего пути для всех его дуг следующим образом:

\begin{equation}
	\tau _{i,j}(t+1)=(1-\rho )\tau _{i,j}(t)+\Delta \tau _{i,j}(t)+n_e\Delta \tau _{i,j}^e(t),
\end{equation}
$
\text{где } n_e - \text{число элитных муравьёв.}
$

\begin{equation}
	\Delta \tau _{i,j}^e(t)=\frac{Q}{L_{best}}
\end{equation}
$
\text{где } L_{best} - \text{длина наилучшего в данный момент времени маршрута}.
$

Проходя ребра, входящие в короткие пути, муравьи с большей вероятностью будут находить еще более короткие пути. 
Таким образом, эффективной стратегией является искусственное увеличение уровня феромонов на самых удачных маршрутах. 
Для этого на каждой итерации алгоритма каждый из элитных муравьев проходит путь, являющийся самым коротким из найденных на данный момент.

\section*{Вывод}
Таким образом, существуют две группы методов для решения задачи коммивояжера - точные и эвристические. 
К точным относится метод полного перебора, к эвристическим - муравьиный метод.
Применение муравьиного алгоритма обосновано в тех случаях, когда необходимо быстро найти решение или когда для решения задачи достаточно получения первого приближения. 
В случае необходимости максимально точного решения используется алгоритм полного перебора.

