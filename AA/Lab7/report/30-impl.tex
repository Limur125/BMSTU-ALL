\chapter{Технологический раздел}

В данном разделе будут приведены средства реализации и листинги кода.

\section{Средства реализации}

В качестве языка программирования для реализации данной лабораторной работы был выбран язык программирования C\# \cite{sharplang}, т.к. его средств достаточно для реализации поставленной задачи.

Структура~ данных ~словарь~ была~ реализована~ с ~помощью ~класса \\Dictionary<string,~int>.

\section{Реализация алгоритмов}
В листингах \ref{lst:ss}, \ref{lst:bf}, \ref{lst:main} представлены реализации алгоритма поиска подстроки в строке и алгоритма полного перебора для поиска по словарю, а также методы, анализирующие вопрос.
\captionsetup{justification=raggedright, singlelinecheck=false}

\begin{lstlisting}[label=lst:ss,caption=Реализация алгоритма поиска подстроки в строке]
public int FindSubstring(string source, string sub)
{
	int res = -1;
	if (source.Length == 0 || sub.Length == 0)
		return res;
	for (int i = 0; i < source.Length - sub.Length + 1; i++)
		for (int j = 0; j < sub.Length; j++)
			if (sub[j] != source[i + j])
				break;
			else if (j == sub.Length - 1)
			{
				res = i;
				break;
			}
	return res;
}
\end{lstlisting}

\begin{lstlisting}[label=lst:bf, caption=Реализация алгоритма полного перебора поиска по словарю]
static Dictionary<string, int> Find(Dictionary<string, int> src, int l, int u)
{
	Dictionary<string, int> res = new Dictionary<string, int>();
	foreach(KeyValuePair<string, int> kv in src)
		if (kv.Value > l && kv.Value < u)
			res.Add(kv.Key, kv.Value);
	return res;
}
\end{lstlisting}
\begin{lstlisting}[label=lst:main, caption=Анализирующие вопрос методы]
static int IsQuestionValid(string question)
{
	SubstringFinder sf = new SubstringFinder();
	int pos;
	if ((pos = sf.FindSubstring(question.ToLowerInvariant(), "найди")) != -1) ;
	else if ((pos = sf.FindSubstring(question.ToLowerInvariant(), "какие")) != -1) ;
	else if ((pos = sf.FindSubstring(question.ToLowerInvariant(), "выбери")) != -1) ;
	else if ((pos = sf.FindSubstring(question.ToLowerInvariant(), "покажи")) != -1) ;
	else return -1;
	int ppos = pos;
	if ((pos = sf.FindSubstring(question, "видео")) != -1) { if (ppos > pos) return -1; }
	else if ((pos = sf.FindSubstring(question, "видосы")) != -1) { if (ppos > pos) return -1; }
	else return 0;
	ppos = pos;
	int[] cat = Category(question);
	int cpos = cat[0], cc = cat[1];
	if (cpos == -1 || cpos < ppos)
	return -1;
	if ((pos = sf.FindSubstring(question, "популярност")) != -1) { if (ppos > pos || cpos > pos) return -1; }
	else if ((pos = sf.FindSubstring(question, "просмотр")) != -1) { if (ppos > pos || cpos > pos) return -1; }
	else if ((pos = sf.FindSubstring(question, "посмотрел")) != -1)
	{
		if (ppos > pos || cpos < pos) return -1;
		ppos = pos;
		if ((pos = sf.FindSubstring(question, "количество людей")) != -1) { if (ppos > pos || pos < cpos) return -1; }
		else if ((pos = sf.FindSubstring(question, "количество человек")) != -1) { if (ppos > pos || cpos > pos) return -1; }
	}
	else return 0;
	return cc;
}
static int[] Category(string question)
{
	SubstringFinder sf = new();
	int[] res = new int[] { -1, 0 };
	if ((res[0] = sf.FindSubstring(question, "не очень маленьк")) > 0)
	res[1] = 2;
	else if ((res[0] = sf.FindSubstring(question, "очень маленьк")) > 0)
	res[1] = 1;
	else if ((res[0] = sf.FindSubstring(question, "немаленьк")) > 0) 
	res[1] = 6;
	else if ((res[0] = sf.FindSubstring(question, "маленьк")) > 0) 
	res[1] = 4;
	else if ((res[0] = sf.FindSubstring(question, "средн")) > 0)
	res[1] = 5;
	else if ((res[0] = sf.FindSubstring(question, "не очень больш")) > 0) 
	res[1] = 7;
	else if ((res[0] = sf.FindSubstring(question, "очень больш")) > 0)
	res[1] = 9;
	else if ((res[0] = sf.FindSubstring(question, "небольш")) > 0) 
	res[1] = 3;
	else if ((res[0] = sf.FindSubstring(question, "невероятно больш")) > 0)
	res[1] = 10;
	else if ((res[0] = sf.FindSubstring(question, "больш")) > 0) 
	res[1] = 8;
	return res;
}
\end{lstlisting}

\section{Тестирование}

В таблице \ref{tbl:test} приведены тесты для метода, реализующего алгоритм поиска строки в подстроке. 
Тесты пройдены успешно.

\begin{table}[H]
	\begin{center}
		
		\caption{\label{tbl:test} Тестирование методов}
		\begin{tabular}{|c|c|c|c|}
			\hline
			Строковый запрос& Искомая строка &Ожидаемый& Полученный  \\
			на поиск && результат &результат\\\hline
			Популярные видео & видео & 11 & 11\\\hline
			Найди видео с & &  &  \\
			немаленьким числом &&&\\
			просмотров.& просмотр & 33 & 33 \\\hline
			Популярные видео & просмотр & -1 & -1\\\hline
			
		\end{tabular}
	\end{center}
\end{table}

\section*{Вывод}

Были реализованы алгоритм поиска подстроки в строке и алгоритм полного перебора поиска по словарю, а также методы, анализирующие вопрос.
