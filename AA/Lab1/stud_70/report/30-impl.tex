\chapter{Технологическая часть}

В данном разделе будут приведены требования к программному обеспечению, средства реализации и листинги кода.

\section{Требования к ПО}

К программе предъявляется ряд требований:
\begin{itemize}
	\item у пользователя есть выбор алгоритма, или какой-то один, или все сразу, а также есть выбор тестирования времени;
	\item на вход подаются две строки на русском или английском языке в любом регистре;
	\item на выходе — искомое расстояние для выбранного метода (выбранных методов) и матрицы расстояний для матричных реализаций.
\end{itemize}

\section{Средства реализации}

В качестве языка программирования для реализации данной лабораторной работы был выбран язык программирования C\# \cite{sharplang}. 

Язык C\# является полностью объектно-ориентированным. 
Все необходимые библиотеки для реализации поставленной задачи являются стандартными.


Время работы алгоритмов было замерено с помощью функции clock() \cite{cpplangtime}.

\section{Сведения о модулях программы}
Программа состоит из семи модулей.
\begin{enumerate}
	\item Programm.cs -- главный файл программы, в котором располагается код меню;
	\item BaseAlgo.cs, BaseRecAlgo.cs -- файлы с базовыми классами алгоритмов.
	\item DamLevAlgo.cs, RecDamLevAlgo.cs, LevensteinAlgo.cs,
	
	RecCacheDamLevAlgo.cs -- файлы с кодами алгоритмов
\end{enumerate}


\section{Реализации алгоритмов}

 В листингах \ref{lst:lev}, \ref{lst:damlev}, \ref{lst:damlevrec}, \ref{lst:damlevcache} приведены реализации алгоритмов нахождения расстояния Левенштейна и Дамерау-Левенштейна.
\begin{lstlisting}[label=lst:lev,caption=Класс с алгоритмом нахождения расстояния Левенштейна.]
internal class LevensteinAlgo : BaseAlgo
{
	public override int[,] DoAlgorithm(string f_str, string s_str)
	{
		int[,] matrix = new int[s_str.Length + 1, f_str.Length + 1];
		for (int i = 0; i < f_str.Length + 1; i++)
		matrix[0, i] = i;
		for (int i = 0; i < s_str.Length + 1; i++)
		matrix[i, 0] = i;
		for (int i = 1; i < s_str.Length + 1; i++)
		for (int j = 1; j < f_str.Length + 1; j++)
		{
			int t1 = matrix[i, j - 1] + 1;
			int t2 = matrix[i - 1, j] + 1;
			int t3 = matrix[i - 1, j - 1] + (s_str[i - 1] == f_str[j - 1] ? 0 : 1);
			matrix[i, j] = Math.Min(t1, Math.Min(t2, t3));
		}
		return matrix;
	}
}
\end{lstlisting}

\begin{lstlisting}[label=lst:damlev,caption=Класс с алгоритмом нахождения расстояния Дамерау-Левенштейна.]
internal class DamLevAlgo : BaseAlgo
{
	public override int[,] DoAlgorithm(string f_str, string s_str)
	{
		int[,] matrix = new int[s_str.Length + 1, f_str.Length + 1];
		for (int i = 0; i < f_str.Length + 1; i++)
		matrix[0, i] = i;
		for (int i = 0; i < s_str.Length + 1; i++)
		matrix[i, 0] = i;
		for (int i = 1; i < s_str.Length + 1; i++)
		for (int j = 1; j < f_str.Length + 1; j++)
		{
			int t1 = matrix[i, j - 1] + 1;
			int t2 = matrix[i - 1, j] + 1;
			int t3 = matrix[i - 1, j - 1] + (s_str[i - 1] == f_str[j - 1] ? 0 : 1);
			int t4 = int.MaxValue;
			if (i > 1 && j > 1 && s_str[i - 1] == f_str[j - 2] && s_str[i - 2] == f_str[j - 1])
			t4 = matrix[i - 2, j - 2] + 1;
			matrix[i, j] = Math.Min(Math.Min(t1, t4), Math.Min(t2, t3));
		}
		return matrix;
	}
}
\end{lstlisting}

\begin{lstlisting}[label=lst:damlevrec,caption=Класс с алгоритмом нахождения расстояния Дамерау-Левенштейна с использованием рекурсии.]
internal class RecDamLevAlgo : BaseRecurAlgo
{
	public override int DoAlgorithm(string f_str, string s_str)
	{
		int n = f_str.Length, m = s_str.Length;
		if (n == 0 || m == 0)
		return Math.Abs(n - m);
		int t1 = DoAlgorithm(f_str[..(n - 1)], s_str[..m]) + 1;
		int t2 = DoAlgorithm(f_str[..n], s_str[..(m - 1)]) + 1;
		int t3 = DoAlgorithm(f_str[..(n - 1)], s_str[..(m - 1)]) + (s_str[^1] == f_str[^1] ? 0 : 1);
		int t4 = int.MaxValue;
		if (m > 1 && n > 1 && s_str[^1] == f_str[^2] && s_str[^2] == f_str[^1])
		t4 = DoAlgorithm(f_str[..(n - 2)], s_str[..(m - 2)]) + 1;
		return Math.Min(Math.Min(t1, t4), Math.Min(t2, t3));
	}
}
\end{lstlisting}

\begin{lstlisting}[label=lst:damlevcache,caption=Класс с алгоритмом нахождения расстояния Левенштейна с использованием рекурсии.]
internal class RecCacheDamLevAlgo : BaseRecurAlgo
{
	public override int DoAlgorithm(string f_str, string s_str)
	{
		int n = f_str.Length, m = s_str.Length;
		int[,] matrix = new int[f_str.Length + 1, s_str.Length + 1];
		
		static int recursive(string f_str, string s_str, int n, int m, int[,] matrix)
		{
			if (matrix[n, m] != -1)
			return matrix[n, m];
			
			if (n == 0)
			{
				matrix[n, m] = m;
				return matrix[n, m];
			}
			
			if (n > 0 && m == 0)
			{
				matrix[n, m] = n;
				return matrix[n, m];
			}
			int delete = recursive(f_str, s_str, n - 1, m, matrix) + 1;
			int add = recursive(f_str, s_str, n, m - 1, matrix) + 1;
			int change = recursive(f_str, s_str, n - 1, m - 1, matrix) + (s_str[m - 1] == f_str[n - 1] ? 0 : 1);
			int xch = int.MaxValue;
			if (m > 1 && n > 1 && s_str[m - 1] == f_str[n - 2] && s_str[m - 2] == f_str[n - 1])
			xch = recursive(f_str, s_str, n - 2, m - 2, matrix) + 1;
			
			matrix[n, m] = Math.Min(Math.Min(add, xch), Math.Min(delete, change));
			
			return matrix[n, m];
		}
		
		for (int i = 0; i < n + 1; i++)
		for (int j = 0; j < m + 1; j++)
		matrix[i, j] = -1;
		
		recursive(f_str, s_str, n, m, matrix);
		
		return matrix[n, m];
		
	}
}
\end{lstlisting}

\section{Функциональные тесты}
В таблице \ref{tabular:functional_test} приведены функциональные тесты для алгоритмов вычисления расстояния Левенштейна (в таблице столбец подписан "Левенштейн") и Дамерау — Левенштейна (в таблице - "Дамерау-Л."). Тестирование проводилось по методу черного ящика. Все тесты пройдены успешно.


\begin{table}[H]
	\begin{center}
		\caption{\label{tabular:functional_test} Функциональные тесты}
		\begin{tabular}{|c|c|c|c|c|}
			\hline
			\multicolumn{3}{|c|}{Входные данные}& \multicolumn{2}{c|}{Ожидаемый результат} \\
			\hline
			№&Строка 1&Строка 2&Левенштейн&Дамерау-Л. \\
			\hline
			1&скат&кот&2&2 \\
			\hline
			2&машина&малина&1&1 \\
			\hline
			3&дворик&доврик&2&1 \\
			\hline
			4&$\lambda$&университет&11&11 \\
			\hline
			5&сентябрь&$\lambda$&8&8 \\
			\hline
			8&тело&телодвижение&8&8 \\
			\hline
			9&ноутбук&планшет&7&7 \\
			\hline
			10&глина&малина&2&2 \\
			\hline
			11&рекурсия&ркерусия&3&2 \\
			\hline
			12&браузер&баурзер&2&2 \\
			\hline
			13&bring&brought&4&4 \\
			\hline
			14&moment&minute&4&4 \\ 
			\hline
			15&person&eye&5&5 \\
			\hline
			16&week&weekend&3&3 \\
			\hline 
			17&city&town&4&4 \\
			\hline
		\end{tabular}
	\end{center}
\end{table}


\section*{Вывод}

Были разработаны и протестированы реализации алгоритмов: нахождения расстояния Дамерау-Левенштейна рекурсивно, с заполнением матрицы и рекурсивно с заполнением матрицы, а также нахождения расстояния Левенштейна с матрицей.
