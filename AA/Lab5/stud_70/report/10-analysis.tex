\chapter{Аналитическая часть}
В данном разделе представлены теоретические сведения о рассматриваемых алгоритмах.

\section{Описание конвейерной обработки данных}

Вычислительный конвейер предполагает перемещение команд или данных по этапам цифрового вычислительного конвейера со скоростью, не зависящей от протяжённости конвейера (количества этапов), а зависит только от скорости подачи информации на конвейерные этапы \cite{conveyor}. 
Скорость задаётся временем, в течение которого один компонент вычислительной операции способен пройти каждый этап, то есть самой большой задержкой на этапе, который выполняет отдельная подзадача. 
Это также означает, что скорость вычислений задаётся и скоростью поступления информации на вход конвейера.

В случае, когда какая-либо функция выполняется за временной интервал $T$, но имеется возможность её деления на поочерёдное исполнение $N$ подфункций, то в идеальном конвейере, если вычисление этой функции повторяется многократно, возможно её исполнение за временной период $\frac{T}{N}$, то есть в $N$ раз увеличить производительность. 

\section{Алгоритм обратной трассировки лучей}
Из виртуального глаза через каждый пиксель изображения испускается луч и находится точка его пересечения с поверхностью сцены \cite{raytrace}.
Лучи, выпущенные из глаза называют первичными. 
Пусть, первичный луч пересекает некий объект 1 в точке H1.
Далее необходимо определить для каждого источника освещения, видна ли из него эта точка. 
Тогда к каждому точечному источнику света испускается теневой луч из точки H1.  
Если теневой луч находит пересечение с другими объектами, расположенными ближе чем источник света, значит, точка H1 находится в тени от этого источника. 
Иначе, освещение считается по некоторой локальной модели. 
При прохождении через частицу дыма цвет пикселя будет изменен, будет уменьшена интенсивность.

Таким образом, задача трассировки лучей может быть разделена на следующие подзадачи.

\begin{enumerate}
	\item Нахождение цвета пикселя алгоритмом обратной трассировки лучей.
	\item Изменение цвета пикселя с учётом тени.
	\item Изменение цвета пикселя с учётом дыма.
	\item Отрисовка пикселя на экран.
\end{enumerate}

\section{Требования к программе}
К программе предъявлены следующие требования.
\begin{enumerate}
	\item Заявки должны последовательно проходить конвейерные линии в заданном порядке.
	\item Для реализации конвейерных линий должны использоваться нативные потоки.
	\item Последняя заявка должна быть пустая, при её поступлении конвейерная линия должна сперва передать её на следующую очередь, а затем завершить свою работу.
	\item Конвейерная линия должна ожидать поступления нового элемента, если её очередь пуста.
	\item Программа должна записывать в лог-файл время начала и конца обработки заявки на каждой конвейерной линии. 
	\item Программа должна замерять время выполнения реализаций алгоритмов. 
\end{enumerate}

\section*{Вывод}

В данном разделе было представлено описание алгоритма обратной трассировки лучей и описание работы конвейера, выдвинуты требования к разрабатываемой программе. 



