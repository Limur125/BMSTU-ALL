\chapter*{\hfill{\centering Заключение}\hfill}
\addcontentsline{toc}{chapter}{Заключение}

В ходе выполнения лабораторной работы были решены следующие задачи:

\begin{itemize}[label=---]
	\item описана задача, для которой будет построена конвейерная обработка данных;
	\item описаны архитектура программы, принципы и алгоритмы обработки лент конвейера;
	\item реализована конвейерная система;
	\item собран лог событий с указанием времени их происхождения; 
	\item описаны и обоснованы полученные результаты.
\end{itemize}

Поставленная цель достигнута: получены навыки организации асинхронного взаимодействия потоков на примере моделирования конвейерной обработки данных.

В результате исследований можно прийти к выводу, что конвейерная реализация алгоритма обратной трассировки лучей эффективнее по времени, чем последовательная на 10 -- 12 \%.

Также было подтверждено, что если объектов на сцене меньше, чем 104, то последовательная реализация эффективнее по времени, чем конвейерная.