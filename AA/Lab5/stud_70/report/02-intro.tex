\chapter*{\hfill{\centering Введение}\hfill}

\addcontentsline{toc}{chapter}{Введение}

Конвейер — способ организации вычислений, используемый в современных процессорах и контроллерах с целью повышения их производительности (увеличения числа инструкций, выполняемых в единицу времени — эксплуатация параллелизма на уровне инструкций), технология, используемая при разработке компьютеров и других цифровых электронных устройств.

Сам термин «конвейер» пришёл из промышленности, где используется подобный принцип работы — материал автоматически подтягивается по ленте конвейера к рабочему, который осуществляет с ним необходимые действия, следующий за ним рабочий выполняет свои функции над получившейся заготовкой, следующий делает ещё что-то и т.д. 
Таким образом, к концу конвейера цепочка рабочих полностью выполняет все поставленные задачи, сохраняя высокий темп производства. 

Целью данной лабораторной работы является получение навыков организации асинхронного взаимодействия потоков на примере моделирования конвейерной обработки данных.

Для достижения данной цели необходимо решить следующие задачи.

\begin{itemize} [label=---]
	\item описание задачи, для которой будет построена конвейерная обработка данных;
	\item описание архитектуры программы, принципы и алгоритмы обработки лент конвейера;
	\item реализация конвейерной системы;
	\item сбор лога событий с указанием времени их происхождения; 
	\item описание и обоснование полученных результатов.
\end{itemize}


