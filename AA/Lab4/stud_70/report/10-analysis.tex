\chapter{Аналитическая часть}
В этом разделе будут представлено описание алгоритма обратной трассировки лучей.

\section{Алгоритм обратной трассировки лучей}
Из источников света испускается множество лучей (первичные лучи). Часть этих лучей не встретит никаких препятствий, а часть попадет на объекты. При попадании на них лучи преломляются и отражаются. При этом часть энергии луча поглотится. Преломленные и отраженные лучи образуют новое поколение лучей. Далее эти лучи опять же преломятся, отразятся и образуют новое поколение лучей. В конечном итоге часть лучей попадет в камеру и сформирует изображение. Это описывает работу прямой трассировки лучей. 

Метод обратной трассировки лучей позволяет значительно сократить перебор световых лучей. В этом методе отслеживаются лучи не от источников, а из камеры. Таким образом, трассируется определенное число лучей, равное разрешению картинки.

Так как для каждого пикселя на экране цвет можно вычислять независимо друг от друга, то для параллельной обратной трассировки лучей будет достаточно равным образом распределить пиксели между потоками.

\section*{Вывод}

В данном разделе было представлено описание алгоритма обратной трассировки лучей. 



