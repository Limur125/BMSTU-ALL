\chapter*{Введение}
\addcontentsline{toc}{chapter}{Введение}

Параллельные вычисления часто используются для увеличения скорости выполнения программ. Однако приемы, применяемые для однопоточных машин, для параллельных могут не подходить.

В данной лабораторной работе будет рассмотрено и реализовано параллельное программирование на примере задачи трассировки лучей.

Целью данной работы является изучение параллельных вычислений на материале трассировки лучей.

В рамках выполнения работы необходимо решить следующие задачи:
\begin{itemize}[label=---]
	\item описание основных положений базового алгоритма расчёта;
	\item применение изученных основ для реализации многопоточности на материале трассировки лучей;
	\item получение практических навыков параллельных вычислений на основе нативных потоков;
	\item проведение сравнительного анализа параллельной и однопоточной реализации алгоритма трассировки лучей;
	\item экспериментальное подтверждение различий во временной эффективности реализации однопоточной и многопоточной трассировки лучей;
	\item описание и обоснование полученных результатов;
\end{itemize}
