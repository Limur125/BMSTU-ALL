\chapter{Технологическая часть}

В данном разделе будут приведены требования к программному обеспечению, средства реализации и листинги кода.

\section{Требования к программному обеспечению}

На вход реализация алгоритма должна принимать две матрицы. 

На выход реализация алгоритма должна выдавать матрицу, которая является результатом умножения двух входных матриц.

\section{Средства реализации}

В качестве языка программирования для реализации данной лабораторной работы был выбран язык программирования C\# \cite{sharplang}. 

Язык C\# является полностью объектно-ориентированным. 
Все необходимые библиотеки для реализации поставленной задачи являются стандартными.

Время работы алгоритмов было замерено с помощью функции clock().

\section{Сведения о модулях программы}
Программа состоит из пяти модулей.
\begin{enumerate}
	\item Programm.cs - главный файл программы, в котором располагается код меню;
	\item BaseAlgo.cs - файл с базовым классом алгоритмов.
	\item Classic.cs, Vinograd.cs, OptimizedVinograd.cs - файлы с кодами алгоритмов
\end{enumerate}


\section{Реализация алгоритмов}
В листингах \ref{lst:classic}, \ref{lst:vino}, \ref{lst:optvino} представлены реализации алгоритмов умножения матриц (классический, Винограда и оптимизированный по Винограду).

\begin{lstlisting}[label=lst:classic,caption=Класс с реализацией классического алгоритма]
internal class Classic : BaseAlgo
{
	public override int[][] Multiply(int[][] A, int[][] B)
	{
		int Ar = A.Length;
		int Br = B.Length;
		
		if (Ar == 0 || Br == 0)
		return null;
		
		int Ac = A[0].Length;
		int Bc = B[0].Length;
		
		if (Ac != Bc)
		return null;
		
		int[][] C = new int[Ar][];
		for (int i = 0; i < Ar; i++)
		C[i] = new int[Bc];
		
		for (int i = 0; i < Ar; i++)
		for (int j = 0; j < Bc; j++)
		for (int k = 0; k < Ac; k++)
		C[i][j] += A[i][k] * B[k][j];
		
		return C;
	}
}
\end{lstlisting}

\begin{lstlisting}[label=lst:vino,caption= Класс с реализацией алгоритма Винограда]
internal class Vinograd : BaseAlgo
{
	public override int[][] Multiply(int[][] A, int[][] B)
	{
		int Ar = A.Length;
		int Br = B.Length;
		
		if (Ar == 0 || Br == 0)
		return null;
		
		int Ac = A[0].Length;
		int Bc = B[0].Length;
		
		if (Ac != Br)
		return null;
		
		int[] mulH = new int[Ar];
		int[] mulV = new int[Bc];
		
		int[][] C = new int[Ar][];
		for (int i = 0; i < Ar; i++)
		C[i] = new int[Bc];
		
		for (int i = 0; i < Ar; i++)
		for (int j = 0; j < Ac / 2; j++)
		mulH[i] = mulH[i] + A[i][j * 2] * A[i][j * 2 + 1];
		
		for (int i = 0; i < Bc; i++)
		for (int j = 0; j < Br / 2; j++)
		mulV[i] = mulV[i] + B[j * 2][i] * B[j * 2 + 1][i];
		
		for (int i = 0; i < Ar; i++)
		for (int j = 0; j < Bc; j++)
		{
			C[i][j] = -mulH[i] - mulV[j];
			for (int k = 0; k < Ac / 2; k++)
			C[i][j] = C[i][j] + (A[i][2 * k + 1] + B[2 * k][j]) * (A[i][2 * k] + B[2 * k + 1][j]);
		}
		
		if (Ac % 2 == 1)
		for (int i = 0; i < Ar; i++)
		for (int j = 0; j < Bc; j++)
		C[i][j] = C[i][j] + A[i][Ac - 1] * B[Ac - 1][j];
		
		return C;
	}
}
\end{lstlisting}

\begin{lstlisting}[label=lst:optvino,caption=Класс с реализацией оптимизированного алгоритма Винограда]
internal class OptimizedVinograd : BaseAlgo
{
    public override int[][] Multiply(int[][] A, int[][] B)
	{
		int Ar = A.Length;
		int Br = B.Length;
		
		if (Ar == 0 || Br == 0)
		return null;
		
		int Ac = A[0].Length;
		int Bc = B[0].Length;
		
		if (Ac != Br)
		return null;
		
		int[] mulH = new int[Ar];
		int[] mulV = new int[Bc];
		
		int[][] C = new int[Ar][];
		for (int i = 0; i < Ar; i++)
		C[i] = new int[Bc];
		
		int Ac2 = Ac >> 1;
		for (int i = 0; i < Ar; i++)
		for (int j = 0; j < Ac2; j++)
		{
			int j2 = j << 1;
			mulH[i] += A[i][j2] * A[i][j2 + 1];
		}
		
		for (int i = 0; i < Bc; i++)
		for (int j = 0; j < Ac2; j++)
		{
			int j2 = j << 1;
			mulV[i] += B[j2][i] * B[j2 + 1][i];
		}
		
		for (int i = 0; i < Ar; i++)
		for (int j = 0; j < Bc; j++)
		{
			int buf = -mulH[i] - mulV[j];
			for (int k = 0; k < Ac2; k++)
			{
				int k2 = k << 1;
				int k21 = k2 + 1;
				buf += (A[i][k21] + B[k2][j]) * (A[i][k2] + B[k21][j]);
			}
			C[i][j] = buf;
		}
		
		
		if (Ac % 2 == 1)
		for (int i = 0; i < Ar; i++)
		for (int j = 0; j < Bc; j++)
		{
			int Ac1 = Ac - 1;
			C[i][j] += A[i][Ac1] * B[Ac1][j];
		}
		
		return C;
	}
}
\end{lstlisting}

\section{Модульные тесты}

В таблице~\ref{tabular:test_rec} приведены тесты для методов, реализующих классический алгоритм умножения матриц, алгоритм Винограда и оптимизированный алгоритм Винограда. Тесты пройдены успешно.

\captionsetup{justification=raggedright,singlelinecheck=false}
\begin{table}[H]
	\caption{\label{tabular:test_rec} Тестирование методов}
	\begin{center}
		\begin{tabular}{|c|c|c|}
			\hline
			Матрица 1 & Матрица 2 &Ожидаемый результат \\ 
			\hline

			$\begin{pmatrix}
				1 & 1 & 1\\
				5 & 5 & 5\\
				2 & 2 & 2
			\end{pmatrix}$ &
			$\begin{pmatrix}
				1 \\
				1 \\
				1 
			\end{pmatrix}$ &
			$\begin{pmatrix}
				3 \\
				15 \\
				6
			\end{pmatrix}$ \\
			\hline

			$\begin{pmatrix}
				1 & 1 & 1
			\end{pmatrix}$ &
			$\begin{pmatrix}
				1 \\
				1 \\
				1
			\end{pmatrix}$ &
			$\begin{pmatrix}
				1 & 1 & 1\\
				1 & 1 & 1 \\
				1 & 1 & 1
			\end{pmatrix}$ \\
			\hline

			$\begin{pmatrix}
				1 & 1 \\
				1 & 1
			\end{pmatrix}$ &
			$\begin{pmatrix}
				1 & 1\\
				1 & 1
			\end{pmatrix}$ &
			$\begin{pmatrix}
				2 & 2\\
				2 & 2
			\end{pmatrix}$ \\
			\hline

			$\begin{pmatrix}
				2
			\end{pmatrix}$ &
			$\begin{pmatrix}
				2
			\end{pmatrix}$ &
			$\begin{pmatrix}
				4
			\end{pmatrix}$ \\
			\hline

			$\begin{pmatrix}
				1 & -2 & 3\\
				1 & 2 & 3\\
				1 & 2 & 3
			\end{pmatrix}$ &
			$\begin{pmatrix}
				-1 & 2 & 3\\
				1 & 2 & 3\\
				1 & 2 & 3
			\end{pmatrix}$ &
			$\begin{pmatrix}
				0 & 4 & 6\\
				4 & 12 & 18\\
				4 & 12 & 18
			\end{pmatrix}$\\
			\hline

			$\begin{pmatrix}
				1 & 2
			\end{pmatrix}$ &
			$\begin{pmatrix}
				1 & 2
			\end{pmatrix}$ &
			Неверный размер\\
			\hline
		\end{tabular}
	\end{center}
	
\end{table}

\section*{Вывод}
В данном разделе были реализованы все три алгоритма умножения матриц, а именно: классический алгоритм, алгоритм Винограда и оптимизированный алгоритм Винограда.