\chapter*{Введение}
\addcontentsline{toc}{chapter}{Введение}

Матрицы используются в вычислениях почти везде. Особенно умножение. Это довольно трудоемкий процесс даже при небольших размерах матриц, так как требуется большое количество операций умножения и сложения различных чисел. По этой причине человек озадачен проблемой оптимизации умножения матриц и ускорения процесса вычисления. Пусть сегодня и существуют системы, ускоряющие подобные операции аппаратно, это не меняет того факта, что эти операции стоит оптимизировать.

Таким образом, эффективное умножение матриц по времени и затратам ресурсов является актуальной проблемой для науки и техники.

Цель лабораторной работы --- изучение трех алгоритмов умножения матриц: классического, алгоритма Винограда и его оптимизации. Для того чтобы добиться этой цели, были поставлены следующие задачи:
\begin{itemize}
	\item изучить и реализовать классический алгоритм умножения матриц и алгоритм Винограда;
	\item оптимизировать работу алгоритма Винограда;
	\item выполнить сравнительный анализ трудоёмкостей алгоритмов;
	\item сравнить эффективность алгоритмов по времени и памяти.
\end{itemize}
