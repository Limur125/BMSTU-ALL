\chapter{Аналитическая часть}

\section{Классический алгоритм умножения}
Матрицей называют математический объект, эквивалентный двумерному массиву. Матрица является таблицей, на пересечении строк и столбцов находятся элементы матрицы. Количество строк и столбцов является размерностью матрицы.

Пусть даны две прямоугольные матрицы A и B размерности $m \times n$, $n \times q$
соответственно:\\
$$A =  \begin{bmatrix} 
	a_{11}& a_{12} &\ldots & a_{1n}\\ 
	a_{21}& a_{22} &\ldots & a_{2n}\\ 
	\vdots& \vdots &\ddots & \vdots\\ 
	a_{m1}& a_{m2} &\ldots & a_{mn} 
\end{bmatrix} $$\\	

$$B = \begin{bmatrix} 
	b_{11}& b_{12} &\ldots & b_{1q}\\ 
	b_{21}& b_{22} &\ldots & b_{2q}\\ 
	\vdots& \vdots &\ddots & \vdots\\ 
	b_{n1}& b_{n2} &\ldots & b_{nq} 
\end{bmatrix} $$\\

Тогда произведением матриц A и B называется матрица C размерностью  $m \times q$ 
$$
	C = \begin{bmatrix} 
		c_{11} & c_{12} & \cdots & c_{1q} \\
		c_{21} & c_{22} & \cdots & c_{2q} \\ 
		\vdots & \vdots & \ddots & \vdots \\ 
		c_{m1} & c_{m2} & \cdots & c_{mq}	
	\end{bmatrix},
$$
в которой:
\begin{equation}
	\label{matrix-mult}
	c_{ij} = \sum_{k=1}^{n} a_{ik}b_{kj} \;\;\; \left(\overline{i = 1 \ldots m};\;\overline{j = 1 \ldots q} \right)
\end{equation}.

\section{Алгоритм Винограда}
Исходя из равенства \ref{matrix-mult}, видно, что каждый элемент матрицы представляет собой скалярное произведение соответствующих строки и столбца исходных матриц. Такое умножение допускает предварительную обработку, позволяющую часть работы выполнить заранее \cite{Vinograd}.

Рассмотрим два вектора U и V таких, что:

\begin{equation}
	\label{u-def}
	U = A_{i} = (u_{1}, u_{2}, \ldots, u_{n}),
\end{equation}
где $U = A_{i}$ -- i-ая строка матрицы A,\\
$u_{k} = a_{ik}, \overline{k = 1 \ldots n}$ -- элемент i-ой строки k-ого столбца матрицы A.\\

\begin{equation}
	\label{v-def}
	V = B_{j} = (v_{1}, v_{2}, \ldots, v_{n}),
\end{equation}
где $V = B_{j}$ -- j-ый столбец матрицы B,\\
$v_{k} = b_{kj}, \overline{k = 1 \ldots n}$ -- элемент k-ой строки j-ого столбца матрицы B.

По определению их скалярное произведение равно:
\begin{equation}
	\label{uv-def}
	U \cdot V = u_{1}v_{1} + u_{2}v_{2} + u_{3}v_{3} + u_{4}v_{4}.
\end{equation}

Равенство \ref{uv-def} можно переписать в виде:

\begin{equation}
	\label{uv}
	U \cdot V = (u_{1} + v_{2})(u_{2} + v_{1}) + (u_{3} + v_{4})(u_{4} + v_{3}) - u_{1}u_{2} - u_{3}u_{4} - v_{1}v_{2} - v_{3}v_{4}.
\end{equation}

В равенстве \ref{uv-def} насчитывается 4 операции умножения и 3 операции сложения, в равенстве \ref{uv} насчитывается 6 операций умножения и 9 операций сложения. Однако выражение $- u_{1}u_{2} - u_{3}u_{4}$ используются повторно при умножении i-ой строки матрицы A на каждый из столбцов матрицы B, а выражение $- v_{1}v_{2} - v_{3}v_{4}$ - при умножении j-ого столбца матрицы B на строки матрицы A. Таким образом, данные выражения можно вычислить предварительно для каждых строк и столбцов матриц для сокращения повторных вычислений. В результате повторно будут выполняться лишь 2 операции умножения и 7 операций сложения (2 операции нужны для добавления предварительно посчитанных произведений).

\section{Оптимизированный алгоритм Винограда}
Для оптимизации алгоритма Винограда могут использоваться такие стратегии, как:
\begin{itemize}
	\item предварительные вычисления повторяющихся одинаковых действий;
	\item использование более быстрых операций при вычислении (такие, как смещение битов вместо умножения или деления на 2);
	\item использование аналогичных конструкций, уменьшающих трудоёмкость операций (к примеру, замена сложения с 1 на инкремент).
\end{itemize}

\section*{Вывод}

В данном разделе были описаны классический алгоритм умножения матриц, а также алгоритм Винограда и способы его оптимизации.

