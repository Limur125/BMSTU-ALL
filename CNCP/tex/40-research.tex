\chapter{ИССЛЕДОВАТЕЛЬСКАЯ ЧАСТЬ}

В данном разделе будут представлены характеристики компьютера, на котором проводилось исследование. Также в данном разделе будут приведено описание и постановка исследования. Будут приведены результаты сравнения разработанного веб-сервера с NGINX~\cite{nginx}.

\section{Технические характеристики устройства}

Технические характеристики устройства, на котором выполнялись измерения:

\begin{itemize}
	\item[---] процессор --- AMD Ryzen 5 4600H с Radeon Graphics 3.00 ГГц~\cite{proc};
	\item[---] операционная система --- Ubuntu 22.04 on WSL;
	\item[---] версия ядра --- 4.4.0-19041-Microsoft;
	\item[---] оперативная память --- 24 Гб;
	\item[---] количество логических ядер --- 12;
	\item[---] ёмкость диска --- 512 GB;
\end{itemize}

Во время тестирования ноутбук был включен в сеть питания и нагружен только приложениями встроенными в операционную систему и системой тестирования. Сторонние приложения запущены не были.

\section{Описание исследования}

Цель эксперимента --- сравнить разработанный веб-сервер с NGINX по времени обработки различных запросов. 

В ходе исследования анализируется время, затрачиваемое на отдачу файлов через разработанный веб-сервер и через NGINX.

Конфигурация NGINX представлена в листинге~\ref{lst:nginx}

\begin{lstlisting}[label=lst:nginx, caption=Конфигурация nginx]
events {
	worker_connections 1024;
}


http
{
	server {
		charset utf-8;
		listen 80;
		location / {
			root /mnt/c/zolot/CNCP/src;
			include /etc/nginx/mime.types;
			index index.html;
		}
	}
}
\end{lstlisting}

Для измерения времени обработки запросов использовалась утилита Apache Benchmark~\cite{ab}

\section{Результаты исследования}

В таблице~\ref{tbl:experiment1} приведены результаты нагрузочного тестирования. На каждом прогоне выполнялось 10000 запросов.

\begin{table}[H]
	\centering
	\captionsetup{justification=raggedleft, singlelinecheck=false}
	\caption{Среднее время (мс) обработки запроса через NGINX и через разработанный веб-сервер при обработке 10000 запросов}
	\label{tbl:experiment1}
	\begin{tabular}{|c|c|c|c|} 
		\hline
		Размер файла&Число конкурентных запросов&веб-сервер& NGINX\\\hline
		\multirow{3}{*}{612 байт}	&10&7.649&3.757\\\cline{2-4}
		&100&53.438&34.555\\\cline{2-4}
		&1000&354.575&214.196\\\hline
		\multirow{3}{*}{441 Кбайт}	&10&10.911 &6.701\\\cline{2-4}
		&100&92.381&60.937\\\cline{2-4}
		&1000&897.818&320.089\\\hline

		
	\end{tabular}
\end{table}

\section{Вывод}

Результаты тестирования показывают, что NGINX работает быстрее чем разработанный веб-сервер на 53\%--180\%. При обработке файла размером 612 байт при увеличении конкурентных запросов среднее время обработки запроса улучшилось на 50\% по сравнению с NGINX. При обработке файла 441 Кбайт можно наблюдать абсолютно противоположную ситуацию. При увеличении числа конкурентных запросов среднее время обработки запроса веб-сервера по сравнению с NGINX ухудшилось на 127\%.

Данные результаты можно объяснить тем, что разработанный сервер имеет более простую архитектуру и менее оптимизированную кодовую базу, чем NGINX. Также мультиплексор select, имеет ограничение на количество открытый файловых дескрипторов.


