\chapter{ТЕХНОЛОГИЧЕСКАЯ ЧАСТЬ}

В этом разделе будут выбраны средства реализации программного обеспечения. Будет выбран язык программирования. Будет приведена демонстрация работы программы.

\section{Выбор средств реализации}

В качестве языка программирования был выбран C в соответствии с заданием на курсовую работу. В качестве мультиплексора был выбран select в соответствии с заданием на курсовую работу. Для параллелизации обработки пользовательских запросов был использован пул потоков.

\section{Реализация веб-сервера}

Функция запуска сервера приведена в листинге~\ref{lst:create-server}.

\captionsetup{justification=raggedright,singlelinecheck=off}
\begin{lstlisting}[label=lst:create-server, caption=Запуск сервера]
int serve(uint16_t port)
{
	tpool_t *tm = tpool_create(THREAD_POOL_CAPACITY);
	
	int serverSocket = socket(AF_INET, SOCK_STREAM,	0);
	int reuse = 1;
	if (setsockopt(serverSocket, SOL_SOCKET, SO_REUSEADDR, &reuse, sizeof(reuse)) < 0) {
		errorHandler(ERROR, "setsockopt() failed", "", 0);
		return EXIT_FAILURE;
	}
	
	struct sockaddr_in serverAddress;
	serverAddress.sin_family = AF_INET;
	serverAddress.sin_port = htons(port);
	serverAddress.sin_addr.s_addr = htonl(INADDR_LOOPBACK);
	
	int bound = bind(serverSocket, (struct sockaddr *) &serverAddress, sizeof(serverAddress));
	if (bound != 0) {
		char portString[5 + MAX_PORT_DIGITS];
		sprintf(portString, "port %d", port);
		errorHandler(ERROR, "Socket not bound", portString, 0);
		return EXIT_FAILURE;
	}
	
	int listening = listen(serverSocket, BACKLOG);
	if (listening < 0) {
		printf("Error: The server is not listening.\n");
		return EXIT_FAILURE;
	}
	report(&serverAddress);
	int clientSocket;
	fd_set client_fds;
	while(1) {
		FD_ZERO(&client_fds);
		FD_SET(serverSocket, &client_fds);
		if (select(serverSocket + 1,&client_fds, NULL,NULL,NULL) == -1)
		{
			errorHandler(ERROR, strerror(errno),  NULL, 0);
			return EXIT_FAILURE;
		}
		if(FD_ISSET(serverSocket, &client_fds))
		{
			threadData_t *data = calloc(1, sizeof(threadData_t));
			LogData *log = calloc(1, sizeof(*log));
			log->clientAddr = calloc(1, sizeof(*log->clientAddr));
			log->req = NULL;
			clientSocket = acceptTCPConnection(serverSocket, log);
			if (clientSocket >=0 ) 
			{
				data->log = log;
				data->socket = clientSocket;
				tpool_add_work(tm, handleHTTPClient, data);
			}
		}
	}
	return 0;
}
\end{lstlisting}

Функция обработки соединения с клиентом представлена в листинге~\ref{lst:client-process}

\captionsetup{justification=raggedright,singlelinecheck=off}
\begin{lstlisting}[label=lst:client-process, caption=Обработка клиентского соединения]
void handleHTTPClient(void *arg)
{
	threadData_t* data = arg;
	int clientSocket = data->socket;
	LogData* log = data->log;
	char recvBuffer[INPUT_BUFFER_SIZE];
	memset(recvBuffer, 0, sizeof(char) * INPUT_BUFFER_SIZE);
	
	int nBytesReceived = recv(clientSocket, recvBuffer, sizeof(recvBuffer), 0);
	if (nBytesReceived < 0) {
		errorHandler(FORBIDDEN, "Failed to read request.", "", clientSocket);
		freeLog(log);
		free(data);
		close(clientSocket);
		return;	
	}
	if (nBytesReceived < INPUT_BUFFER_SIZE) {
		recvBuffer[nBytesReceived] = 0;
	} 
	
	char *response = NULL;
	char *filename = NULL;
	char *req = NULL;
	firstLine(recvBuffer, &req);
	char *tmpReq = realloc(log->req, (strlen(req) * sizeof(*(log->req))) + 1);
	if (tmpReq == NULL) 
	{
		ErrorSystemMessage("Memory allocation: realloc() failure.");
		free(response);
		free(filename);
		freeLog(log);
		free(data);
		close(clientSocket);
		return;	
	}
	log->req = tmpReq;
	strcpy(log->req, req);
	free(req);
	if (router(recvBuffer, clientSocket, &filename) != 0)
	{
		free(response);
		free(filename);
		freeLog(log);
		free(data);
		close(clientSocket);
		return;
	}
	
	ssize_t mimeTypeIndex = -1;
	if ((mimeTypeIndex = fileTypeAllowed(filename)) == -1) {
		free(filename);
		errorHandler(FORBIDDEN, "mime type", "Requested file type not allowed.", clientSocket);
		free(response);
		freeLog(log);
		free(data);
		close(clientSocket);
		return;
	}
	
	if (setResponse(filename, &response, OK, mimeTypeIndex, clientSocket, log) != 0) {
		free(response);
		freeLog(log);
		free(data);
		close(clientSocket);
		return;
	}
	
	send(clientSocket, response, strlen(response) + 1, 0);
	free(response);
	free(filename);
	logConnection(log);
	free(data);
	close(clientSocket);
}
\end{lstlisting}

\section{Поддерживаемые запросы}

Разработанный сервер может обработать GET и HEAD запросы. Когда клиент выполняет GET запрос, он получает в теле ответа запрошенный файл. Когда клиент выполняет HEAD запрос он в ответе получает только заголовки Content-Type и Content-Length. Ниже перечислены статусы ответов сервера, которые были реализованы.
\begin{itemize}
	\item[---] 200 — успешное завершение обработки запроса.
	\item[---] 403 — доступ к запрошенному файлу запрещён, запрошен неподдерживаемый тип файла.
	\item[---] 404 — запрашиваемый файл не найден.
	\item[---] 405 — неподдерживаемый HTTP-метод (POST, PUT и т.д.).
\end{itemize}

В соответствии с заданием веб-сервер может отдавать файлы следующих форматов:
\begin{itemize}
	\item[---] html (text/html);
	\item[---] css (text/css);
	\item[---] js (text/javascript);
	\item[---] png (image/png);
	\item[---] jpg (image/jpg);
	\item[---] jpeg (image/jpeg);
	\item[---] gif (image/gif);
	\item[---] svg (image/svg);
	\item[---] swf (application/x-shockwave-flash).
\end{itemize}
