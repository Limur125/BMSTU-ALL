\chapter*{ВВЕДЕНИЕ}
\addcontentsline{toc}{chapter}{ВВЕДЕНИЕ}

Веб-сервером~\cite{web-server} называется не только оборудование, но и обслуживающие веб-сервер программы. Также этим словом обозначается и то, и другое в совокупности.

Оборудование для веб-сервера представляет собой хранилище файлов сайта. На нем хранятся как отдельные страницы и файлы стилей, так и мультимедийные файлы – аудио, видео, графика и др. С сервера контент попадает на компьютер, с которого был отправлен запрос, и выводится в наглядном виде через браузер.
Программная составляющая веб-сервера позволяет осуществлять управление размещенными на нем данными, обеспечивает доступ пользователей. Минимально для этого требуется HTTP-сервер, то есть программа, которая может распознавать URL-адреса и работает на протоколе HTTP, который необходим для доступа к веб-странице.

Веб-серверы для публикации сайтов делятся на статические и динамические. Статические веб-серверы --- это «железо» с установленным на нем ПО для HTTP, которое направляет размещенные файлы в браузер в неизменном виде.

В динамических веб-серверах на статические веб-сервера устанавливается дополнительное программное обеспечение, чаще всего сервера приложения и базы данных. В таких серверах исходные файлы изменяются перед отправкой по HTTP.

Цель курсовой работы --- разработать статический веб-сервер. 

Для достижения поставленной цели необходимо решить следующие задачи:

\begin{itemize}
	\item[---] провести анализ предметной области и формализовать задачу;
	\item[---] спроектировать структуру программного обеспечения;
	\item[---] реализовать программное обеспечение, которое будет обслуживать контент, хранящийся во вторичной памяти;
	\item[---] провести нагрузочное тестирование и сравнить с распространёнными
	аналогами.
\end{itemize}

